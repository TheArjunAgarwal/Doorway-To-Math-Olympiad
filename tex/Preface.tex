\chapter*{Preface}
\section{Why does this book exist?}
As Austin Kleon wisely put it, ``Write the book you'd want to read.''\par
Well, here I am, crafting a math adventure that's as enjoyable as your favorite Netflix series 
(okay, maybe not that enjoyable, but close).\par\medskip
I've been diving into math contests since grade 3, where ``speed math'' was my jam. 
Fast forward to 8th grade, and I stumbled upon Olympiad math, and it was love at first equation.
My evenings turned into PRMO paper-solving sessions with friends (who were probably having more fun looking at me cursing the problem, 
than actually doing the problems). I even found myself deep in calculus on a lazy Sunday just because a tricky problem caught my eye.
\par
\medskip
However, my math journey, while thrilling, was a bit like a rollercoaster with no seatbelt 
--- wild and unpredictable. I'd study whatever piqued my curiosity, sometimes going way beyond what was needed. 
That led me down rabbit holes like exploring the mystical nature of $\pi$ in grade 4 and sneakily applying calculus in a non-calc test in grade 10.
 Oh, and let's not forget the countless distractions like sketch videos and attempting problems way beyond my level, 
 only to ponder the meaning of life afterward.\par

But wait, there's hope! This book is my attempt at a cure. 
It's the guide I desperately wished I had during my math escapades.

\section{Secret Sauce}
If you've ever stumbled upon the Wikipedia pages of IMO gold medalists, you might notice a common thread –-- one or both of their parents likely hold fancy STEM degrees. 
Now, before you roll your eyes, there's a simple reason behind this: success in math Olympiads often starts early and requires some serious elbow grease.\footnote{Polgár, L. (1989). Bring Up Genius! Budapest, H. v1.1 2017-07-31}\par
They say it takes a whopping 10,000 hours of practice to become an expert at something.\footnote{From Malcom Gladwell's Outliers: The story of success} 
Most of us didn't even consider real math until later in life. I mean, till grade 7, my idea of math was calculating the total of the menu of the restaurant before the starters arrived! 
So, hitting that 10,000-hour goal for Olympiad-level math is a bit of a stretch.\par\medskip
But hold on, there's more to it than just clocking in the hours. To improve, we need feedback, genuine feedback.
 Unfortunately, most of our well-meaning teachers, parents, and pals struggle to tackle even the simplest Olympiad questions. Case in point:\\
\begin{example}[IMO 2023, P1]
Determine all composite integers $n>1$ that satisfy the following property: if $d_1,d_2,\dots,d_k$ are all the positive divisors of $n$ with $1=d_1<d_2<\dots<d_k=n$, then $d_i$ divides $d_{i+1}+d_{i+2}$ for every $1\le i \le k-2$.
\end{example}
They mean well, but let's face it –-- their feedback can be as skewed as a wonky compass. Only the best teachers are willing to go the extra mile to meet your needs and help you out. 
But that's often not enough. Why, you ask?\par
Well, studies show that anesthesiologists get better at their job over time, while radiologists don't improve much from their initial performance. 
Why? Anesthesiologists get instant feedback, while radiologists have to rack their brains to figure out what went wrong weeks ago.\footnote{Goldberg, S. B., Rousmaniere, T., Miller, S. D., Whipple, J., Nielsen, S. L., Hoyt, W. T., \& Wampold, B. E. (2016). 
Do psychotherapists improve with time and experience? A longitudinal analysis of outcomes in a clinical setting. Journal of Counseling Psychology, 63(1), 1.}\par \medskip
The second secret sauce is a valid environment and deliberate practice. We can't just study random topics and solve random problems and expect to shine. 
We need a structured plan for learning and practice that pushes us just beyond our comfort zone.\footnote{Ericsson, K. A., Krampe, R. T., \& Tesch-Römer, C. (1993). The role of deliberate practice in the acquisition of expert performance. Psychological Review, 100(3), 363} Remember, it's like chess players vs. roulette players. 
After 10 years, a chess player is a chess wizard, while a roulette player \ldots well, they're just spinning a wheel and praying. 
We don't want to leave our Olympiad journey to chance, do we? \par \medskip
For the final secret sauce, we turn to the school system. 
Now, I know the school system has its flaws, but there's one thing it excels at: repetition. Repetition is the secret sauce to improvement. 
Chess grand masters can play the entire game blindfolded imagining the board and pieces in their head, but if you give them a random arrangement of pieces(which can't occur in a chess game) they are just as bad as remembering it as everyone else.\footnote{Chase, W. G., \& Simon, H. A. (1973). Perception in chess. Cognitive psychology, 4(1), 55-81}
 By repeating the same things, we can start seeing the patterns, making our brains better at dealing with them. We need to do he same with problems.\par\medskip

We've got the complete pedagogy of every chapter laid out in Appendix A for all the teachers out there who might be using this text or for the curious learners looking to understand our secret sauce, ingridient by ingridient.\par

\section{A small history}
This book's journey began back in class 9, during a car trip to my sister's wedding venue. However, it took a little hiatus for two years while I tackled my boards and sailed through 11th grade. 
But come 12th grade, I had a wild idea—teaching Olympiad math. And as I dove into teaching, my notes slowly evolved into what you're holding now.\par
As I write this latest version of the preface, the book is gearing up for its first round of updates—more hints, more questions, a dash more theory, and answers to select questions.
So, without further ado, let the math adventures begin!\\ \bigskip
Happy reading,\\
Warm regards,\\
Arjun Agarwal




