\chapter{Modular Arithmetic}
This chapter we will talk more about Congruence Modulo. It is used in number theory for variety of purposes, from telling time to literally securing passwords, through cryptography. \\
While our uses of them will be elementary but we'll form the basis of checking card numbers, pins and internet encryption. Also, they help in telling the time.\\
So with no waste of time, let's begin.
\section{Modulo Inverse}
We start with a strange claim, If $\gcd(a,p)=1$ then $\{0,a,2a,3a,\dots,(p-1)a\} \pmod{p}$ is pairwise distinct.\\
We can simply prove it by assuming, to the contrary, that two elements $ai$ and $aj$ are equal modulus $p$.\\
$\therefore ai \equiv aj \pmod{p} \iff a(i-j) = 0\pmod{p}\\
\iff p|a$ or $p|(i-j)$\\
The first case is not possible as $\gcd{a,p}=1, \therefore p|(i-j)$ which is false as $|i-j|< p$ as $i,j \in \{0,1,2,\dots,p-1\}$\\
Hence, the assumption is false and hence, no two elements must be equal.\\
This through pigeonhole principle(remember?), means that the sets $\{0,a,2a,3a,\dots,(p-1)a\} \equiv \{0,1,2,\dots,p-1\} \pmod{p}$\\
\begin{theorem}
[Equal Sets lemma]
$\{0,a,2a,3a,\dots,(p-1)a\} \equiv \{0,1,2,\dots,p-1\} \pmod{p}$ for $\gcd(a,p)=1$
\end{theorem}
We need to note that the sets are equal in the fact that they have the same elements. The position of the elements is clearly different.\\
Using the equal sets lemma, we can say for any integer $0 < b < p$, we can find an integer $x$ such that $ax \equiv b\pmod{p}$.
In particular, if $b = 1$, then $ax \equiv 1 (mod p)$. What this means is if $\gcd(a, p) = 1$, then there always exists a multiple of a which is $1$ modulo $p$. This allows us to define:
\begin{definition}
We say that the inverse of a number $a$ modulo $m$, refers to $b \leq p$ such that $ab \equiv 1 \pmod{m}$.\\
$b$ is commonly denoted as $a^{-1} \pmod{m}$
\end{definition}
All of the work above was done to reach this point as inverses are very useful because they finally enable us to divide under the modulo.\\
\begin{theorem}
If $b\not\equiv0$ (mod p), then
\[\frac{a}{b} \equiv a \cdot b^{-1} \pmod{p}\]
\end{theorem}
\section{Fermat's Little Theorem}
\begin{theorem}
    [Fermat's Little Theorem]
    If p is prime and does not divide a, then $a^p \equiv a \pmod{p}$, which can also be written as: $a^{p-1} \equiv 1 \pmod{p}$
\end{theorem}
Fermat's little theorem is, as we alreadt know, the more benevolent short form of FLT. The other one is Fermat's Last Theorem.\\
The proof for this FLT is much shorter and sweeter(the last theorem proof is about 130 pages and filled with very complex math).\\
\begin{proof}
    The most straightforward way to prove this theorem is by by applying the induction principle. We fix $p$ as a prime number. 
(B) $1^p \equiv 1 \pmod{p}$, is obviously true. \\
(S) Suppose the statement $a^p \equiv a \pmod{p}$ is true. Then, by the binomial theorem,\\
    \[(a+1)^p = a^p + {p \choose 1} a^{p-1} + {p \choose 2} a^{p-2} + \cdots + {p \choose p-1} a + 1.\]
    Note that $p$ divides into any binomial coefficient of the form ${p \choose k}$ for $1 \le k \le p-1$. This follows by the definition of the binomial coefficient as ${p \choose k} = \frac{p!}{k! (p-k)!}$; since $p$ is prime, then $p$ divides the numerator, but not the denominator.\\
    Taken $\mod p$, all of the middle terms disappear, and we end up with $(a+1)^p \equiv a^p + 1 \pmod{p}$. Since we also know that $a^p \equiv a\pmod{p}$, then $(a+1)^p \equiv a+1 \pmod{p}$, as desired
\end{proof}
\section{Euler's Totient Theorem}
45 years after death of Pierre De Fermat, influenced by his work, Leonhard Euler proved the following, which is also called Euler's generalization or the Fermat-Euler theorem. I however, have gone for the old name which is...
\begin{theorem}
    [Euler Totient Function]
    Given the general prime factorization of ${n} = {p}_1^{e_1}{p}_2^{e_2} \cdots {p}_m^{e_m}$, one can compute $\phi(n)$ using the formula\[\phi(n)= n\left(1-\frac{1}{p_1} \right) \left(1-\frac{1}{p_2} \right)\cdots \left(1-\frac{1}{p_m}\right).\]
    $\phi{n}$ represents the number of integers in the range $\{1,2,3\cdots{,n}\}$ which are relatively prime to $n$
\end{theorem}
\begin{theorem}
    [Euler's Totient Theorem]
    If ${a}$ is an integer and $m$ is a positive integer relatively prime to $a$, then ${a}^{\phi (m)}\equiv 1 \pmod {m}$.
\end{theorem}
Let's now prove both these things, 
\begin{proof}
    To derive the formula, let us first define the prime factorization of $n$ as $n =\prod_{i=1}^{m}p_i^{e_i} =p_1^{e_1}p_2^{e_2}\cdots p_m^{e_m}$ where the $p_i$ are distinct prime numbers. Now, we can use a PIE argument to count the number of numbers less than or equal to $n$ that are relatively prime to it.

First, let's count the complement of what we want (i.e. all the numbers less than or equal to $n$ that share a common factor with it). There are $\frac{n}{p_1}$ positive integers less than or equal to $n$ that are divisible by $p_1$. If we do the same for each $p_i$ and add these up, we get

\[\frac{n}{p_1} + \frac{n}{p_2} + \cdots + \frac{n}{p_m} = \sum^m_{i=1}\frac{n}{p_i}.\]
But we are obviously over counting. We then subtract out those divisible by two of the $p_i$. There are $\sum_{1 \le i_1 < i_2 \le m}\frac{n}{p_{i_1}p_{i_2}}$ such numbers. We continue with this PIE argument to figure out that the number of elements in the complement of what we want is

\[\sum_{1 \le i \le m}\frac{n}{p_i} - \sum_{1 \le i_1 < i_2 \le m}\frac{n}{p_{i_1}p_{i_2}} + \cdots + (-1)^{m+1}\frac{n}{p_1p_2\ldots p_m}.\]
This sum represents the number of numbers less than $n$ sharing a common factor with $n$, so

\[\phi(n) = n - \left(\sum_{1 \le i \le m}\frac{n}{p_i}- \sum_{1 \le i_1 < i_2 \le m}\frac{n}{p_{i_1}p_{i_2}} + \cdots + (-1)^{m+1}\frac{n}{p_1p_2\ldots p_m}\right)\]

\[\phi(n)= n\left(1 - \sum_{1 \le i \le m}\frac{1}{p_i} + \sum_{1 \le i_1 < i_2 \le m}\frac{1}{p_{i_1}p_{i_2}} - \cdots + (-1)^{m}\frac{1}{p_1p_2\ldots p_m}\right)\]

\[\phi(n)= n\left(1-\frac{1}{p_1} \right) \left(1-\frac{1}{p_2} \right)\cdots \left(1-\frac{1}{p_m}\right).\]
\end{proof}
And next we shall prove Euler's Totient Theorem.
\begin{proof}
    Consider the set of numbers $A = \{ n_1, n_2, ... n_{\phi(m)} \} \pmod{m}$ such that the elements of the set are the numbers relatively prime to $m$. We can prove that this set is the same as the set $B = \{ an_1, an_2, ... an_{\phi(m)} \} \pmod{m}$ where $\gcd(a, m) = 1$ as all elements of $B$ are relatively prime to $m$ and distinct, which by the pigeonhole principle means that $B$ has the same elements as $A.$ In other words, each element of $B$ is congruent to one of $A$. This means that $n_1 n_2 ... n_{\phi(m)} \equiv an_1 \cdot an_2 ... an_{\phi(m)} \pmod{m}$ $\implies$ $a^{\phi (m)} \cdot (n_1 n_2 ... n_{\phi(m)}) \equiv n_1 n_2 ... n_{\phi(m)} \pmod{m}$ $\implies$ $a^{\phi (m)} \equiv 1 \pmod{m}$ as desired. Note that dividing by $n_1 n_2 ... n_{\phi(m)}$ is allowed since it is relatively prime to $m$ and therefore has an inverse.
\end{proof}
\section{Wilson's Theorem}
A few years later, French mathmatician John Wilson read both Fermat and Euler's work and gave us:
\begin{theorem}
[Wilson's Theorem]
    if integer $p > 1$ , then $(p-1)! + 1$ is divisible by $p$ if and only if $p$ is prime. Essentially, $(p-1)!=-1 \pmod(p)$ for prime p.
\end{theorem}
Let's now prove this as well.\\
\begin{proof}
Let's define a polynomial with the roots $1,2,3,\dots p-1$:\\
\[g(x)=(x-1)(x-2)\dots(x-(p-1))\]\\
Also we can define h(x) such that:
\[h(x)=x^{p-1}-1\]
\[\therefore h(x)\pmod{p} = \{1,2,3,\dots p-1\}\]
This is a consequence of Fermat's Little Theorem.\\
Subtracting $g(x)$ from $h(x)$ will give us:\\
$f(x)=x^{p-1}-1-(x-1)(x-2)\dots(x-(p-1))$\\
On taking mod p, we will still have roots $1,2,3,\dots p-1$. This means, all terms of $f(x)$ are divisible by $p$, Which means the constant term is also divisible by $p$\\
$\therefore (p-1)!+1 \equiv 0 \pmod{p}\\
\therefore (p-1)! \equiv -1 \pmod{p}$
\end{proof}
This triad makes up the fundamental theorems of Modulo.\\
We can use them independently as in:\\
\begin{example}
     For how many integer values of $i$, $1 \leq i \leq 1000$, does there exist an integer $j$, $1 \leq j \leq 1000$, such that $i$ is a divisor of $2^j - 1$
\end{example}
\begin{example}
     How many prime numbers $p$ are there such that $29^p + 1$ is a multiple of $p$
\end{example}
\begin{example}
(ARML 2002) Let $a \in \mathbb{N}$ such that\\
\[1+\frac{1}{2}+\frac{1}{3}+\dots+\frac{1}{22}+\frac{1}{23}=\frac{a}{23!}\]
Find $a \pmod{13}$
\end{example}
Before you read the given solution, I advise you to try to guess which theorem goes to which question.\\
\begin{proof}
    [Solution]
    The first one, we can notice that $i$ can't be even as $2^j-1$ is odd for every $j$. For an odd $i$, we can be certain that $2^{\phi(i)}-1$ is divisible by $i$ using the Euler Totient theorem. Therefore, we have $500$ such $i$ from $0-100$.\\
    The second one follows from $29^p+1 \equiv 29+1 \equiv 30 \pmod{p}$ using FLT. As $p$ is a divisor of $29^p+1$, therefore $p|30$. As $p$ is prime, we can say $p=2,3,5$\\
    The third one can be solved by isolating $a=23!(1+\frac{1}{2}+\frac{1}{3}+\dots+\frac{1}{22}+\frac{1}{23})=23!+\frac{23!}{2}+\frac{23!}{3}+\dots+\frac{23!}{22}+\frac{23!}{23}$, at this stage we can notice that all are divisible by $13$ other than $\frac{23!}{13}$. This reduces the question to $\frac{23!}{13} \equiv 12!*(13*14*15*\dots*23 \equiv = 12!*10! = 12*\frac{11!}{11}\equiv 12*\frac{1}{11} \equiv 12*11^{-1}  \pmod{13}$\\
    We just used Wilson twice. We'll now finally use inverse of $11$ which is $6$ as $66 \equiv 1 \pmod{13}$. This means,\\
    $a \pmod{13} \equiv 12*6 \equiv 72 \equiv 7$
\end{proof}
We'll also see them used together in some while.
\section{Chinese Remainder Theorem}
\begin{theorem}
[Chinese Remainder Theorem]
    If a positive number $x$ satisfies the system of congruence's:
    \[
    \begin{aligned}
        x &\equiv a_1 \pmod{n_1} \\
        x &\equiv a_2 \pmod{n_2} \\
        &\vdots \\
        x &\equiv a_k \pmod{n_k}
    \end{aligned}
    \]
    where all $n_i$ are relatively prime, then $x \equiv A \pmod{N}$ where $N=n_1 \cdot n_2 \cdot n_3 \dots n_k$
\end{theorem}
After a bit of a dry stretch, we have a theorem which is so obvious that it entails no proof(You can easily prove it using induction if you are not convinced). However, It only tells you the upper bound of solution, you then will either need to make a Diophantine(we'll learn about them later) or do hit and trial to get the solution.\\
While this theorem is a bit weak right now, we'll see its true power in the constructions chapter.\\
The uses we see in this chapter looks like follows\\
\begin{example}
    (AIME 2012)
    For a positive integer $p$, define the positive integer $n$ to be $p$-safe if $n$ differs in absolute value by more than $2$ from all multiples of $p$. For example, the set of $10$-safe numbers is $\{ 3, 4, 5, 6, 7, 13, 14, 15, 16, 17, 23, \ldots\}$. Find the number of positive integers less than or equal to $10,000$ which are simultaneously $7$-safe, $11$-safe, and $13$-safe
\end{example}
\begin{proof}
    [Solution]
    We can notice that our number must be $3,4 \pmod{7}, 3,4,5,6,7,8,9 \pmod{11}, 3,4,5,6,7,8,9,10,11 \pmod{13}$. Which means that the number is defined for $\pmod{1001}$. We also need to notice that we'll have an unique solution for unique remainder we have in any of the cases.\\
    This leads to $96$ possible values per $1001$ integers. Hence we have $960$ such numbers in $1-10,010$. We just check for $10001- 10010$ to get that even $100006$ and $10007$ fulfill the condition. Therefore, we have $960-2=958$ such numbers.
\end{proof}
\begin{example}
Consider a number line consisting of all positive integers greater than 7. A hole punch traverses the number line, starting from 7 and
working its way up. It checks each positive integer $n$ and punches it if and only if $\binom{n}{7}$ is divisible by $12$. As the hole punch checks more and more numbers, the fraction of checked numbers that are punched approaches a limiting number $R$. Find $R$
\end{example}
\begin{proof}
    [Solution]
    Using CRT in reverse, $0 \pmod{12}$ can be broken down to $0 \pmod{3}$ and $0 \pmod{4}$. As
    \[\binom{n}{7}=\frac{n(n-1)(n-2)(n-3)(n-4)(n-5)(n-6)}{2^4 3^2 5 7}\]
    The mod $3$ condition forces us to look for $7$ consecutive numbers divisible by $27$ due to the $3^2$ in the denominator. This can obviously only occur if and only if one of the numbers is divisible by $9$. This means $n \equiv 0,1,2,3,4,5,6 \pmod{9}$\\
    The mod $4$ condition forces us to look for $7$ consecutive numbers divisible by $2^6$ due to the $2^4$ in the denominator. If $n$ is even, this is trivially true.\\
    If $n$ is odd, we have to either have $(n-1),(n-3),(n-5)$ divisible by $2$. We also either have $4| n-3$  or $4| (n-1), (n-5)$\\
    This means that in the first case, $16| n-3$ or in the second case $8| (n-1), (n-5)$, As everything in this condition perfectly divides $16$, we can convert all the condition for $16$ and say $n \equiv 0, 1, 2, 3, 4, 5, 6, 8, 9, 10, 12, 13, 14 \pmod{16}$\\
    Here we again use CRT to say that of every $9*16=144$ values of $n$, only $7*13=91$ satisfy the condition.\\
    Thus, $R=\frac{91}{144}$
\end{proof}
And just as a small taste of construction till we reach the actual chapter:\\
\begin{example}
Call a lattice point “visible” if the greatest common divisor of its coordinates is $1$. Prove that there exists a $100 \times 100$ square on the board none of whose points are visible. Generalize it for $n \times n$ square.
\end{example}
\begin{proof}
    Without loss of generality, we can let one of the corners be $(a,b)$\\
    As we are only proving the existence of something, we don't really care if a simpler case exists. We only care if a case exists. Here we can let $a \equiv 0 \pmod{p_1*p_2\dots*p_{100}}$  where $p_i$ is prime.\\
    This means $b \equiv 0 \pmod{p_1}, b+1 \equiv 0 \pmod{p_2} \dots b+99 \equiv 0 \pmod{p_{100}}$\\
    We can now declare $a+1= 0 \pmod{p_{101}*\dots*p_{200}}$\\
    This means $b \equiv 0 \pmod{p_{101}}, b+1 \equiv \pmod{p_{102}} \dots b+99 \equiv 0\pmod{p_{200}}$\\
    So on and so forth. We know that every term $b, b+1 \dots b+99$ exists using CRT as $p_i$ are prime.\\
    Hence, such a square defiantly exists.\\
    The generalization is left to you to solve.
\end{proof}
The square we made may or may not be the only one, that is a whole differently and much more difficult question. However, it defiantly exists. Such proves are called constructions, and we'll see more of them, and more techniques to solve them, later.
\begin{xcb}{Exercises}
\begin{enumerate}
\item(AMC 12) Let $S$ be a subset of $\{1,2,3,\dots,30\}$ with the property that no pair of distinct elements in $S$ has a sum divisible by $5$. What is the largest possible size of $S$?
\item (AMC 12) Let $k={2008}^{2}+{2}^{2008}$. What is the units digit of $k^2+2^k$?
\item (AIME) Let $a_n=6^{n}+8^{n}$. Determine the remainder upon dividing $a_ {83}$ by $49$.
\item (AMC 10) What is the hundreds digit of $2011^{2011}?$
\item(AIME) The positive integers $N$ and $N^2$ both end in the same sequence of four digits $abcd$ when written in base $10$, where digit $a$ is not zero. Find the three-digit number $abc$.
\item (AMC 10) An integer $N$ is selected at random in the range $1\leq N \leq 2020$ . What is the probability that the remainder when $N^{16}$ is divided by $5$ is $1$?
\item (PUMAC) If $p, q, r$ are primes with $pqr = 7(p + q + r)$, find $p + q + r$
\item What are the last two digits of the integer $17^{198}$?
\item (AMC 10) Let $a_1,a_2,\dots,a_{2018}$ be a strictly increasing sequence of positive integers such that\[a_1+a_2+\cdots+a_{2018}=2018^{2018}.\]What is the remainder when $a_1^3+a_2^3+\cdots+a_{2018}^3$ is divided by $6$?
\item How many of the first $2018$ numbers in the sequence $101, 1001, 10001, 100001, \dots$ are divisible by $101$?
\item (AMC 10) What is the remainder when $3^0 + 3^1 + 3^2 + \cdots + 3^{2009}$ is divided by 8?
\item (AMC 10)What is the greatest power of $2$ that is a factor of $10^{1002} - 4^{501}$?
\item (AMC 10) Let $n$ be a $5$-digit number, and let $q$ and $r$ be the quotient and the remainder, respectively, when $n$ is divided by $100$. For how many values of $n$ is $q+r$ divisible by $11$?
\item (AMC 10) Let $N=123456789101112\dots4344$ be the $79$-digit number that is formed by writing the integers from $1$ to $44$ in order, one after the other. What is the remainder when $N$ is divided by $45$?
\item (AMC 10) A palindrome between $1000$ and $10, 000$ is chosen at random. What is the probability
that it is divisible by $7$?
\item (Purple Comet 2013) There is a pile of eggs. Joan counted the eggs, but her count was off by $1$ in the ones place. Tom counted in the eggs, but his count was off by $1$ in the tens place. Raoul counted the eggs, but his count was off by $1$ in the hundreds place. Sasha, Jose, Peter, and Morris all counted the eggs and got the correct count. When these seven people added their counts together, the sum was $3162$. How many eggs were in the pile?
\item (Mandelbrot 2008) Determine the smallest positive integer $m$ such that $m^2 + 7m + 89$ is a multiple of $77$.
\item  Find some positive multiple of $21$ has $241$ as its final three digits.
\item Show that if $n$ is an integer greater than $1$, then n does not divide $2^n - 1$.
\item . Let n be a positive integer such that $n + 1$ is divisible by $24$. Prove that the sum of the divisors of $n$ is divisible by $24$
\item Do there exist $2004$ consecutive integers such that each is divisible by a perfect cube bigger than $1$?
\item . Find all primes of the form $n^4 + 4$.
\item Prove that $3^n-2^n$ is not divisible by $n$.

\end{enumerate}
\end{xcb}
