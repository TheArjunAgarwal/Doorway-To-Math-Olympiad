\begin{xcb}{Exercises}
\begin{enumerate}
\item(IMO/4 2000) A magician has one hundred cards numbered $1$ to $100$. He puts them into three boxes, a red one, a white one and a blue one, so that each box contains at least one card. A member of the audience selects two of the three boxes, chooses one card from each and announces the sum of the numbers on the chosen cards. Given this sum, the magician identifies the box from which no card has been chosen. How many ways are there to put all the cards into the boxes so that this trick always works?\\
\item (Japan 2009)  Let $N$ be a positive integer. Suppose that a collection of integers was written in a blackboard so that the following properties hold:\\
- each written number k satisfies $1 \leq k \leq N$;\\
- each k with $1 \leq k \leq N$ was written at least once;\\
- the sum of all written numbers is even.\\
Show that it is possible to label some numbers with $\heartsuit$ and the rest with $\spadesuit$ so that the sum of all numbers with $\heartsuit$ is the same as the sum of all numbers with $\spadesuit$
\item (Ukraine 1999)Two players alternately write integers on a blackboard as follows: the first player writes $a_1$ arbitrarily, then the second player writes $a_2$ arbitrarily, and thereafter a player writes a number that is equal to the sum of the two preceding numbers. The player after whose move the obtained sequence contains terms such that $a_i-a_j$ and $a_{i+1}-a_{j+1}~(i\ne j)$ are divisible by $1999$, wins the game. Which of the players has a winning strategy?
\end{enumerate}
\end{xcb}