%Add [H] to tables and images.
\documentclass{maaprb}
%    If you need symbols beyond the basic set, uncomment this command.
\usepackage{amssymb}
\usepackage{float}
\usepackage{tcolorbox}
\usepackage{mdframed}
\usepackage{tikz}
\usepackage{graphicx}
\usepackage{float}
\usepackage{booktabs}
\usepackage{epigraph}
\usepackage{hyperref}
\usepackage{natbib}
\usepackage{forloop}
\usepackage{pgfmath}
\usepackage{etoolbox}
\usepackage{ifthen}
\usepackage{answers}
\usepackage{cancel}
\usepackage{draftwatermark}
% Watermark
\SetWatermarkLightness{ 0.9 }
\SetWatermarkText{DRAFT}
\SetWatermarkScale{2}
%Expected Value
\newcommand{\E}[1]{\mathbb{E}({#1})}
%LCM command
\DeclareMathOperator*{\lcm}{lcm}
%true and false
\DeclareMathOperator*{\true}{true}
\DeclareMathOperator*{\false}{false}
% Points commad
\newcommand{\points}[1]{[$#1 \star$]}
% Setup counters
\newcounter{hindex}\setcounter{hindex}{0}
\newcounter{hintcounter}\setcounter{hintcounter}{0}
% Define \addhint and \gethint
\newcommand\addhint[1]{%
	\stepcounter{hintcounter}%
	\ref{hint:\thehintcounter}%
	\expandafter\gdef\csname hintlist\thehintcounter\endcsname{#1}%
}
\newcommand\gethint[1]{%
	\item \csname hintlist#1\endcsname \label{hint:#1}
}
\newenvironment{hint}{\footnotesize \normalfont \textbf{Hints}:}{\hspace{-0.5ex}}
\pgfmathsetseed{65536} % or any other number: sets the random seed

\newtheorem{theorem}{Theorem}[chapter]
\newtheorem{lemma}[theorem]{Lemma}

\theoremstyle{definition}
\newtheorem{definition}[theorem]{Definition}
\newtheorem{example}[theorem]{Example}
\newtheorem{xca}[theorem]{Exercise}

\theoremstyle{remark}
\newtheorem{remark}[theorem]{Remark}

\numberwithin{section}{chapter}
\numberwithin{equation}{chapter}
\setcounter{chapter}{-3}

\setcounter{chapter}{0}
\title{Doorway to Math Olympiads}
\author{Arjun Agarwal}
\begin{document}
\frontmatter
\maketitle
\epigraph{The study of mathematics, like the Nile, begins in minuteness but ends in magnificence.}{\textit{Charles Caleb Colton}}
\epigraph{The essence of mathematics is not to make simple things complicated, but to make complicated things simple.}{S. Gudder}
\pagebreak

\begin{center}
\copyright\ 2023 Arjun Agarwal \\
Text licensed under CC-by-SA-4.0 \\
This is (still!) an incomplete draft. \\
Please send corrections, comments, pictures of kittens, etc. \\
to \href{mailto:anulick0@gmail.com}{anulick0@gmail.com} \\
Last updated \today.
\end{center}
\pagebreak

\thispagestyle{empty}
\vspace*{13.5pc}
\begin{center}
{\em To Ashu Agarwal Ma'am, Aparna Thakur Ma'am and Sylvia Marcarhanas Ma'am, who made me fall in love with mathematics.} 
\end{center}
\pagebreak


\tableofcontents



\mainmatter

\part{Combinatorics}
\input{1PnC/ConstCount}
\input{1PnC/casecomp}
\input{1PnC/guess}
\input{1PnC/starbars}
\input{1PnC/geocount}

\part{Down the Rabbit Hole}
\input{2rabbithole/powerup}
\input{2rabbithole/graphtheory}
\input{2rabbithole/countmethods}

\part{Algebra}
\input{3alg/algman.tex}
\input{3alg/ineq.tex}
\input{3alg/seqseries.tex}

\part{The Red Pill}
\input{4redpill/calc1.tex}
\input{4redpill/calc2.tex}
\input{4redpill/linalg.tex}
\input{4redpill/ineqrev.tex}

\part{Number Theory}
\input{5nt/ntintro.tex}
\input{5nt/modarith.tex}
\input{5nt/funceq.tex}
\input{5nt/dioeq.tex}

\part{The Number's Awaken}
\input{6ntadv/bazooka.tex}
\input{6ntadv/const.tex}

\backmatter

\appendix
\part{Appendix}
\input{Appendix/hints.tex}
\input{Appendix/borrow.tex}
\input{Appendix/sources.tex}
\clearpage
\end{document}
