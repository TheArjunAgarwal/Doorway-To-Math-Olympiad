%Add [H] to tables and images.
\documentclass{maaprb}
%    If you need symbols beyond the basic set, uncomment this command.
\usepackage{amssymb}
\usepackage{float}
\usepackage{tcolorbox}
\usepackage{mdframed}
\usepackage{tikz}
\usepackage{graphicx}
\usepackage{float}
\usepackage{booktabs}
\usepackage{epigraph}
\usepackage{hyperref}
\usepackage{natbib}
\usepackage{forloop}
\usepackage{pgfmath}
\usepackage{etoolbox}
\usepackage{ifthen}
\usepackage{answers}
\usepackage{cancel}
\usepackage{draftwatermark}
% Watermark
\SetWatermarkLightness{ 0.9 }
\SetWatermarkText{DRAFT}
\SetWatermarkScale{2}
%Expected Value
\newcommand{\E}[1]{\mathbb{E}({#1})}
%LCM command
\DeclareMathOperator*{\lcm}{lcm}
%true and false
\DeclareMathOperator*{\true}{true}
\DeclareMathOperator*{\false}{false}
% Points commad
\newcommand{\points}[1]{[$#1 \star$]}
% Setup counters
\newcounter{hindex}\setcounter{hindex}{0}
\newcounter{hintcounter}\setcounter{hintcounter}{0}
% Define \addhint and \gethint
\newcommand\addhint[1]{%
	\stepcounter{hintcounter}%
	\ref{hint:\thehintcounter}%
	\expandafter\gdef\csname hintlist\thehintcounter\endcsname{#1}%
}
\newcommand\gethint[1]{%
	\item \csname hintlist#1\endcsname \label{hint:#1}
}
\newenvironment{hint}{\footnotesize \normalfont \textbf{Hints}:}{\hspace{-0.5ex}}
\pgfmathsetseed{65536} % or any other number: sets the random seed

\newtheorem{theorem}{Theorem}[chapter]
\newtheorem{lemma}[theorem]{Lemma}

\theoremstyle{definition}
\newtheorem{definition}[theorem]{Definition}
\newtheorem{example}[theorem]{Example}
\newtheorem{xca}[theorem]{Exercise}

\theoremstyle{remark}
\newtheorem{remark}[theorem]{Remark}

\numberwithin{section}{chapter}
\numberwithin{equation}{chapter}
\setcounter{chapter}{-3}

\setcounter{chapter}{0}
\title{Doorway to Math Olympiads}
\author{Arjun Agarwal}
\begin{document}
\frontmatter
\maketitle
\epigraph{The study of mathematics, like the Nile, begins in minuteness but ends in magnificence.}{\textit{Charles Caleb Colton}}
\epigraph{The essence of mathematics is not to make simple things complicated, but to make complicated things simple.}{S. Gudder}
\pagebreak

\begin{center}
\copyright\ 2023 Arjun Agarwal \\
Text licensed under CC-by-SA-4.0 \\
This is (still!) an incomplete draft. \\
Please send corrections, comments, pictures of kittens, etc. \\
to \href{mailto:anulick0@gmail.com}{anulick0@gmail.com} \\
Last updated \today.
\end{center}
\pagebreak

\thispagestyle{empty}
\vspace*{13.5pc}
\begin{center}
{\em To Ashu Agarwal Ma'am, Aparna Thakur Ma'am and Sylvia Marcarhanas Ma'am, who made me fall in love with mathematics.} 
\end{center}
\pagebreak


\tableofcontents



\mainmatter

\part{Combinatorics}
\input{1PnC/ConstCount}
\chapter{Some methods of counting}
The last chapter dealt with mostly constructive counting. 
There we directly using permutations and combinations try to count exactly 
what we are asked to count. Straightforward, right?.\par
However, life is not that simple. Sometimes it is easier to count what we are 
not asked to count and simply subtract(complimentary counting) or divide our problem
into smaller cases and add them(casework). Let me illustrate:
\section{Casework}
\begin{example}
    [Motivating Example]
    For how many pairs of consecutive integers lying in 
    {$1000, 1001, 1002, \dots 1998, 1999, 2000$}, does their addition happen without any carry over?
\end{example}
\begin{proof}
    [Solution]
We need to notice that carry over happens if the sum of two digits at a place are above 10. 
This means we can easily limit out the numbers which are like $1a85$ or $1a89$ as there successor 
has a $9$ right below the $8$.
    Keeping our observations in mind, we can see that: 
    $1999$ meets the criteria\par
    $1a99$ meets the criteria if and only if $a=0,1,2,3,4$\par
    $1ab9$ meets the criteria if and only if $a,b=0,1,2,3,4$\par
    $1abc$ meets the criteria if and only if $a,b,c=0,1,2,3,4$\par
    That will lead to $1+5+5^2+5^3=156$ such pairs existing.
\end{proof}
When we break a question down into multiple cases and then solve them and add it, it is called casework.\par
This technique is the messiest of the bunch and PnC as branch has developed to try to avoid it. Much of our 
further chapters will be about developing stronger techniques to avoid casework.
\section{Complementary Counting}
\begin{example}
    [Motivating Example] How many three digit numbers exist such that the number contains at least one 0 or 5?
\end{example}
\begin{proof}
    [Solution]
    Instead of looking for the digits which with the condition, it is easier in this case to look 
    for digits which don't. If a three digit number doesn't have 0 or 5, we have 8 choices for every 
    digit. Thus, we have $8^3=512$ such digits.\par
    Thus, as their are 900 total three digit numbers and 512 are not suitable so we have $900-512=388$
\end{proof}
This concept in known as complementary counting, counting the thing opposite to asked and then 
subtracting from the total. \par
It is the hardest to spot in the wild but also the most versatile. I normally encourage you 
to at least give it a thought if the phrase 'at least' makes an appearance.
\section{Principle of Inclusion and Exclusion}
\begin{example}
    [Motivating Example]
    (AIME) Many states use a sequence of three letters followed by a sequence of three 
    digits as their standard license-plate pattern. Given that each three-letter three-digit 
    arrangement is equally likely, the probability that such a license plate will contain at 
    least one palindrome (a three-letter arrangement or a three-digit arrangement that reads 
    the same left-to-right as it does right-to-left) is $m/n$, where m and n are relatively 
    prime positive integers. Find $m + n$.
\end{example}
\begin{proof}
    [Solution]
    A three character palindrome is of the form "$\spadesuit \heartsuit \spadesuit$". 
    Note that $\heartsuit = \spadesuit$ is also valid.\par
    This means that the number of letter palindromes is $26*26=26^2$ and number of number 
    palindromes is $10*10=10^2$. So total plates with at least one palindrome is the sum of 
    plates where number is the palindrome and the plates where letter is the palindrome. 
    However, we'll subtract number of plates where both are palindrome as they were counted 
    twice. So we can hence get the total plates with at least one palindrome as 
    $26^2*10^3+10^2*26^3-26^2*10^2$. As the question asks probability, lets 
    divide this by total number of plates which is $26^3*10^3$\par 
\begin{align*}
\frac{26^2*10^3+10^2*26^3-26^2*10^2}{26^3*10^3}\\
= \frac{26^2*10^2(10+26-1)}{26^2*10^2(26*10)}\\
=\frac{35}{260}=\frac{7}{52}
\end{align*}
Hence the answer is $7+52=59$
\end{proof}
This is called the principal of inclusion and exclusion or PIE for short. 
This is used in questions where two or more conditions are to be satisfied. 
Here, we would like introduce a general theorem for the same but will revise 
two definitions before doing the same:
\begin{definition}
$A \cup B$ refers to the set of all elements which occur in A and B, 
with duplicates(elements common to both sets) are only written once. 
This is known as union and read as A union B.
\end{definition}
\begin{definition}
    $A \cap B$ refers to the set of the common elements which occur in 
    both A and B. This is known as intersection and read as A intersection B.    
\end{definition}
Now you are ready for the PIE theorem:
\begin{theorem}
    $A \cup B = A + B - A \cap B$
\end{theorem}
Which is what we just intuitively used in the above example.\par
We also have PIE for three sets which is as follows:
\begin{theorem}
    \begin{multline*}
        A \cup B \cup C = A + B + C \\
        - A \cap B - B \cap C - C \cap A \\
        + A \cap B \cap C
    \end{multline*}    
\end{theorem}
This is easier to understand using a Venn diagram. 
If you draw it and label each circle as a set, you'll intuitively get why this is true.\par
This also extends to union of 4 or more sets. While we can't draw such 
Venn Diagrams easily, we can prove that the pattern holds true for larger number of sets using induction.
\begin{theorem}
    If $(A_i)_{1\leq i\leq n}$ are finite sets, then:
\begin{multline*}
    \left|\bigcup_{i=1}^n A_i\right|=\sum_{i=1}^n\left|A_i\right| \\
    -\sum_{i < j}\left|A_i\cap A_j\right|\\
    +\sum_{i<j<k}\left|A_i\cap A_j\cap A_k\right|\\
    -\cdots\ \\
    +(-1)^{n-1} \left|A_1\cap\cdots\cap A_n\right|{}
\end{multline*}
\end{theorem}
\begin{proof}
    (B) We can see this obviously works if we have a single set $A$.\par
    (S) Let's assume this is true for some $n$ number of sets. We will prove that then it will be true for $n+1$ number of sets as well.\par
    Let's consider an element $k \subset A_{n+1}$ which is also part of $A_1 \cup A_2 \cup \dots \cup A_n$.\par
    This means we need to prove that we still count it only once.\par
    Without loss of generality, assume that $k$ is a part of $A_1, A_2, A_3 \dots A_l$.\par
    This means the number of times we count $k$ is originally:\par
    $\binom{l}{1}-\binom{l}{2}+\binom{l}{3}-\dots=1$\par
    This will be $1$ as we have assumed the statement true for $n$.\par
    This number of times we will count $k$ when $A_{n+1}$ is present is:\par
    $\binom{l+1}{1}-\binom{l+1}{2}+\binom{l+1}{3}-\dots$\par
    Here we make the observation:\par
    \begin{align*}
        \binom{n}{k}+\binom{n}{k-1} & = \frac{n!}{k!(n-k)!}+ \frac{n!}{(k-1)!(n-k+1)!}\\
        & = \frac{n!}{(k-1)!(n-k)!}\left( \frac{1}{k}+\frac{1}{n-k+1} \right)\\
        & = \frac{n!}{(k-1)!(n-k)!}\left( \frac{n+1}{(k)(n-k+1)} \right)\\
        & = \frac{(n+1)!}{k!(n-k+1)!}\\
        & = \binom{n+1}{k}\\
    \end{align*}
    This is known as Pascal's identity. We will discuss it in detail later.\par
    But for this proof, we can use it to break the present sum.
\begin{align*}
\binom{l+1}{1}-\binom{l+1}{2}+\binom{l+1}{3}-\dots & \\
&= \left(\binom{l}{1}+\binom{l}{0}\right)-\left(\binom{l}{2}+\binom{l}{1}\right)\\
+\left(\binom{l}{3}+\binom{l}{2}\right)\dots\\
&= \binom{l}{0}+\left( \binom{l}{1}-\binom{l}{2}+\dots \right)\\
-\left( \binom{l}{1}-\binom{l}{2}+\dots \right)\\
&= 1+1-1=1
\end{align*}
Thus, by induction, the above is true.
\end{proof}
Please feel free to forget this proof as more than $3$ subsets is rare in math contests. 
Programming although does often walk into larger cases. But anyways, A rare specimen of more than $3$ 
subsets from ARML is included in the problems.\par
This brings us at the end of fundamental principles of counting. Here are the arithmetic rules we used:
\begin{theorem}
    \textbf{Multiplication rule}: If two choices are independent, we can multiply them.\par
    \textbf{Addition rule:} If we can break a problem into multiple sub problems after a certain choice, 
    we can add the results.\par
    \textbf{Subtraction Rule:} If we can count the complement of what we are asked, 
    then we can subtract it from the total.\par
    \textbf{Division Rule}: We can divide to account for multiple countings of repeated objects.
\end{theorem}
\section{Exercises}
Solve at least questions worth \points{50}. This exercise has a total of \points{71}.
\begin{xcb}{Exercises}
\begin{enumerate}
\item (AMC 12 2014) \points{2} A fancy bed and breakfast inn has $5$ rooms, each with a distinctive color-coded decor. One day $5$ friends arrive to spend the night. There are no other guests that night. The friends can room in any combination they wish, but with no more than $2$ friends per room. In how many ways can the innkeeper assign the guests to the rooms?
\begin{hint}
    \addhint{In how many ways can the friends be put into rooms? Like we can have one friend per room, $2$ in one room and rest in single room, $2$ sharing a room, another $2$ sharing a room and one person living in a single room}.
\end{hint}
\item (AMC 10 2021) \points{3} A farmer’s rectangular field is partitioned into $2$ by $2$ grid of $4$ rectangles. In each section the farmer will plant one crop: corn, wheat, soybeans, or potatoes. The farmer does not want to grow corn and wheat in any two sections that share a border, and the farmer does not want to grow soybeans and potatoes in any two sections that share a border. Given these restrictions, in how many ways can the farmer choose crops to plant in each of the four sections of the field?
\item (AMC 12 2021) \points{3} Azar and Carl play a game of tic-tac-toe. Suppose the players make their moves at random, rather than trying to follow a rational strategy, and that Carl wins the game when he places his third O. How many ways can the board look after the game is over?
\begin{hint}
    \addhint{Make the winning combinations and then try to arrange the 3 X's in any non winning way.}
\end{hint}
\item (AMC 10 2004) \points{2} Coin A is flipped three times and coin B is flipped four times. What is the probability that the number of heads obtained from flipping the two fair coins is the same?
\item (AMC 10 2014) \points{3} Three fair six-sided dice are rolled. What is the probability that the values shown on two of the dice sum to the value shown on the remaining die?
\item(AMC 10 2015) \points{2} How many rearrangements of $abcd$ are there in which no two adjacent letters are also adjacent letters in the alphabet? For example, no such rearrangements could include either $ab$ or $ba$.
\begin{hint}
    \addhint{With such a strict condition and only $4!$ maximum cases, writing by hand seems quite promising.}
\end{hint}
\item \points{3} How many three-digit numbers are composed of three distinct digits such the tens digit is the average of the other two?
\begin{hint}
    \addhint{Write by hand, the condition is quite stringent}
    \addhint{We can notice that if $abc$ is such a number then $cba$ is also such number}
\end{hint}
\item (AMC 10 2020) \points{9} There are 10 people standing equally spaced around a circle. Each person knows exactly 3 of the other 9 people: the 2 people standing next to her or him, as well as the person directly across the circle. How many ways are there for the 10 people to split up into 5 pairs so that the members of each pair know each other?
\begin{hint}
    \addhint{Experiment with smaller cases. Maybe you'll notice something}
    \addhint{Try casework about the number of diameters used}
\end{hint}
\item (AMC 10 2020) \points{9} Jason rolls three fair standard six-sided dice. Then he looks at the rolls and chooses a subset of the dice (possibly empty, possibly all three dice) to reroll. After rerolling, he wins if and only if the sum of the numbers face up on the three dice is exactly 7. Jason always plays to optimize his chances of winning. What is the probability that he chooses to reroll exactly two of the dice?
\begin{hint}
    \addhint{Think about the cases when he'll roll $0,1,2$ or $3$ dice}
\end{hint}
\item(AMC 12 2021) \points{9} Each of the 12 edges of a cube is labeled 0 or 1. Two labeling are considered different even if one can be obtained from the other by a sequence of one or more rotations and/or reflections. For how many such labeling is the sum of the labels on the edges of each of the 6 faces of the cube equal to 2?
\begin{hint}
    \addhint{Try using casework on the fact that the opposite edges have same number or not.}
\end{hint}
\item(AMC 10 2018) \points{2} How many subsets of ${2, 3, 4, 5, 6, 7, 8, 9}$ contain at least one prime number?
\item \points{3} Let $(a, b, c, d)$ be an ordered quadruple of not necessarily distinct integers, each one of them in the set $0, 1, 2, 3$ For how many such quadruples is it true that $a*d -b*c$ is odd?
\item (AMC 10) \points{3} In how many ways can the sequence $1, 2, 3, 4, 5$ be rearranged so that no three consecutive terms are increasing and no three consecutive terms are decreasing?
\item \points{9} Each unit square of a 3-by-3 unit-square grid is to be colored either blue or red. For each square, either color is equally likely to be used. The probability of obtaining a grid that does not have a 2-by-2 red square is $m/n$ , where m and n are relatively prime positive. Find $m+n$
\begin{hint}
    \addhint{Complimentary counting seems quite fine}
    \addhint{You'll also need to do some casework on the number of red squares}
\end{hint}
\item      (2015 ARML) \points{9} Six people of different heights are getting in line to buy donuts. Compute the number of ways they can arrange themselves in line such that no three consecutive people are in increasing order of height, from front to back.[The rare specimen I promised!]
\begin{hint}
\addhint{Use Complementary counting and the PIE}
\end{hint}
\end{enumerate}
\end{xcb}
\input{1PnC/guess}
\chapter{Stars, Bars and Hockey Sticks}
While most problems can be solved using the basic methods, sometimes its better to 
remember a certain fact. This chapter covers some common equivalence as well as some 
common combinatorial sums. We will start with Stars and Bars or Beggar's Theorem which 
is one of the most famous equivalences ever.
\section{Stars and Bars}
\begin{example}
    [Motivating Example] In how many ways can $10$ chocolates be divided 
    among $3$ children(The children are distinguishable, the chocolates aren't)?
\end{example}
\begin{proof}
    [Solution]
    This question is basically number of ordered tuples $(a,b,c)$ such 
    that $a+b+c=10$ given that $a,b,c \geq 0$.\par
This means we can re frame the question as ways to insert two 
identical bars among ten identical stars. Note that this is equivalent 
as the stars will be divided into three parts which will sum to $10$. As the 
problem is same, so is the solution, hence:\par
We are looking for the permutations of this configuration which will be: 
$\frac{(10+2)!}{2!*10!}$ Which is equal to $66$.
\end{proof}
Note that it is also possible to get the solution with casework in this 
particular question. However, doing so will become increasingly impractical 
as the number of children and chocolates will increase.\par 
We can generalize the above idea as:
\begin{theorem}
[Stars and Bars]
    we can say the number ways to put $n$ similar objects in $k$ distinguishable bins 
    is equivalent to permuting $n$ stars and $k-1$ bars which is equal 
    to $\frac{(n+k-1)!}{n!*(k-1)!}=\binom{n+k-1}{n}$
\end{theorem}
The Stars and Bars has various uses. The vanilla use of it comes up 
routinely in collage entrances. It can be made a little more spicy by doing a small change.
\begin{example}
    Alice has $24$ apples. In how many ways can she share them with Becky and 
    Chris so that each of the three people has at least two apples?
\end{example}
\begin{proof}
    [Solution]
    As everyone needs to have $2$ apples, let's start by giving everyone $2$ apples 
    to begin with. This means we have to divide $18$ apples with $3$ people. This is 
    the classical stars and bars. So the answer is $\binom{20}{2}$.
\end{proof}
This question can be boosted a bit more if every apple limit? For 
example let's say Alice must have at least $3$ apples, Becky at least $2$ and Chris 
at least $1$. The solving(and in this case the answer) remains the same.\par
We can get it a bit more spicy.
\begin{example}
    Find the number of positive integer quadruples $(a, b, c, d)$ that satisfy $a+b+c+d<24$.
\end{example}
\begin{proof}
    [Solution]
    Let $a+b+c+d=24-e \iff a+b+c+d+e=24$ where $e$ is a positive integer. 
    This means $a,b,c,d,e$ are all positive integers. Note that $0$ is neither 
    positive nor negative. So all of them are at least $1$.\par
    This means we are back to a Stars and Bars problem. Give all of them $1$ to start with 
    and then inserting in the theorem we get $\binom{23}{4}$.
\end{proof}
To lead us to the next part, I'll need you to notice something:\par
We could use casework on this question. So we want to solve $a+b+c+d=k$ for $k=3,\dots,23$. Using 
Stars and bars, we get the solution to each of the case by giving $a,b,c,d$ $1$ each to begin 
with. So using stars and bars, $\binom{k-4+3}{3}=\binom{k-1}{3}$ for some value of $k$. 
This means $\binom{3}{3}+\dots+\binom{22}{3}$ is the answer.\par
We already know the answer is $\binom{23}{4}$ and as they are equal, 
we get: $\binom{3}{3}+\dots+\binom{22}{3} = \binom{23}{4}$\par
\section{Counting in two ways}
Let's talk about a last chapters once. While playing the guessing game, 
you may have found yourself having two or more ways to solve the same question both, 
hopefully, leading to the same answer. But as one method is easier to compute than the 
other, we should use that. But that doesn't prevent us from thinking about it.\par
Given a configuration, we should get the same answer from any which methods. This 
means we can count in two ways.\par
This idea is what we use to create some combinatorial identities. We calculate the 
same thing generalized thing in two ways and equate them to get an identity, which 
we can use elsewhere as it is true in general.\par
This is called counting in two ways. Let's see it in action:
\begin{example}
    [Motivating Example]
    How many councils with at least $1$ member and at most $n$ members be made 
    from a pool of $n$ people?
\end{example}
\begin{proof}
    [Solution]
    We obviously know that the answer is $2^n - 1$ from the subset theorem. \par
    However we can also write this as $\binom{n}{1} + \binom{n}{2} \dots \binom{n}{n}$\par
    We can, hence say, $\binom{n}{1} + \binom{n}{2} \dots \binom{n}{n}= 2^n -1$ \par
    $\therefore \binom{n}{1} + \binom{n}{2} \dots \binom{n}{n} +1 = 2^n$ \par
    as we know $\binom{n}{0}=1$, we can say: $\binom{n}{0} + \binom{n}{1} + \binom{n}{2} 
    \dots \binom{n}{n} = 2^n$
\end{proof}
What we just derived is called the Binomial identity.
\begin{theorem}
    $\binom{n}{0} + \binom{n}{1} + \binom{n}{2} \dots \binom{n}{n} = 2^n$
\end{theorem}
Let's talk about the name of the identity for a while.\par
Binomial in math refers to polynomial with $2$ terms. For example $x+2$ or $3x+7$, in general $ax+b$.\par
If you have already studied some algebras, you may know that $(a+b)^2=a^2+2ab+b^2$ 
and $(a+b)^3=a^3+3a^2b+3ab^2+b^3$. These are normally derived by opening the brackets and multiplying(FOIL).\par
But how do we find $(a+b)^4$ or worse $(a+b)^{10}$. \par
We can notice that all the terms of the expansion of $(a+b)^k$ are $a^mb^n$ where $m+n=k$. 
I smell some combinatorics...\par
\begin{theorem}
    [Binomial Theorem]
    \[(a+b)^n = \binom{n}{0}a^{n}b^0+\binom{n}{1}a^{n-1}b^1+\dots+
    \binom{n}{n-1}a^{1}b^{n-1}+\binom{n}{n}a^{0}b^{n} = 
    \sum_{m=0}^{k}{\binom{k}{m}}\cdot a^m\cdot b^{k-m}\]
\end{theorem}
\begin{proof}
    As expected, The Binomial Theorem has a nice combinatorial proof:\par
We can write $(a+b)^k=\underbrace{ (a+b)\cdot(a+b)\cdot(a+b)\cdot\cdots\cdot(a+b) }_{k}$. 
Repeatedly using the distributive property, we see that for a term $a^m b^{k-m}$, we must 
choose $m$ of the $k$ terms to contribute an $a$ to the term, and then each of the other $k-m$ terms 
of the product must contribute a $b$. Thus, the coefficient of $a^m b^{k-m}$ is the number of ways to 
choose $m$ objects from objects $k$, or $\binom{k}{m}$. Extending this to all possible values of $m$ 
from $0$ to $k$, we see that $(a+b)^k = \sum_{m=0}^{k}{\binom{k}{m}}\cdot a^m\cdot b^{k-m}$, as claimed.
\end{proof}
This is also the reason $\binom{n}{k}$ is called a binomial coefficient.\par
Why is the binomial theorem useful here? We can use the binomial theorem to 
expand $(1+1)^n=\binom{n}{0}+\binom{n}{1}+\dots+\binom{n}{n-1}+\binom{n}{n}=2^n$\par
There a lot more uses of binomial theorem as well. Unfortunately, we shall not cover them in this book.\par
Let's look at some more identities.\par
\begin{example}
    [Motivating Example]
     Suppose a committee consists of $m$ men and $n$ women. In how many ways can a 
     subcommittee of $r$ members be formed?
\end{example}
\begin{proof}
    [Solution]
    We obviously know that the answer is $\binom{m+n}{r}$. \par
    However we can also do some case work. Let's say a committee has $0$ men 
    and $r$ women. Then $1$ man and $r-1$ women. And so on. We can write 
    this as $\binom{m}{0}*\binom{n}{r-0} + \binom{m}{1}*\binom{n}{r-1} 
    \dots +\binom{m}{r-1}*\binom{n}{1}+\binom{m}{r}*\binom{n}{0}$\par
    We can, hence say, 
    \[\binom{m}{0}*\binom{n}{r-0} + \binom{m}{1}*\binom{n}{r-1} 
    \dots +\binom{m}{r-1}*\binom{n}{1}+\binom{m}{r}*\binom{n}{0}=\binom{m+n}{r}\]
    \end{proof}
This is called Vandermonde’s Identity.
\begin{theorem}
    $\binom{m}{0}*\binom{n}{r-0} + \binom{m}{1}*\binom{n}{r-1} \dots +
    \binom{m}{r-1}*\binom{n}{1}+\binom{m}{r}*\binom{n}{0}=\binom{m+n}{r}$
\end{theorem}
\section{Pascal's Triangle}
\begin{figure}
    \centering
    \includegraphics[width=0.75\linewidth]{Photos/Pascal's Triangle.png}
    \caption{Pascal's Triangle: A wonder of the mathematical world}    
\end{figure}
The triangle somewhere on the top of the page is called Pascal's triangle. \par
It has a lot of fun and amazing properties(try to find them, you'll be surprised).\par
We are going to exploit two of them right now. First being, Every term in 
subsequent line is made by adding the two above it. As it turned out, every ancient 
civilization did reach this triangle by doing just that. The second one, the reason 
Blaise Pascal gets to have his name on it.
\begin{figure}
    \centering
    \includegraphics[width=0.75\linewidth]{Photos/Binomial Pascal}
    \caption{As it turns out, we can write everything as a binomial coefficient here.}
\end{figure}
This is exceptionally useful as this leads us to Pascal's Identity....
\begin{theorem}
    $\binom{n}{k} + \binom{n}{k+1}=\binom{n+1}{k+1}$
\end{theorem}
Remember, we saw this while proving principle of inclusion exclusion. 
There we proved it algebraically. We also 'proved' it in a sort of hand wavy manner 
above using pascal's triangle. Here is an another proof, combinatorial this time, for it.
\begin{proof}
    Let's consider choosing a team of $k$ lawyers from a pool of $n-1$ junior lawyers and $1$ Harvey Specter.\par
    The answer is $\binom{n}{k}$. Using casework, We can either have Harvey on the 
    team or not on it. If Harvey is on the team, we have $\binom{n-1}{k-1}$ ways to 
    choose rest of the lawyers. If we don't have Harvey on it, we have $\binom{n-1}{k}$ ways to 
    choose the team.\par
    This means $\binom{n-1}{k-1}+\binom{n-1}{k}=\binom{n}{k}$.
\end{proof}
Finally, here is the hockey stick identity(try looking at a diagonal 
in the triangle, that's where the name comes from). We started this 
section at this identity, and I'll let you prove it.
\begin{theorem}
    $\binom{k}{k}+\binom{k+1}{k}+ \dots +\binom{n}{k}=\binom{n+1}{k+1}$
\end{theorem}
Now here is the first thing, you should remember these identities. They can be memorized 
quite simply by writing them on a piece of paper and taping them to a wall next to where 
you sleep. Every morning look at it just after you wake up, and every night just before you 
sleep. You'll have them memorized in less then a week.\par
The another way to remember(the one which I remember), is by simply solving the questions and 
deriving every identity you forget, no turning the pages back. That will get it done in less 
than two hours flat.
\section{Exercises}
Solve at least questions worth \points{60}. This exercise has a total of \points{75}.
\begin{xcb}{Exercises}
\begin{enumerate}
\item \points{2} In how many ways can one get $10$ upon rolling $7$ dice?
\item \points{2} How many $4$ digit numbers have a sum of $9$?
\item \points{3} How many ordered pairs $(a,b,c,d)$ where $a \le b \le c \le d \le 5$ and $a,b,c,d \in \mathbb{N}$?
\begin{hint}
    \addhint{Let $a=1+x$, $b=a+y$ and so on. Does this look like a Stars and Bars now?}
\end{hint}
\item (AIME 2000) \points{9} Given that\par
\[\frac 1{2!17!}+\frac 1{3!16!}+\frac 1{4!15!}+\frac 1{5!14!}+\frac 1{6!13!}+\frac 1{7!12!}+\frac 1{8!11!}+\frac 1{9!10!}=\frac N{1!18!}\]
find the greatest integer that is less than $\frac N{100}$.
\begin{hint}
    \addhint{Multiply both sides by $19!$}\par
    \addhint{Multiply both sides by $2$ and use $\binom{n}{k}=\binom{n}{n-k}$}
    \addhint{Add terms to reach an identity which we know the summation to}
\end{hint}
\item(AIME 2001) \points{3} A fair die is rolled four times. The probability that each of the final three rolls is at least as large as the roll preceding it may be expressed in the form $m/n$ where m and n are relatively prime positive integers. Find $m + n$.
\item (AMC 8 2018) \points{3} From a regular octagon, a triangle is formed by connecting three randomly chosen vertices of the octagon. What is the probability that at least one of the sides of the triangle is also a side of the octagon?
\begin{hint}
    \addhint{Can we convert the choosing of $3$ points as partitioning of remaining $5$ points into $3$?}
\end{hint}
\item(AIME 2020) \points{2} A club consisting of $11$ men and $12$ women needs to choose a committee from among its members so that the number of women on the committee is one more than the number of men on the committee. The committee could have as few as $1$ member or as many as $23$ members. Let $N$ be the number of such committees that can be formed. If $N=\binom{a}{b}$, find $a + b$
\begin{hint}
    \addhint{We'll use Vandermonde identity, now try to solve it.}
\end{hint}
\item (AIME 2015) \points{3} Consider all 1000-element subsets of the set ${{1, 2, 3, \dots , 2015}}$. From each such subset choose the least element. The arithmetic mean of all of these least elements is $p/q$, where $p$ and $q$ are relatively prime positive integers. Find $p + q$.
\begin{hint}
    \addhint{$a_1+2a_2+3a_3+\dots na_n=(a_1+a_2+\dots+a_n)+(a_2+a_3+\dots)+\dots+(a_n)$}
\end{hint}
\item \points{2} For how many positive integers $x_1, x_2, \dots, x_{10}$ do we have $x_1 + x_2 + \dots + x_{10} = 50$?
\item (AIME 2011) \points{5} Ed has five identical green marbles, and a large supply of identical red marbles. He arranges the green marbles and some of the red ones in a row and finds that the number of marbles whose right hand neighbor is the same color as themselves is equal to the number of marbles whose right hand neighbor is the other color. An example of such an arrangement is $GGRRRGGRG$. Let $m$ be the maximum number of red marbles for which such an arrangement is possible, and let $N$ be the number of ways he can arrange the $m+5$ marbles to satisfy the requirement. Find the remainder when $N$ is divided by $1000$.
\begin{hint}
    \addhint{$m$ is limited by the fact that number of marbles whose right hand neighbor is the other color is maximum 10. How many red marbles will we need for this to lead to a valid arrangement?}
    \addhint{As we know $m$ now, can we simply use stars and bars with one red marble always between 2 green.}
\end{hint}
\item (AIME 2011) \points{9} Six men and some number of women stand in a line in random order. Let $p$ be the probability that a group of at least four men stand together in the line, given that every man stands next to at least one other man. Find the least number of women in the line such that $p$ does not exceed 1 percent.
\begin{hint}
    \addhint{In what ways can the men be arranged in? Use the groups of men as bars and the women as stars}
    \addhint{Use Pascal to simplify and then open the binomial}
\end{hint}
\item \points{2} Let $n$ be a positive integer. In how many ways can one write a sum of at least two positive integers that add up to $n$?
\item (AIME 2013) \points{5} Melinda has three empty boxes and $12$ textbooks, three of which are mathematics textbooks. One box will hold any three of her textbooks, one will hold any four of her textbooks, and one will hold any five of her textbooks. If Melinda packs her textbooks into these boxes in random order, the probability that all three mathematics textbooks end up in the same box can be written as $\frac{m}{n}$, where $m$ and $n$ are relatively prime positive integers. Find $m+n$.
\begin{hint}
    \addhint{Casework onto the box in which the three math textbooks are in}
    \addhint{Simplify before you compute! Take $9!$ common, multiply numerator and denominator by $3!4!5!$}
\end{hint}
\item (AMC 10 2016) \points{3} For some particular value of $N$, when $(a + b + c + d + 1)^N$ is expanded and like terms are combined, the resulting expression contains exactly $1001$ terms that include all four variables $a, b, c,$ and $d$, each to some positive power. What is $N$?
\item (AMC 12 2021) \points{9} A choir director must select a group of singers from among his $6$ tenors and $8$ basses. The only requirements are that the difference between the number of tenors and basses must be a multiple of $4$, and the group must have at least one singer. Let $N$ be the number of different groups that could be selected. What is the remainder when $N$ is divided by $100$?
\begin{hint}
    \addhint{Casework on the tenors $-$ basses combined with Vandermonde will work.}
\end{hint}
\item (IMO 1981/2) \points{9} Let $\displaystyle 1 \le r \le n$ and consider all subsets of $\displaystyle r$ elements of the set $\{ 1, 2, \ldots , n \}$. Each of these subsets has a smallest member. Let $\displaystyle F(n,r)$ denote the arithmetic mean of these smallest numbers; prove that\par
\[F(n,r) = \frac{n+1}{r+1}.\]
\begin{hint}
    \addhint{We have already solved a case of it as an AIME 2015 problem, won't the same technique work in general?}
\end{hint}
\item \points{5} Prove that \[\sum_{k=0}^n k{\binom{n}{k}}^2=n\binom{2n-1}{n-1}\]
\item Given a positive integer $n$, what is the largest $k$ such that the numbers $1, 2, \dots , n$ can be put
into $k$ boxes such that the sum of the numbers in each box is the same?
\end{enumerate}
\end{xcb}
\input{1PnC/geocount}

\part{Down the Rabbit Hole}
\input{2rabbithole/powerup}
\chapter{Graph Theory}
Now in normal language, Graph refers to a paper with a grid of squares. In more formal language we define it as follows:\\
A graph is a mathematical object we use to think about networks. It consists of a bunch of points, called vertices, and lines joining pairs of vertices, called edges.\\
While it is not 'explicitly' required in Olympiads, it makes easy work of a lot of problems which would otherwise be a lot harder. Also it's an active area of research so familiarity with it would not hurt.\\
\section{Definitions}
As I mentioned above, graphs are usually networks. We usually let $G$ denote the graph, $V$ the set of vertices and $E$ the set of edges, and write $G = (V, E)$.
\begin{figure}[h]
    \centering
    \includegraphics[width=0.5\linewidth]{Photos/Small graphs.png}
    \caption{Some graphs}
\end{figure}
We say two vertices are adjacent or are neighbors if they are joined by an edge, and nonadjacent otherwise. We say an edge and a vertex are incident if the vertex is one of the endpoints of the edge. The degree of a vertex $v$ is the number of edges incident with $v$.\\
\begin{example}
    What is least number of edges a graph on n vertices can have? What is the most number of edges a graph on n vertices can have?
\end{example}
The least number will be 0, as that will just be some points on a plane, with no edges joining them. That's called an independent set or empty graph.\\
The most will be $\binom{n}{2}$ where every node is attached to every other node. This is called a clique graph or a complete graph.\\
Believe it or not, with so little theory, we are now ready for our very first theorem. \\
\begin{theorem}
Let $G = (V, E)$ be a graph. Then:
\[ \sum _{v\in V}\deg v=2|E| \]
\end{theorem}
The proof is quite trivial. The LHS counts the degree of every vertices, which will be twice the number of edges as every vertices has 2 edges, which will both add it to the total. This is called the Handshake lemma as we can let the vertices be the number of people in a corporate boardroom and the edges be the handshake between them. If we ask every person how many hands he shook(the degree of his vertices) and add it up, it is quite elementary that we'll get twice the number of handshakes.\\
\begin{example}
     Is there a family of graphs such that every vertex has degree 0? 1? 2? 1 or 2? $\clubsuit$ Given integers $p$ and $q$, is there a graph such that every vertex either has degree $p$ or $q$
\end{example}
\begin{example}
     Let a vertex be even if it has even degree, and odd if it has odd degree. Which of the following doesn't exist: \\
     \begin{enumerate}
         \item Graph with only even vertices
         \item Graph with only odd vertices
         \item Graph with exactly one even vertex
         \item Graph with exactly one odd vertex
     \end{enumerate}
\end{example}
\section{Paths and Walks}
A walk in a graph $G$ is a sequence of vertices $v_0 - v_1 - v_2 - \dots - v_n$ such that each vertex $v_i$ is adjacent to the vertex $v_{i-1}$ before it and the vertex $v_{i+1}$ after it. We call $L$ the length of the walk.\\
A path in a graph $G$ is called a walk if all the vertices are different. Basically, every walk is a path, but every path is not a walk.\\
This leads us to ask: Given a walk between two vertices in a graph, how do we obtain a path between them? Is there always a walk between two vertices in a graph?\\
The answer is trivial, if a walk has cycles where we start a $v_n$ and after exploring some more vertices end at $v_n$, we can just remove them and have a path.\\
The answer to the second one is simple: no. For example in an independent set, we'll obviously not have a walk between any two points. \\
This leads us to a few more definitions:\\
\begin{definition}
    A cycle in a graph G is a sequence of vertices $v_0 - v_1 - v_2 - \dots  - v_n = v_0$ such that each vertex $v_i$ is adjacent to the vertex $v_{i-1}$ before it and the vertex $v_{i+1}$ after it. We call $L$ the length of the cycle.
\end{definition}
\begin{definition}
    A graph is disconnected if it can be divided into two parts with no edges between the parts, otherwise it is connected.
\end{definition}
\begin{definition}
    A connected component of a graph is a connected subgraph which is as large as possible.
\end{definition}
\begin{definition}
    Let $G$ be a connected graph. A cut vertex in $G$ is a vertex whose removal disconnects the graph. A cut edge in $G$ is an edge whose remove disconnects the graph
    \begin{figure}[h]
        \centering
        \includegraphics[width=0.5\linewidth]{Photos/Cut vertex.png}
        \caption{$e$ is the cut edge and $u,v$ the cut vertices}
        
    \end{figure}
\end{definition}
And now a theorem
\begin{theorem}
    If a graph has a cut edge, it has a cut vertex.\\
    However the converse of this is not true. If it has a cut vertex, it may or may not have a cut edge.
\end{theorem}
\begin{proof}
    The theorem is trivial for a $graph$ with less than $3$ vertices.\\
    Let $G$ have more than 3 vertices, and the cut edge $e=uv$. The other vertices will have a path to either $u$ or $v$ without using $e$ but not both as that would violate the assumption that $e$ is the cut edge.\\
    Here if one of $u$ or $v$ is removed, the graph would get disconcerted. If and only if $\deg u$ or $\deg v$ is zero, it will not be a cut vertex(which can happen for only one of them).\\
    The converse of this theorem is however not true. A counterexample will suffice:
    \begin{figure}[h]
        \centering
        \includegraphics[width=0.5\linewidth]{Photos/counterexample.png}
        \caption{The Counter-example}
    \end{figure}
\end{proof}
\begin{example}
    Let G be a graph with $n \geq 2$ vertices such that every vertex has degree at least $\frac{n-1}{2}$. Show that G is connected.
\end{example}
\section{Trees}
\begin{example}
    What is the smallest number of edges a connected graph of $n$ vertices can have?
\end{example}
If you try to draw connected graphs with the fewest number of edges possible, you’ll start to notice that they have a particular structure. They should, if you look closely, look like trees. \\
\begin{definition}
    A graph is acyclic if it contains no cycles. We call an acyclic graph a forest, and a connected acyclic graph a tree. A leaf is a vertex in a tree with degree one.
\end{definition}
All trees are forests, but all forests are not trees.\\
\begin{figure}[h]
    \centering
    \includegraphics[width=0.5\linewidth]{Photos/trees.png}
    \caption{Some trees}    
\end{figure}
You would have noticed that the number of edges in a tree of $n$ vertices is $n-1$.  You should also notice that every edge is a cut edge. Another thing that is elementry to note is that adding any new edge will create a cyclic path. A less obvious fact, however, is:\\
\begin{theorem}
    A forest with $n$ vertices and $k$ components contains $n - k$ edges
\end{theorem}
\begin{proof}
    Forest is called forest as every component will be a tree(as in real life). So we have $k$ trees which in total have $n$ vertices. Let the individual trees have $n_1, n_2, n_3 \dots n_k$ vertices and $n_1+n_2+n_3\dots +n_k=n$, so the number of edges will be $n_1 -1 +n_2 -1 + n_3 -1 \dots n_k -1= n-k$\\
\end{proof}
Let's now talk about one very last topic, and the very reason why we studied trees.
\begin{definition}
    A spanning tree of a graph $G$ is a subgraph of $G$ that is a tree containing all the vertices of $G$.
\end{definition}
The fun fact is:\\
\begin{theorem}
    A graph is connected if and only if it has a spanning tree.
\end{theorem}
\begin{proof}
As the statement has if and only if, we'll need to prove the theorem as well as its converse. \\
Let's first prove that if $G$ has a spanning tree it is connected. Suppose $G$ has a spanning tree. Then $G$ is connected, since for every pair of vertices $u, v$ there is a path from $u$ to $v$, namely the path in the spanning tree.\\
Now we'll prove that every connected graph has a spanning tree.\\ Suppose $G$ is connected. Let $e1, e2, \dots, e_m$ be the edges of $G$ in some order. If we go through the edges of $G$ in order and remove edge $e_i$ which is in a cycle then we end up with a graph $G'$
It it trivial that $G'$ is connected (if we remove $e_i$ then since $e_i$ is in a cycle the remaining graph is still connected). If we break all such cycles, we'll be left with $T$ which will be the spanned tree of graph $G$. 
\end{proof}
\begin{example}
    How many spanning trees does a tree have? How many spanning trees does a cycle have? How many spanning trees does a complete graph have?
\end{example}
\section{Real life applications}
I had promised in the start that graph theory has a varity of real lie applications. We'll go through a few here.\\
One application of paths in graphs is finding driving directions. Let's say I want to go from Cambridge University to Mathematics Bridge.  What is the fastest way to get there?
\begin{figure}[h]
    \centering
    \includegraphics[width=0.5\linewidth]{Photos/map.png}
    \caption{The google map}
\end{figure}
We can formulate this problem as a graph theory problem, where $G = (V, E)$ is a graph whose vertices $V$ are intersection points between roads in Cambridge, and edges are road segments. Then we can think of the university and Math Bridge as two vertices $u$ and $v$ in my graph $G$, and my questions become the following: How can we find whether there is a path between $u$ and $v$? How can we find a shortest path from $u$ to $v$?\\
We can do this in many ways, the most basic being the depth-first search and the breadth-first search algorithm.\\
\subsection{Depth-first search}
The idea of depth first search is: : Start at $u$ and ‘keep walking’, i.e. walk as far as possible searching for $v$, and if you hit a dead end backtrack.\\
The algorithm is as follows: Start at$u$ and move to any of it's neighbours. Let this be $v'$. Now from $v'$ move to another neighbour other than $u$. The algorithm stops on three conditions:\\
\begin{enumerate}
    \item If it reaches $v$, it then returns the path as the output.
    \item If it reaches a $v'$ with no unvisited neighbours. It will retract its path and move till the next fork where it will make a new choice and try again.
    \item If it reaches $v'=u$ and has no unvisted neighbours, it will declare that we have no path from $u$ to $v$
\end{enumerate}
Depth-first search allows us to find a path from u to v, or verify that no such path exists. But the path it finds has no guarantee of being the shortest path. In sad cases, it may generate a path from Cambridge to Kings Collage and then to the bridge.\\
\subsection{Breadth-first search}
The idea od Breadth first search is: Start at $u$, keep track of the vertices ‘closest’ to $u$ ($u$’s neighbors) and see if they’re $v$. If not, see if any of vertices ‘second-closest’ to $u$ (neighbors of $u$’s neighbors) are $v$. Eventually we'll find $v$.\\
This method is superior to depth search in the aspect that it will find the path with least number of edges. However, it also assumes that there is a path from $u$ to $v$, which if untrue, it will not detect and go till the system crashes(or it reaches a point where all vertices are neighbour less and terminates and declares no path).\\
However, they both don't give us the fastest path as we are taking all edges as equal. But roads are of different lengths and we know that longer roads take more time to travel. According to both of or methods, we'll end up considering Saudi's Highway 10(256 km) as just as short as Scotland's Ebenezer Place(2.46 meters).\\
To capture this, I will need the concept of weighted edges. A weighted graph is a graph where each edge is assigned a number, $w$ called its weight. For an edge $e$ we let $\omega(e)$ denote its weight.
\subsection{Dijkstra’s Algorithm}
The idea of Dijkstra is Start at $u$, for each vertex $v'$ keep track of an estimated (weighted) distance from $u$ to $v$. Visit the vertices in order of how ‘close’ they are to$u$ and see if they’re $v$. The more nitty gritty of this is left to the diligent reader.\\
However, despite being quite efficient, google maps doesn't work on Dijkstra. It instead uses $A^{*}$ algorithm. It basically makes the estimation step more accurate. Google also uses bi-directional search to make it quicker(looking for path from $u$ to $v$ as well as $v$ to $u$ simultaneously hoping that they meet somewhere in the middle) along with a lot of trade secret pruning, shortcuts and caching. But that is more computer science and less math.\\

\begin{xcb}{Exercises}
\begin{enumerate}
        \item Prove that for any graph with $n$ vertices and $m$ edges, has the lowest $\deg v \leq \frac{2m}{n}$ and the highest $\deg v \geq \frac{2m}{n}$.
        \item . Is it possible to build a house with exactly eight rooms, each with three doors, and such that exactly three of the house’s doors lead outside?
        \item The complement of a graph $G$ is the graph $G$ obtained by including an edge if and only if it was not present in the original graph. Prove that between $G$ and $G'$, one and only one is connected.
        \item Show that at any party, there are always at least two people with exactly the same number of friends at the party.
        \item(Martin Gardner) My wife and I recently attended a party at which there were four other married couples. Various handshakes took place. No one shook hands with oneself, nor with one's spouse, and no one shook hands with the same person more than once. After all the handshakes were over, I asked each person, including my wife, how many hands he (or she) had shaken. To my surprise each gave a different answer. How many hands did my wife shake?
        \item(IMOSL 2001) Define a $k$-clique to be a set of $k$ people such that every pair of them are acquainted with each other. At a certain party, every pair of 3-cliques has at least one person in common, and there are no 5-cliques. Prove that there are two or fewer people at the party whose departure leaves no 3-clique remaining
        \item (Italy 2007) Let $n$ be a positive odd integer. There are $n$ computers and exactly one cable joining each pair of computers. You are to color the computers and cables such that no two computers have the same color, no two cables joined to a common computer have the same color, and no computer is assigned the same color as any cable joined to it. Prove that this can be done using $n$ colors.
        \item {USAMO 2007, edited} An animal with n cells is a connected figure consisting of $n$ equal-sized cells which are square(a $n-$tile polymino). A dinosaur is an animal with at least 2023 cells. It is said to be primitive if its cells cannot be partitioned into two or more dinosaurs. Find the maximum number of cells in a primitive dinosaur
        \item Show that the Petersen graph has 2000 spanning trees 
\begin{figure}[h]
        \centering
        \includegraphics[width=0.5\linewidth]{Photos/Petersen Graph.png}
        \caption{Petersen Graph}
    \end{figure}
    \item (Cayley's Formula) A labelled tree of $n$ vertices is a tree of $n$ vertices where all vertices are given distinct labels from $1$ to $n$. Prove that there are $n^{n-2}$ labelled trees of $n$ vertices.
    \item(Veblen's Theorem) Prove that the edges of a graph can be partitioned into cycles if and only if each vertex has even degree.
    \item(Euler's Formula) Prove that given graph $G$ be a connected planar graph with $V$ vertices, $E$ edges and $F$ faces(number of parts it divdes the plane into, note: the outside of graph is also a face). Then $F + V - E = 2$
    \item (IrMO 1989) . Each of the $n$ members of a club is given a different item of information. They are allowed to share the information, but, for security reasons, only in the following way: A pair may communicate by telephone. During a telephone call only one member may speak. The member who speaks may tell the other member all the information he or she knows. Determine the minimal number of phone calls that are required to convey all the information to each other.
    \item (IrMO 1994) If a square is partitioned into n convex polygons, determine the maximum number of edges present in the resulting figure.
    \item (IMO 2019, edited) A social network has 2023 users, some pairs of which are friends (friendship is symmetric). If $A, B, C$ are three users such that $AB$ are friends and $AC$ are friends but $BC$ is not, then the administrator may perform the following operation: change the friendships such that $BC$ are friends, but $AB$ and $AC$ are no longer friends. Initially, $1011$ users have $1012$ friends and $1012$ users have $1011$ friends. Prove that the administrator can make a sequence of operations such that all users have at most 1 friend.
    \item (Moser's Circle Problem) Given $n$ points on the circumference of pizza, what is the maximum number of parts the circle is separated  by the chords connecting all the $n$ points to each other? (Note: \cancel{Don't} use Engineer's induction)
    \end{enumerate}
\end{xcb}
\chapter{A New Breed of Counting}
We are going to encounter a new breed of Combinitorics now onward where we are not so 
interested in counting in the ways to do something but in proving that there exists 
some way of doing something or more often, there exists no way of doing something. 
We'll now look at tiling's and coloring and expected values and betting and games and a lot, 
lot more. While this section begins our foray into the more robust real life uses of Combi, 
it is still relevant to Olympiads. A beginning example could be:\\
\begin{example}
    (RMO 2023) Consider a set of $16$ points arranged in $4 \times 4$ square grid formation. Prove that if any $7$ of these points are coloured blue, then there exists an isosceles right-angled triangle whose vertices are all blue.
\end{example}
\begin{proof}
    Probably the hardest problem on the test.\\
    Fun story, I actually gave the test when this question came. Overconfident in my combi, I messed it up pretty bad as I created a non-existent counter example. I'll leave the details to your imagination.\\
    The solution proceeds by observing the possible isosceles triangles.\\
    \begin{figure}{H}
        \centering
        \includegraphics[width=0.5\linewidth]{Photos/RMO6 2023.png}
        
    \end{figure}
    We need to see that either three points lie on the vertices of a square or $2$ on the vertices of the surrounding three squares and $1$ on the center.\\
    We therefore have at most $2$ points on the outside squares, which is $2*3=6$. But as $7$ points are colored, either a square has more than $2$ vertices colored or the center square has a colored vertex.\\
    Thus, the stated claim is true.\\
\end{proof}
Let's get started with the most complex simple thing!
\section{Pigeon Hole Principle}
This PHP while very powerful, seems stupid when heard/read for the first time:
\begin{theorem}
    [Pigeon hole principle]
    Consider a flock of pigeons nestled in a set of $n$ pigeonholes. If there are $n$ pigeons, then it is possible for all of the pigeons to rest happily in separate pigeonholes. However, if at least one more pigeon arrives, making a total of more than $n$ pigeons, then at least one of the pigeonholes, inevitably, will end up with more than one pigeon.\\
    In particular, if $k$ pigeons are put into $n$ holes, there are at least $\lceil \frac{k}{n} \rceil$ in one of the holes.
\end{theorem}
While PHP may seem pointless, even offensively funny to some, a correct choice of pigeons and holes can simplify quite a lot of problems. It was first used by Dirichlet and is therefore also called the Dirichlet principle. Here is an IMO problem to show PHP's power:\\
\begin{example}
    (IMO/1 1972) Prove that from a set of ten distinct two-digit numbers (in the decimal system), it is possible to select two disjoint subsets whose members have the same sum?
\end{example}
\begin{proof}
    Let the number of subsets be the pigeons and the possible sums be the holes. \\
    We have $2^{10} -2 =1022$ subsets and $90+91+92 \dots +99 - 10 + 1$ possible sums. As $100*10=1000 \leq 1022$, the pigeons are far less than the holes. Hence, two are sure to share the hole and hence, we will always have two disjoint subsets whose members have the same sum.
\end{proof}
You have already used PHP before in a much simpler form to answer riddles like: "You have $14$ brown socks, $14$ blue socks and $14$ black socks in your sock drawer. How many socks must you remove (without looking to be sure) to have a matched pair?"\\
We didn't give it a name then as the uses were obvious. As they become more complex, a name was given to the idea.\\
PHP, while simple,  can be extended to real analysis and other branches of mathematics in the following form:\\
\begin{theorem}
[Infinite PHP]
Given an infinite pigeons, if they are put into finite pigeon holes, there is at least one pigeon hole with infinite pigeons.
\end{theorem}
We can use this to prove something of the form:\\
\begin{example}
    A $100\times100$ board is divided into unit squares. In every square there is an arrow that points up, down, left or right. The board square is surrounded by a wall, except for the right side of the top right corner square. An insect is placed in one of the squares. Each second, the insect moves one unit in the direction of the arrow in its square. When the insect moves, the arrow of the square it was in moves $90$ degrees clockwise. If the indicated movement cannot be done, the insect does not move that second, but the arrow in its squares does move. Is it possible that the insect never leaves the board?
\end{example}
\begin{proof}
    Let's assume to the contrary that there is some arrangement such that the insect is trapped.\\
    In this case, it will make infinite moves. As there are only $100^2$ pigeon holes, the insect visits some square infinite time.\\
    As the arrows keep changing, it also visits the neighbouring squares infinite times.\\
    By the same logic, it visits all the squares infinite times. Hence, it visits the top right square infinite times, which would mean it is not trapped.\\
    This is a contradiction, hence our initial assumption is false. There exists no such arrangement such that the insect is trapped.\\
    Hence, proved.
\end{proof}
Before we move ahead, here is a classic PHP question for you to try which we can also do using Extremal principle(later in this chapter).\\
\begin{example}
    There are $17$ points inside an equilateral triangle with side lengths $1$. Prove there are at least
two points within distance $\frac{1}{4}$ of each other.\\
Hint: Try dividing the triangle in equal parts.
\end{example}
\section{Expected Value and Probabilistic Method}
The probabilistic method refers to proving the existence of some element in a set by showing the probability of its existence is non-zero.\\
Cutting out the jagron, let's say I have bag of candy. How do you prove that I have a Kit-Kat in the bag? You either keep on removing chocolates till you get a Kit-Kat, or through some tricks you prove that the probability of drawing a Kit-Kat is not zero.\\
Another allegory would be thinking of it like rolling a dice. If you roll a regular six-sided dice, there's a chance of getting any number from 1 to 6. The probabilistic method is a bit like saying, "Hey, there's a chance, maybe a small one, but it's not zero, that I'll get a 6." So, you can be pretty confident that a 6 exists on that dice, even if you haven't seen it yet.\\
This was explained thrice to provide the motivation to define the following.
\begin{definition}
[Expected Value]
For a random variable $X$,  $\E{x}=\sum {x_i \cdot P(x_i)}$ where $x_i$ are the possible values of X and $P(x_i)$ is the probability of the value being $x_i$
\end{definition}
For example for a dice, let $X$ be the number on the die. Then $\E{x}=\frac{1}{6}1+\frac{1}{6}2+\dots+\frac{1}{6}6 =3.5$.\\
What this tells us is that if we roll a dice, we expect the value to be $3.5$. While $3.5$ is not a number on the dice, over a lot of throws this is the average of roll of a dice.\\
We now define the most important theorem of the probabilistic method
\begin{theorem}
[Linearity of Expectations] $\E{X_1+X_2+X_3+\dots +X_n}=\E{x_1}+\E{x_2}+\dots +\E{x_n}$\\
\end{theorem}
This is obvious when the variables are independent. However, the beauty of this theorem is based on the fact that that even if they are not independent, this still stands. The proof has been omitted as it can be made using a incidence matrices and then summing up. You will end up with a ugly summation which will mean the same as above. Google it if you are curious!\\
Let's see a classic example\\
\begin{example}
[HMMT 2006] At a nursery, $2006$ babies sit in a circle. Suddenly, each baby randomly pokes either the baby to its left or to its right. What is the expected value of the number of unpoked babies?
\end{example}
\begin{proof}
    [Solution]
Let me introduce you to the most classy way of using the probabilistic method. We define a variable $P_n$ for the $n^{th}$ baby in the circle where $P=0$ if the baby is poked and $P=1$ if it is unpoked.\\
The probability of a baby being unpoked is $\frac{1}{4}$. Which means expected value of $P_n$ is $\frac{1}{4}$ which means using Linearity of Expectations(LOE) we can say that $\E{\text{Number of unpoked babies}}=\E{P_1+P_2+\dots+P_{2006}}=\E{P_1}+\dots+\E{P_{2006}}=\frac{2006}{4}$\\
\end{proof}
We can also have questions which while can be done using the probabilistic method, also have simpler solutions.
\begin{example}
(AMC 10 2021) Five balls are arranged around a circle. Chris chooses two adjacent balls at random and interchanges them. Then Silva does the same, with her choice of adjacent balls to interchange being independent of Chris’s. What is the expected number of balls that occupy their original positions after these two successive transpositions?
\end{example}
\begin{proof}
    [Solution]
We can also do this by the definition of expected value. Without loss of generality, let's say Chris interchanged the first ball with the second ball.\\
Silva can now interchange either the same two($\frac{1}{5}$ probability) which leaves us with $5$ balls in the original position.\\
Silva can interchange one of the balls Chris switched with another neighbour( $\frac{2}{5}$ probability) which leaves $2$ balls in the original position.\\
Silva can interchange two entirely new balls($\frac{2}{5}$ probability) which leaves $1$ ball in the original position.\\
This means the expected value is $\frac{1}{5}*5+\frac{2}{5}*2+\frac{2}{5}*1=1+0.8+0.4=2.2$\\
If we want to use probabilistic method, we can define a variable $P$ for every ball which is $0$ if the ball is not on its original position and $1$ if it is.\\
So the Expected value of $P$ is the sum of the probability of the ball never being switched and it being switched twice.\\
$\therefore \E{P}=(\frac{3}{5})^2+\frac{2}{5}\cdot\frac{1}{5}=\frac{11}{25}$\\
As we have $5$ balls, Using LOE, the expected number of balls in the original position are $\frac{11}{25}*5=\frac{11}{5}=2.2$
\end{proof}
To the contrary of the last example, here is one which cannot be solved neatly without LOE.\\
\begin{example}
    (Putnam 2014, A4)Suppose $X$ is a random variable that takes on only non-negative integer values, with $\E{X}=1$, $\E{X^2} = 2$, and $\E{X^3} = 5$. Determine the smallest possible value of the probability of the event $X=0$.
\end{example}
\begin{proof}
    We need to realize that given expected value of $x, x^2$ and $x^3$, we can find the expected value of any polynomial with degree $3$ or less courtesy linearity of expectations.\\
    This means we know the expected value of $f(x)=(x-1)(x-2)(x-3)=x^3-6x^2+11x-6$ which is $0$ for $1,2,3$.\\
    $\E{f(x)}=5-6*2+11*1-6=5-12+11-6=-2$\\
    We need to now notice that $f(0)=-6$ and $f(x)>0$ for all $x>3$. So as the probability of $x>3$ increases, probability of $x=0$ also increases.\\
    For the minimum case, let probability of $x=0$ be $p$ and probability of $x=1,2,3$ be $1-p$.\\
    This means $-6p=-2 \iff p=\frac{1}{3}$.\\
    We can construct probability for $1,2,3$ to satisfy the equations. That is left for you to try yourself.
\end{proof}
Before moving onto some other applications of this method, I'll give a brief view of how it ends up used in active research. Also I'll introduce a powerful graph theory theorem at the same time, and show off another way of setting the variable. One stone, three pigeons\\
\begin{theorem}
[Turan's Theorem] For $G = (V, E)$. Let an independent set be a set of vertices such that no two are adjacent. $G$ contains an independent set of size at least $\sum_{v \in V}\frac{1}{d(v)+1}$
\end{theorem}
\begin{proof}
We'll permute all the vertices in a line. Let's define a variable $S$ as the number of vertices such that a vertex in the permutation lies before all vertices adjacent to it. The expected value of this is obviously the expected size of an independent set.\\
Let another variable $P$ be $1$ if all neighbours of a vertex are after it in the permutation, and $0$ otherwise. Using LOE, we can say that $\E{S}=\sum_{v \in V} \E{P_v}$.\\
We need to note that for any vertex $v$, it had $d(v)$ adjacent vertices. Only one of them, among the vertex and its neighbours, has all its neighbours after it. This means that the probability of $P_v$ being $1$ is $\frac{1}{d(v)+1}$ which means $\E{P_v}=\frac{1}{d(v)+1}$\\
Thus, expected size of independent set $\E{S}=\sum_{v \in V}\frac{1}{d(v)+1}$
\end{proof}
We will also prove the same using Extremal principal later.\\

\begin{xcb}{Exercises}
    \begin{enumerate}
\item (AMC 10 2006) A player pays $5$ dollars to play a game. A die is rolled. If the number on the die is odd, the game is lost. If the number on the die is even, the die is rolled again. In this case the player wins if the second number matches the first and loses otherwise. How much should the player win if the game is fair? (In a fair game the probability of winning times the amount won is what the player should pay.)
\item(AMC 12) A school has 100 students and 5 teachers. In the first period, each student is taking one class, and each teacher is teaching one class. The enrollments in the classes are 50, 20, 20, 5, and 5. Let t be the average value obtained if a teacher is picked at random and the number of students in their class is noted. Let s be the average value obtained if a student was picked at random and the number of students in their class, including the student, is noted. What is $t-s$?
\item (OMM 1998)The sides and diagonals of a regular octagon are colored black or red. Show that there are at least $7$ monochromatic triangles with vertices in the vertices of the octagon.
\item  Let $F_n$ be the $n^{th}$ Fibonacci numbers. Show that for some $n \geq 1$, $F_n$ ends with 2023 zeros.
\item Let $S$ be a subset of ${1, 2, 3, . . . , 2n}$ with $n + 1$ elements. Show that there are two elements in $S$ which are relatively prime.
\item (continuing of the above question) Show that there are two elements in $S$, one divisible by the other
\item At a party, certain pairs of individuals have shaken hands. Prove that there exist two persons who have shaken the same number of hands
\item Given $7$ lines on the plane, prove that two of them form an angle less than $26^{\circ}$
\item A chess grandmaster has $77$ days to prepare for a tournament. He wants to play at least one game per day, but not more than 132 games in total. Prove that there is a sequence of successive days on which he plays exactly 21 games in total.
\item (Canada 2004, edited) Let $T$ be the set of all positive integer divisors of $2023^{100}$. What is the largest possible number of elements that a subset S of T can have if no element of S is an integer multiple of any other element of S?
\item Show that there is some $n$ for which $111 \dots 111$ (with $n$ ones) is divisible by $2023$
\item (IMO/4 1964)Seventeen people correspond by mail with one another - each one with all the rest. In their letters only three different topics are discussed. Each pair of correspondents deals with only one of these topics. Prove that there are at least three people who write to each other about the same topic.
\item Prove that the decimal representation of any irrational number has at least two digits appearing
infinitely often
\item (IMO/4 1985) Given a set $M$ of $1985$ distinct positive integers, none of which has a prime divisor greater than $23$, prove that $M$ contains a subset of $4$ elements whose product is the $4$th power of an integer.
\item  Prove that among any $2m + 1$ distinct integers of absolute value less than or equal to $2m - 1$, there are three whose sum is zero.(Although can be solved using PHP, there is a much simpler way out)
\item (Putnam/A2 2002)Given any five points on a sphere, show that some four of them must lie on a closed hemisphere.
\item  Given nine points inside the unit square, prove that some three of them form a triangle whose area does not exceed $\frac{1}{8}$.
    \end{enumerate}
\end{xcb}

\part{Algebra}
\input{3alg/algman.tex}
\chapter{Inequalities}
Till now we were dealing with the cases where $f(x)=g(y)$. The thing to note is that they were mostly equal. This is normally not true in real life. Say you have a budget of $10,000$ dollars for your wedding. Our expenditure needs to be less than or equal to $10,000$ dollars. I don't thing people will hold it against you for saving money.\\
This chapter doesn't use $=$ much. Instead we'll have two sides which will not be equal.\\
\section{Another note on notation}
We will make use of two new notations, $\sum_{\text{cyc}}$ and $\sum_{\text{sym}}$ which mean to cycle through all the $n$ values and go through all the $n!$ values respectively. For example:\\
$\sum_{\text{cyc}}a^2=a^2+b^2+c^2+d^2\\
\sum_{\text{cyc}}ab=ab+bc+ca+ad\\
\sum_{\text{sym}}a^2=6(a^2+b^2+c^2+d^2)\\
\sum_{\text{sym}}ab=ab+ac+ad+bc+bd+ad\\$
\section{AM-GM and Muirhead}
\begin{theorem}
[AM-GM]
    \[\frac{x_1 + x_2 + \ldots + x_n}{n} \geq \sqrt[n]{x_1 \cdot x_2 \cdot \ldots \cdot x_n}\]
    The equality holds, if and only if, $x_1=x_2=x_3\dots=x_n$
\end{theorem}
\begin{proof}
    AM-GM can be proven in various ways. However, the simplest and most logical one is presented here.\\
    We'll refer to the inequality as $I_n$ with $n$ being the number of terms. \\
    Let's first prove it for $n=2$.\\
    Let $x_1+k=x_2$. Proof by induction:\\
    (B) for $k=1$, $(\frac{x_1+x_1+1}{2})^2\\
    =(x_1+\frac{1}{2})^2\\
    =x_1^2+x_1+\frac{1}{4}
    > x_1(x_1+1)=x_1^2+x_1$\\
    (S) Let's assume this is true for some $k$ leading to $x_1^2+kx_1+\frac{k^2}{4}>x_1(x_1+k)$, then:\\
    $x_1(x_1+k+1)\\
= x_1^2+kx_1+x_1\\
= x_1(x_1+k)+x_1\\
< x_1^2+kx_1+\frac{k}{4}+x_1\\
< x_1^2+(k+1)x_1+\frac{(k+1)^2}{4}$\\
Hence, proved.\\
Now with this information,  let's do something unique. Assuming, $I_n$ holds, let's prove that $I_{2n}$ also holds.\\
Let $x_1, x_2, \ldots, x_{2n}$ be any list of nonnegative reals. Then, because the two lists $x_1, x_2 \ldots, x_n$ and $x_{n+1}, x_{n+2}, \ldots, x_{2n}$, each have $n$ variables,
\[\frac{x_1 + x_2 + \cdots + x_n}{n} \geq \sqrt[n]{x_1 x_2 \cdots x_n}\] and\\
\[\frac{x_{n+1} + x_{n+2} + \cdots + x_{2n}}{n} \geq \sqrt[n]{x_{n+1} x_{n+2} \cdots x_{2n}}.\]
Adding these two inequalities together and dividing by $2$ yields\[\frac{x_1 + x_2 + \cdots + x_{2n}}{2n} \geq \frac{\sqrt[n]{x_1 x_2 \cdots x_n} + \sqrt[n]{x_{n+1} x_{n+2} \cdots x_{2n}}}{2}.\]
Using $I_2$, on $\sqrt[n]{x_1 x_2 \cdots x_n}$ and $\sqrt[n]{x_{n+1} x_{n+2} \cdots x_{2n}}$ we get
\[\frac{\sqrt[n]{x_1 x_2 \cdots x_n} + \sqrt[n]{x_{n+1} x_{n+2} \cdots x_{2n}}}{2} \geq \sqrt[2n]{x_1 x_2 \ldots x_{2n}}.\]\\Plugging this in proves this for $I_{2n}$\\
Finally, we'll perform the third and final (S) this time we'll assume that $I_n$ holds and prove that so does $I_{n-1}$, causing all the dominos to topple.\\
Letting $x_n = \frac{x_1 + x_2 + \cdots + x_{n-1}}{n-1}$, we have that\[\frac{x_1 + x_2 + \cdots + x_{n-1} + \frac{x_1 + x_2 + \cdots + x_{n-1}}{n-1}}{n} \geq \sqrt[n]{x_1 x_2 \cdots x_{n-1} \left(\frac{x_1 + x_2 + \cdots + x_{n-1}}{n-1}\right)}.\]Because we assumed AM-GM in $n$ variables, equality holds if and only if $x_1 = x_2 = \cdots = x_{n-1} = \frac{x_1 + x_2 + \cdots + x_{n-1}}{n-1}$. However, note that the last equality is implied if all the numbers of $x_1, x_2, \ldots, x_{n-1}$ are the same; thus, equality holds if and only if $x_1 = x_2 = \cdots = x_{n-1}$.

We first simplify the lefthand side. Multiplying both sides of the fraction by $n-1$ and combining like terms, we get that\[\frac{x_1 + x_2 + \cdots + x_{n-1} + \frac{x_1 + x_2 + \cdots + x_{n-1}}{n-1}}{n} = \frac{nx_1 + nx_2 + \cdots + nx_{n-1}}{n(n-1)} = \frac{x_1 + x_2 + \cdots + x_{n-1}}{n-1}.\]Plugging this into the earlier inequality yields\[\frac{x_1 + x_2 + \cdots + x_{n-1}}{n-1} \geq \sqrt[n]{x_1 x_2 \cdots x_{n-1} \left(\frac{x_1 + x_2 + \cdots + x_{n-1}}{n-1} \right)}.\]Raising both sides to the $n$th power yields\[\left( \frac{x_1 + x_2 + \cdots + x_{n-1}}{n-1}\right)^n \geq x_1 x_2 \cdots x_{n-1}\left(\frac{x_1 + x_2 + \cdots + x_{n-1}}{n-1}\right).\]From here, we divide by $\frac{x_1 + x_2 + \cdots + x_{n-1}}{n-1}$ and take the $(n-1)^{\textrm{th}}$ root to get that\[\frac{x_1 + x_2 + \cdots + x_{n-1}}{n-1} \geq \sqrt[n-1]{x_1 x_2 \cdots x_{n-1}}.\]This is $I_{n-1}$. \\
With the most advanced thing we have used and will probably use in induction, we can finally say:\\
Hence, proved.\\
\end{proof}
This theorem is the king of inequalities. A lot of people believe it to be one of the most important facts in maths. It stands for 'arithmetic mean, geometric mean'. While it is quite a well known inequality, it's proof is less known. If you ever find yourself having crush on a math lover, just present this proof from memory. Thank me later.\\
Now back to math, We can use AM-GM to get the following things:\\
$a^2+b^2 \geq 2ab$ and $a^3+b^3+c^3 \geq 3abc$\\
You can sum such inequalities to solve simple questions.\\
\begin{example}
    For $a, b, c > 0$ prove that $a^2+b^2+c^2 \geq ab + bc + ca$
\end{example}
\begin{proof}
    As we just discussed above, $a^2+b^2 \geq 2ab$, $b^2+c^2 \geq 2bc$, $c^2+a^2 \geq 2ac$. Adding them gives us:\\
    $2a^2+2b^2+2c^2 \geq 2ab+2bc+2ca\\
    \therefore a^2+b^2+c^2 \geq ab+bc+ca$\\
\end{proof}
\begin{example}
    For $a, b, c > 0$ prove that $a^4+b^4+c^4 \geq a^2bc+ab^2c+abc^2$
\end{example}
\begin{proof}
    $a^4+a^4+b^4+c^4 \geq 4a^2bc$\\
    $a^4+b^4+b^4+c^4 \geq 4ab^2c$\\
    $a^4+b^4+c^4+c^4 \geq 4abc^2$\\
    Adding these gives us the inequality.
\end{proof}
In most symmetric inequality, both sides will be equal when
we set all variables equal.\\
Moreover, we often compare expressions which are the same degree, or homogeneous. For example when we write $a^2 + b^2 + c^2 \geq ab + bc + ca$, both sides are degree 2. (Notice that the AM-GM inequality itself has the same property!) \\
The reason for this is $x^5$ and $x^3$ are not comparable for generic $x > 0$, since the behaviors when $x$ is very small and x is very large are different. So a non-homogeneous inequality like $a^2+b^2+c^2 \geq a^3+b^3+c^3$ will definitely not be true in general, since the behaviors if $a=b=c=0.01$ and $a=b=c=100$ will be different. You may have already picked up some intuition: more “mixed” terms are smaller. For example, for degree 3, the polynomial $a^3 +b^3 +c^3$ is biggest and $3abc$ is the smallest. Roughly, the more “mixed” polynomials are the smaller.\\
This intuition can be formalized as Muirhead inequality\\
\begin{definition}
     If $x_1+x_2+\dots + x_k \geq y_1 + y_2+ \dots + y_k$ for all $0\leq k \leq n$ then ${x_1,x_2,\dots,x_n} \succ {y_1,y_2,\dots,y_n}$
\end{definition}
The $\succ$ symbol in speaking changes to majorizes. Using this definaion, we can say
   \begin{theorem}
   [Muirhead]
    \[\sum_{\text{sym}}a_1^{x_1}a_2^{x_2}\dots a_n^{x_n} \geq \sum_{\text{sym}}a_1^{y_1}a_2^{y_2}\dots a_n^{y_n}\]
    If $x_1,x_2,x_3 \dots x_n \succ y_1,y_2,y_3 \dots y_n$
\end{theorem}
Since, $(5,0,0)\succ(3,1,1)\succ(2,2,1)$\\
$a^5+a^5+b^5+b^5+c^5+c^5 \geq a^3bc+a^3bc+ab^3c+ab^3c+abc^3+abc^3 \geq a^2b^2c+a^2b^2+ab^2c^2+ab^2c^2+a^2bc^2+a^2bc^2$\\
Note: It is symmetric, not cyclic. Our only tool for cyclic is still AM-GM.\\
\section{Non-homogeneous equations}
\subsection{Even's Substitution}
If an inequality has the condition $abc = 1$, one can also sometimes use the substitution $(a, b, c) = (x/y, y/z, z/x)$ which will transform it into a homogeneous inequality automatically. I call it the Even's substitution as I first learnt it from Evan Chan's handouts(Its not an official name.).
\begin{example}
    Prove that if $abc = 1$ then $a^2 + b^2 + c^2 \geq a + b + c$
\end{example}
\begin{proof}
    Using Evan's substitution, the problem statement transforms to
    $\frac{x^2y^4+y^2z^4+z^2x^4}{x^2y^2z^2}\geq \frac{xy^2+yz^2+zx^2}{xyz}\\
    \iff x^2y^4+y^2z^4+z^2x^4 \geq x^2y^3z + xy^2z^3 + x^3yz^2$\\
    Using, $4x^2y^4+y^2z^4+z^2x^4 \geq 6x^2y^3z$, and adding its cycles, we get the inequality.
\end{proof}
\subsection{Ravi's Substitution}
Sometimes, an inequality will refer to the $a, b, c$ as the sides of a triangle. In that case, one can replace $(a, b, c) = (y + z, z + x, x + y)$ where $x, y, z > 0$ are real numbers using the properties of in-circle. \\
\begin{figure} [h]
    \centering
    \includegraphics[width=0.5\linewidth]{Photos/Ravi's substitution.png}
    \caption{Ravi's Substitution}
    
\end{figure}
This is colloquially known as the Ravi substitution, after Ravi Vakil, a well known math professor at Stanford and an IMO gold medalist(and 4 time Putnam fellow)
\begin{example}
     Find the smallest constant k such that
$k>\frac{a^2+b^2+c^2}{ab+bc+ca}$
where $a,b,c$ are the sides of a triangle.
\end{example}
\begin{proof}
    Using Ravi's Substitution,\\
    $\frac{a^2+b^2+c^2}{ab+bc+ca}\\
    =\frac{(x+y)^2+(y+z)^2+(z+x)^2}{(x+y)(y+z)+(y+z)(z+x)+(x+z)(x+y)}\\
    =\frac{2(x^2+y^2+z^2+xy+yz+xz)}{x^2+y^2+z^2+3(xy+yz+zx)}$\\
    Hence we are looking for k such that:\\
    $k(x^2+y^2+z^2)+3k(xy+yz+zx)> 2(x^2+y^2+z^2)+2(xy+yz+xz)\\
    (3k-2)(xy+yz+zx)>(2-k)(x^2+y^2+z^2)$\\
    We already know that $x^2+y^2+z^2 > xy+yz+zx$; Hence,\\
    $(3k-2)(x^2+y^2+z^2)>(2-k)(x^2+y^2+z^2)\\
    3k-2 > 2-k\\
    4k > 4\\
    k > 1$\\
    Thus, minimum value of k is 1.
\end{proof}
\subsection{Schur's Inequality}
It's canny how the three tools for non homogeneity all are named after someone but for completely different reasons. This one was discovered by Issai Schur, the Russian mathematician and is hence named after him.
\begin{theorem}
[Schur's Inequality]
     For all non-negative $a,b,c \in \mathbb{R}$ and $r>0$:
\[a^r(a-b)(a-c)+b^r(b-a)(b-c)+c^r(c-a)(c-b) \geq 0\]
The four equality cases occur when $a=b=c$ or when two of $a,b,c$ are equal and the third is ${0}$.
\end{theorem}
\begin{proof}
    Let $a\geq b\geq c$.\\
    $\therefore a^r(a-b)(a-c)+b^r(b-a)(b-c)\\
    = a^r(a-b)(a-c)-b^r(a-b)(b-c)\\
    = (a-b)(a^r(a-c)-b^r(b-c))$\\ 
    Clearly, $a^r \geq b^r \geq 0$, and $a-c \geq b-c \geq 0$.\\ 
    Thus, $(a-b)(a^r(a-c)-b^r(b-c)) \geq 0\\
    \therefore a^r(a-b)(a-c)+b^r(b-a)(b-c) \geq 0$.\\ 
    As, $c^r(c-a)(c-b) \geq 0\\
    \therefore a^r(a-b)(a-c)+b^r(b-a)(b-c)+c^r(c-a)(c-b) \geq 0$.
\end{proof}

\section{Some advanced inequalities}
\subsection{Extended AM-GM}
\begin{theorem}
    [Weighted Power Mean] For $a_1,a_2,a_3 \dots a_n$ positive reals and $w_1,w_2, w_3, \dots w_n$ where $w_1+w_2+w_3+\dots +w_n=1$. For any real number R, we'll define the weighted power mean as:\\
   \[P(x)=\begin{cases}
       (w_1a_1^r+w_2a_2^r \dots +w_na_n^r)^{1/r} & r \neq 0\\
       a_1^{w_1}a_2^{w_2} \dots a_n^{w_n} & r=0\\
   \end{cases}\]
   Then, if $k>l$ then $P(k) \geq P(l)$. Equality occurs when $a_1=a_2=\dots=a_n$\\
    \end{theorem}
\begin{theorem}
    [Simplified Weighted Mean] The $r^{th}$ power mean refers to:\\
    \[p(x)=\begin{cases}
       (\frac{a_1^r+a_2^r \dots +a_n^r}{n})^{1/r} & r \neq 0\\
       \sqrt[n]{a_1a_2\dots a_n} & r=0\\
   \end{cases}\]\\
   Basically, we have set $w_1=w_2=\dots=w_n=\frac{1}{n}$. The inequality still holds true, if $k>l$ then $P(k) \geq P(l)$. Equality occurs when $a_1=a_2=\dots=a_n$
\end{theorem}
Using the Simplified Weighted Mean, and setting $r$ to $(2,1,0,-1)$ respectively will give us:\\
\begin{theorem}[RMS-AM-GM-HM]
    \begin{align*}
\sqrt{\frac{a_1^2 + a_2^2 + \ldots + a_n^2}{n}} &\geq \frac{a_1 + a_2 + \ldots + a_n}{n} \\
&\geq \sqrt[n]{a_1 \cdot a_2 \cdot \ldots \cdot a_n} \\
&\geq \frac{n}{\frac{1}{a_1} + \frac{1}{a_2} + \ldots + \frac{1}{a_n}}
\end{align*}
\end{theorem}
I'll not prove any of these. While RMS-AM-GM-HM can be proven using the same mechanism we used to prove AM-GM, the weighted form will use some math which we are yet to explore. That technique appears in chapter 15\\
\subsection{Cauchy-Schwarz(aka SEBACS)}
The following theorem is taught under many names. Cauchy had originally proposed it, it was generalized by Schwarz and Bunyakovsky, given an Olympiad friendly form by Arthur Engel and popularized by Titu Andreescu and Nairi Sedrakyan.  I propose it being called SEBACS Inequality.
\begin{theorem}
    [Orignal Form]
    for any list of reals $a_1, a_2, \ldots, a_n$ and $b_1, b_2, \ldots, b_n$,\[(a_1^2 + a_2^2 + \cdots + a_n^2)(b_1^2 + b_2^2 + \cdots + b_n^2) \geq (a_1b_1 + a_2b_2 + \cdots + a_nb_n)^2,\]\\
\end{theorem}
\begin{theorem}
[Mordern Form]
for any list of reals $a_1, a_2, \ldots, a_n$ and $b_1, b_2, \ldots, b_n$:\\
\[\frac{ a_1^2 } { b_1 } + \frac{ a_2 ^2 } { b_2 } + \cdots + \frac{ a_n ^2 } { b_n } \geq \frac{ (a_1 + a_2 + \cdots+ a_n ) ^2 } { b_1 + b_2 + \cdots+ b_n }.\] 
\end{theorem}
Proving any of these will require us to use vectors which we'll learn later(in geometry). With them it will become quite trivial and we'll prove it then. The non-geometrical proof is found in chapter 15\\
For now, let's try an example:
\begin{example}
    for positive $a, b, c$, prove that
$\frac{a}{b+c} + \frac{b}{c+a} + \frac{c}{a+b} \ge \frac{3}{2}$,
\end{example}
\begin{proof}
    By SEBACS Inequality:\\
    $[\sqrt{(b+c)}^2 + \sqrt{(c+a)}^2 + \sqrt(a+b)^2]\left( \sqrt{\frac{1}{b+c}}^2 + \sqrt{\frac{1}{c+a}}^2 + \sqrt{\frac{1}{a+b}}^2 \right) \geq (\sqrt{\frac{b+c}{b+c}}+\sqrt{\frac{a+c}{a+c}}+\sqrt{\frac{b+a}{b+a}})^2$,\\
    Upon expanding this gives us:\\
    $2\left( \frac{a+b+c}{b+c} + \frac{a+b+c}{c+a} + \frac{a+b+c}{a+b} \right) \geq 9\\
   \iff \frac{a}{b+c} + \frac{b}{c+a} + \frac{c}{a+b} \ge \frac{3}{2}$
\end{proof}
\section{A note}
A lot of more advanced inequalities need calculus for complete exploration. Hence, we've decided to explore them in greater detail in Chapter-15.  Inequalities tend to not occur in lower levels and occur at an advanced state at higher level. Hence, we have been unable to include a lot of PYQs here. 
\begin{xcb}{Exercises}
\begin{enumerate}
\item Prove that $a^5 + b^5 + c^5 \geq a^3bc + b^3ca + c^3ab \geq abc(ab + bc + ca).$
\item (PRMO) What is the largest positive integer n such that $\frac{a^2}{\frac{b}{29} +\frac{c}{31}} +\frac{b^2}{\frac{c}{29}+\frac{a}{31}} +\frac{c^2}{\frac{a}{29}+\frac{b}{31}}\geq n(a+b+c)$
\item For a,b,c the sides of some triangle, prove the inequality $\dfrac{1}{b+c-a}+\dfrac{1}{c+a-b}+\dfrac{1}{a+b-c}\ge\dfrac1a+\dfrac1b+\dfrac1c$ 
\item (Candian MO) For positive reals $a,b,c$, \\
\[\frac{a^3}{bc}+\frac{b^3}{ac}+\frac{c^3}{ab} \geq a+b+c\]
\item Prove that for any non-negative $a,b,c$:\\
\[\frac{a}{b+c}+\frac{b}{a+c}+\frac{c}{a+b}\geq \frac{3}{2}\]
\item Prove that:\\
$\sqrt{a_1^2+a_2^2+\dots}+\sqrt{b_1^2+b_2^2+\dots} \geq \sqrt{(a_1+b_1)^2+(a_2+b_2)^2+\dots}$
\item If $a,b,c>0$, prove that:\\
\[a^3b+b^3c+c^3a \geq abc(a+b+c)\]\\
\item (IMO 1995) Let $a, b, c$ be positive real numbers such that $abc = 1$. Prove that\[\frac{1}{a^3(b+c)} + \frac{1}{b^3(c+a)} + \frac{1}{c^3(a+b)} \geq \frac{3}{2}.\]
\item If $\frac{1}{a}+\frac{1}{b}+\frac{1}{c}=1$, then $(a + 1)(b + 1)(c + 1) \geq 64$\\
\item . If $abcd = 1$, then $a^4b + b^4c + c^4d + d^4a \geq a + b + c + d$.
\item Prove that:\\
$\frac{a}{b+c}+\frac{b}{a+c}+\frac{c}{a+b} \geq  \frac{3}{2}$\\
for any non-negative $a, b, c$\\
\item If $a, b, c > 0$ prove that\\
$abc(a + b + c) \geq a^3b + b^3c + c^3a$\\
\item For $a,b,c > 0$, prove that:\\
$abc \geq (a + b - c)(b + c - a)(c + a - b)$
\item Let $a, b, c, d$ be non-negative real numbers such that $a + b + c + d = 1$. Find the minimum value of $a^2 + b^2 + c^2 + d^2$\\
\item (AMC 10) Let $A$, $M$, and $C$ be nonnegative integers such that $A+M+C=10$. What is the maximum value of $A\cdot M\cdot C+A\cdot M+M\cdot C+C\cdot A$?
\item (AMC 10) Let $a,b,$ and $c$ be real numbers such that
\[a+b+c=2, \text{ and}\]\[a^2+b^2+c^2=12\]
What is the difference between the maximum and minimum possible values of $c$?
\item  (MPFG) Find the least real number K such that for all real numbers $x$ and $y$, we have
$(1 + 20x^2)(1 + 19y^2)\geq Kxy$
\item Prove that \[\frac{x}{x+y}+\frac{y}{y+z}+\frac{z}{z+x} \leq 2\]
\end{enumerate}
\end{xcb}
\chapter{Sequence and Series}
I normally like to start this chapter with a very well known story. Let's go to a classroom in somewhere near 1785. The math teacher is a man who is not interested in teaching and wants to chat with the female school workers. So he decides that he must keep you all engaged and quiet while he tries to get a life partner. He writes on the board: $1+2+3+\dots +100$ Do not speak until complete\\
Saying that he leaves. While everyone else is diligently adding, one of the kid stands up after a minute and walks outside the class to find the teacher. "The answer is $5050$" he announces.\\
"It's wrong!" The teacher replies, as he doesn't really know the correct answer. He just figures that a 8 year old couldn't add this quickly.\\
The child replies, "Sir, its correct."\\
"Are you the teacher or am I?"\\
"With all due respect, sir it's correct."\\
"See, you could by no means add so quickly. You are just guessing."\\
"Sir, I didn't add much. I multiplied."\\
"What?"\\
"$1+100=101$, $2+99=101$ and so on, $50+51=101$. We have $50$ such pairs. So the answer must be $50*101=5050$."
The teacher patted the child, impressed. "Well that's something."\\
The child was the great mathematician Carl Friedrich Gauss. The teacher, although more intrested in love than math, took notice of Gauss's talents and worked actively with the university to nurture him.\\
\section{Some notes on notation}
\begin{definition}
    A sequence is an ordered list of numbers. A series is the sum of that list.
\end{definition}
This chapter will also be the first time we'll use the summation notation. Why use it instead of just some terms, $\dots$ and more terms? Here is a quote from Calculus II for Dummies.
\begin{quote}
    Mathematicians just love sigma notation for two reasons. First, it provides a convenient way to express a long or even infinite series. But even more important, it looks really cool and scary, which frightens non-mathematicians into revering mathematicians and paying them more money.
\end{quote}
\begin{definition}
    In the summation notation or sigma notation:\\
    \[\sum^b_{a=1}f(a)=f(1)+f(2)+\dots +f(b-1)+f(b)\]
\end{definition}
By the definition itself, we can state two very useful properties of sigma notation.
\begin{theorem}
Distributively
    \[\sum kf(x)=f\sum f(x)\]\\
Commutativity
    \[\sum[f(x)+g(x)]=\sum f(x) + \sum g(x)\]
\end{theorem}
And that's all I wish to tell you about notation. Let's come back to math now.
\section{Arithmetic Progression}
An arithmetic progression is the more a sequence of numbers such that the difference between any two consecutive terms is constant. This constant is called the common difference of the sequence. $8,11,14,17$ is an AP\\
More formally: \\
\begin{definition}
    The sequence $a_1, a_2, \ldots , a_n$ is an arithmetic progression if and only if $a_2 - a_1 = a_3 - a_2 = \cdots = a_n - a_{n-1}$
\end{definition}
Now Lets consider $a_1, a_2, a_3 \dots a_n$ an AP with common difference $d$.
\begin{theorem}
    The $n^{th}$ theorem of an Arithmetic Sequence is:\\
    \[a_{n}=a+(n-1)d\]
\end{theorem}
This follows from the definition. 
\begin{theorem}
    Number of terms in an AP:\\
    \[n=\frac{a_n-a_1}{2}+2\]
\end{theorem}
This follows from the $n^{th}$ term formula.
\begin{theorem}
    Average of n terms of AP:\\
    \[\frac{a_1+a_n}{2}\]
\end{theorem}
This also follows from the definition of AP.
\begin{theorem}
    Sum of n terms of an AP:
    \[\frac{a_1+a_n}{2}\cdot n\]\\
    \[=\frac{a_1+a_1+(n-1)d}{2}\cdot n\]\\
    \[=\frac{2a_1+(n-1)d}{2}\cdot n\]
\end{theorem}
This follows from the definition of average.\\
With these trivial observations in your toolkit, you can now make some remarkable observations.\\
Let's consider the AP $1,2,3,\dots,n$ with $a_1=1$ and $d=1$. Then\\
\begin{theorem}
     \[\sum^n_{k=1}k=\frac{n+1}{2}\cdot n=\frac{n(n+1)}{2}\]
\end{theorem}
The numbers which can be written in the form of $\frac{n(n+1)}{2}$ are called trianguler numbers. This is because a given thoose number of balls, we can arrange them into a perfect triangle with a base of $n$. \\
The theorem above can also be shown without words 
\begin{figure}
    \centering
    \includegraphics[width=0.5\linewidth]{Photos/triangle numbers.png}
    \caption{A proof without words for triangle numbers}
\end{figure}
You can also realize that this is exactly what Gauss did.
Now let's consider the AP $2,4,8,16 \dots 2n$
\begin{theorem}
    \[\sum^n_{k=1}2k=2\sum^n_{k=1}k=2*\frac{n(n+1)}{2}=n(n+1)\]
\end{theorem}
And finally, we'll consider the AP $1,3,5,7 \dots 2n-1$. It's sum is quite beautiful:\\
\begin{theorem}
    \[\sum^n_{k=1}2k-1=\sum^n_{k=1}2k-\sum^n_{k=1}1=n(n+1)-n=n(n+1-1)=n^2\]
\end{theorem}
This can also be basically proven without words as:\\
\begin{figure}
    \centering
    \includegraphics[width=0.5\linewidth]{Photos/Sum of odds.png}
    \caption{Counting the number of squares in the alternating black and white gives us the proof}
    
\end{figure}
Now using whatever we just learnt, let's obliterate some questions.\\
\begin{example}
    (AIME 2012) The terms of an arithmetic sequence add to $715$. The first term of the sequence is increased by $1$, the second term is increased by $3$, the third term is increased by $5$, and in general, the $k$th term is increased by the $k$th odd positive integer. The terms of the new sequence add to $836$. Find the sum of the first, last, and middle terms of the original sequence.
\end{example}
\begin{proof}
    [Solution]
    Let's first realize that the question is asking the sum of first, last and the middle term of the AP. We also know that the middle term is the average of first and last term. So basically, we are looking for three times the middle term.\\
    Now let's find it. Let the sequence of $a_1,a_2,a_3 \dots a_n$:\\
    $\sum^n_{i=1}a_i=715$
    and adding the odd numbers\\
    $sum^n_{i=1}a_i+2n-1=835\\
    \iff sum^n_{i=1}a_i+sum^n_{i=1}2n-1=835\\
    \iff 715+n^2=835\\
    \iff n=11$\\
    Hence, we have $11$ terms in the sequence. As the sum is $715$, the middle term(or the average) is $65$. Hence, our answer is $65*3=195$\\
\end{proof}
\begin{example}
    (AIME 2005, edited)For each positive integer $k$, let $S_k$ denote the increasing arithmetic sequence of integers whose first term is $1$ and whose common difference is $k$. For example, $S_3$ is the sequence $1,4,7,10,\ldots.$ For how many values of $k$ does $S_k$ contain the term $2023$?
\end{example}
\section{Geometric Progression}
A lot of math jokes begin with: An infinite amount of mathematicians walk into a bar.\\
This can have one continuation for every mathematician in the bar. The most classic one is: The first one asks for a beer. The second one asks for half a beer. The third one asks for a quarter of a beer. The fourth one asks for an eighth of a beer... . The bartender pours two beers, saying "You guys ought to know your limits".\\
Let's try to understand what just happened. A geometric progression, is a sequence of numbers such that the ratio between any two consecutive terms is constant. This constant is called the common ratio of the sequence. For example, $1, 2, 4, 8$ is a geometric sequence with common ratio $2$ and $100, -50, 25, -25/2$ is a geometric sequence with common ratio $-1/2$; however, $1, 3, 9, -27$ and $-3, 1, 5, 9, \ldots$ are not geometric sequences, as the ratio between consecutive terms varies.
More formally, \\
\begin{definition}
    The sequence $a_1, a_2, \ldots , a_n$ is a geometric progression if and only if $a_2 / a_1 = a_3 / a_2 = \cdots = a_n / a_{n-1}$
\end{definition}
Let $g_1, g_2, g_3 \dots g_n$ be a GP with common ratio $r$.\\
\begin{theorem}
    The $n^{th}$ term in a GP is:
    \[g_n=g_1\cdot r^{n-1} \]
    \[g_n=g_m\cdot r^{n-m} \]    
\end{theorem}
This follows from the definition.
\begin{theorem}
    Sum of first n terms of a GP is:\\
    \[g_1\cdot \frac{r^n-1}{r-1}\]\\
    \textbf{NOTE: $\frac{r^n-1}{r-1}$ may be replaced with $\frac{1-r^n}{1-r}$ whenever convenient.}
\end{theorem}
\begin{proof}
This is one of the most beautiful proof in all of mathematics.
Let $S_n = a_1 + a_2 + \dots + a_n$\\
This can be re-written as:
$S_n = a_1 + a_1r + \dots + a_1r^{n-1}$
Multiplying both sides by $r$,
$S_nr = a_1r + a_1r^2 + \dots + a_1r^n$
Subtracting the original equation from this equation,\\
$S_nr - S_n = a_1r^n - a_1\\
\iff S_n(r-1)=a_1(r^n-1)\\
\iff S_n=\frac{a_1(r^n-1)}{r-1}$
\end{proof}
Quite cool, but here is something even more cool.
\begin{theorem}
    Sum of infinite terms of a converging GP:\\
    \[\frac{g_1}{1-r}\]\\
    \textbf{NOTE: This only holds for $-1\geq r \geq 1$, as other series sum keeps increasing to either $\infty$ or $-\infty$}
\end{theorem}
\begin{proof}
The proof is quite similar to above. With the only difference being that $n \rightarrow \infty$
Let $S_n = a_1 + a_2 + \dots$\\
This can be re-written as:
$S_n = a_1 + a_1r + \dots$
Multiplying both sides by $r$,
$S_nr = a_1r + a_1r^2 + \dots $
Subtracting this equation from the original equation,\\
$S_n-S_nr = a_1\\
\iff S_n(1-r)=a_1\\
\iff S_n=\frac{a_1}{1-r}$
\end{proof}
Now we can truly understand the joke at the start of the chapter.\\
The first mathematician asked for 1 beer, The second for half, the third for a qurter. This is a GP with $a=1$ and common ration $r=\frac{1}{2}$. So we can easily add it using the above formula:\\
$\frac{1}{1-\frac{1}{2}}=\frac{1}{\frac{1}{2}}=2$\\
Now let's blast some questions:\\
\begin{example}
    (AIME 2002) Two distinct, real, infinite geometric series each have a sum of $1$ and have the same second term. The third term of one of the series is $1/8$, and the second term of both series can be written in the form $\frac{\sqrt{m}-n}p$, where $m$, $n$, and $p$ are positive integers and $m$ is not divisible by the square of any prime. Find $100m+10n+p$.
\end{example}
\begin{proof}
    [Solution]
    Let the second term of both the series be $x$. Then the first term of the first series is $8x^2$ and the common ratio $\frac{1}{8x}$. As it's infinite sum is $1$,\\
    $1=\frac{8x^2}{1-\frac{1}{8x}}\\
    \iff 64x^3-8x+1=0\\
    \iff (4x-1)(16x^2+4x-1)=0
    \iff x=\frac{1}{4}, \frac{\sqrt{5}-1}{8}, \frac{-\sqrt{5}-1}{8}$\\
    As $x$ is of the form $\frac{\sqrt{m}-n}p$\\
    $\therefore x=\frac{\sqrt{5}-1}{8}\\
    \therefore m=5, n=1, p=8$\\
    Hence, $100m+10n+p=518$\\
\end{proof}
\begin{example}
    The sum of the first $2011$ terms of a geometric sequence is $200$. The sum of the first $4022$ terms is $380$. Find
the sum of the first $6033$ terms.
\end{example}
\section{Induction and Summation}
While many of the summations are quite trivial to derive, some of them cannot be derived. Only proven. These are typically observed and then proved using a technique called induction.\\
Almost everyone has once had fun arranging dominoes(or jenga bricks) in a row and starting a wave. Push the first domino and it topples the second, the second will topple the third and so forth until all the dominoes are toppled. Now let us change the set of dominoes into an infinite series of propositions: $P_1 , P_2, P_3 \dots$. Assume that we can prove
that:\\
(B): that $P_1$ of the series is true;\\
(S): If $P_k$ is true, than $P_{k+1}$ is also true.\\
Then, in fact, we will have proven that all the propositions in the series are true. We just pushed the first domino and proved that every dominoes is close enough that it will topple with the falling of its predesceor. \\
This is a description of the method of mathematical induction (MMI). Theorem (B) is called base of induction, and theorem (S) is the inductive step.\\
This is much better understood by actually seeing it being used:\\
\begin{theorem}
        \[\sum^n_{k=1}k^2=\frac{n(n+1)(2n+1)}{6}\]
\end{theorem}
\begin{proof}
    (B): For n=1, $\frac{1*2*(2*1+1)}{6}=1^2=1$. Hence, it is true for $n=1$\\
    (S): Let $\sum^m_{k=1}k^2=\frac{m(m+1)(2m+1)}{6}$ be true for some $m$. Then:\\
    $\sum^{m+1}_{k=1}k^2\\
    = (m+1)^2+\sum^m_{k=1}k^2\\
    = (m+1)^2 + \frac{m(m+1)(2m+1)}{6}\\
    = \frac{6(m+1)^2+m(m+1)(2m+1)}{6}\\
    = \frac{(m+1)(2m^2+m+6m+6}{6}\\
    = \frac{(m+1)(2m^2+7m+6}{6}\\
    =\frac{(m+1)(2m+3)(m+2)}{6}\\
    =\frac{(m+1)(m+1+1)(2(m+1)+1)}{6}$\\
    As now both (B) and (S) have been proved, our theorem is also true.
\end{proof}
We'll leave one as exercise for you. It's almost the same, but with cube.\\
\begin{theorem}
        \[\sum^n_{k=1}k^3=(\frac{k(k+1)}{2})^2\]
\end{theorem}
\section{Telescoping}
Telescoping is a method of writing a summation out in such a way that a lot of it gets canceled. While naturally such series don't occur a lot, they do appear in human generated numbers. A very common use is in the dividend discount model. You may have heard that the stock prices rise and plummet on the basis of demand and supply. How and why of this is guarded by mathematical algorithms. The Dividend discount model prices a stock based on the amount of money it will pay to the holder in dividends discounted to present value. Let's break that down. A stock is basically a part, a stake in the company you have. Basically, companies want you to hold their stocks. So they pay yearly dividends to the holders. Basically, its your part of the pie for owning a part of the company. It's value is not decided by a person but an algorithm which based on market patterns predicts what the stock will pay you in dividends over perpetuity(a very long period of time, technically $\infty$, however taken as 5-10 year period for the purpose of economic calculations).\\
However, you might have heard people say that 'inflation is rising'. Your parents or grandparents probably would have said that in their days the things were cheaper. So whatever the stock is expected to pay you in the future, will be affected by inflation. So we adjust it for the same.\\
If a company pays us the same amount and the rate of inflation is constant, we'll have a simple infinite GP. However, stocks with constant dividends are less liked, hence companies increase the dividends from time to time. This leads to an infinite summation which normally gives us the value of the stock.\\
For example: ITC, Indian Tobacco Company, had a divided of 5.25 INR in 2012 and it has increased annually by an average 2.85\%. The inflation in Indian market has increased by an average of 6.02\% in the same time period, ouch!. So based on this data, its price should be $5.25+5.25*(1+\frac{2.85}{100})*(1-\frac{6.02}{100})+\dots = \sum^{\infty}_{n=1}5.25*((\frac{2.85}{100})*(1-\frac{6.02}{100}))^{n-1}\\
=\sum^{\infty}_{n=1}5.25*0.9683^(n-1)$\\
This is now a summation we can compute. While this is not telescopic, note that the numbers were averages. Exact dividend growth and inflation are different every year which will lead to a telescope, but we'll not demonstrate that here.\\
Solving this, we get $165.61$ as the price which is shockingly close to its price in 2012 mid year when dividends were paid(167).\\
The actual price is found using estimates for inflation and other data which are useful but messy functions. However, as these numbers are based on human activity or selected by humans, they tend to lead to telescoping.\\
Telescoping is something that is also a very common question pattern in Olympiads. We basically ask you to compute some strange summation which can be broken down into subtraction of a few functions which happen to cancel out leaving very little computation to be done. Just like a telescope.\\
\begin{example}
    [Motivating Example]
    (Stanford 2011) Evaluate the sum:\\
    \[\sum^{\infty}_{k=1}\frac{7k+32}{k(k+2)}\cdot\frac{3^k}{4^k}\]
\end{example}
\begin{proof}
    [Solution]
    Decomposing the denominator:
    \[\frac{7k+32}{k(k+2)}=\frac{16}{k}-\frac{9}{k+2}\]\\
    As $9=16*\frac{3^2}{4^2}$\\
    \[\therefore \sum^{\infty}_{k=1}\frac{7k+32}{k(k+2)}\cdot\frac{3^k}{4^k} = \frac{16}{k}*\frac{3k}\\
    {4^k}-\frac{9}{k+2}*\frac{3^k}{4^k}\]\\
    \[=\frac{16}{k}*\frac{3^k}{4^k}-\frac{16}{k+2}*\frac{3^{k+2}}{4^{k+2}}\]
    We'll only need to solve for $k=1,2$ as rest will get canceled. Thus, the answer is: $16*\frac{3}{4}+8*\frac{9}{16}=12+\frac{9}{2}=\frac{33}{2}$
\end{proof}
In the motivating example, in order to cancel things out, we decomposed the denominator to reach a solution. The thig is that this can be done with every factorizeable denominator and is of great value in integral calculus.\\
\begin{table}[htbp]
  \centering
  \caption{Rational Functions and Their Decompositions}
  \label{tab:rational-functions}
  \begin{tabular}{|c|c|}
    \hline
    \textbf{General Form} & \textbf{Decomposition} \\
    \hline
    $\frac{px+q}{(x-a)(x-b)}$ & $\frac{A}{x-a}+\frac{B}{x-b}$ \\
    \hline    
    $\frac{px+q}{(x-a)(x-a)^2}$ & $\frac{A}{x-a}+\frac{B}{(x-a)^2}$ \\
    \hline
    $\frac{px^2+qx+r}{(x-a)(x-b)(x-c)}$ & $\frac{A}{x-a}+\frac{B}{x-b}+\frac{C}{x-c}$\\
    \hline
   $\frac{px^2+qx+r}{(x-a)^2(x-c)}$ & $\frac{A}{x-a}+\frac{B}{(x-a)^2}+\frac{C}{x-b}$\\
    \hline
    $\frac{px^2+qx+r}{(x-a)(x^2+bx+c)}$ & $\frac{A}{x-a}+\frac{Bx+C}{x^2+bx+c}$\\
  \end{tabular}
\end{table}
$A,B,C$ need to be figured out separately, which you can do in two ways. I'll demonstrate them:\\
\begin{example}
    \[\sum^{\infty}_{n=4}\frac{n}{n^3-6n2^+11n-6)}\]
\end{example}
\begin{proof}
    [Solution]
    We can factorize $n^3-6n2^+11n-6$ as $(n-1)(n-2)(n-3)$.\\
    Therefore we can decompose $\frac{n}{n^3-6n2^+11n-6)} = \frac{A}{(n-1} + \frac{B}{n-2} + \frac{C}{n-3}\\
    \therefore A(n-2)(n-3)+B(n-1)(n-3)+C(n-1)(n-2)=n$\\
    Here the paths diverge. We'll start with Method I, which is more widely published, but much slower.\\
    $\iff A(n^2-5n+6) + B(n^2-4n+3) + C(n^2-3n+2)=n\\
    \iff (A+B+C)n^2-(5A+4B+3C)n+(6A+3B+2C)=0n^2+n+0\\
    \iff A+B+C=0; 5A+4B+3C=-1; 6A+3B+2C=0\\
    \iff 2A+B=-1; 4A+B=0 \text{This follows by repeated subtraction of the first equation}\\
    \iff A=\frac{1}{2}\\
    \therefore A=\frac{1}{2}; B=-2; C=\frac{3}{2}$\\
    If we move back to the fork in the road, we can now use method II, which is less commonly known but much faster.\\
    As we have, $A(n-2)(n-3)+B(n-1)(n-3)+C(n-1)(n-2)=n$, If $n-1=0 \iff n=1$ then:
    $A(1-2)(1-3)+B(1-1)(1-3)+C(1-1)(1-2)=1\\
    \iff A(-1)(-2)=1\\
    \iff A=\frac{1}{2}$\\
    We can now take $n-2=0 \iff n=2$ and then $n-3=0 \iff n=3$ giving us $B=-2$ and $C=\frac{3}{2}$. It's so simple that you can probably do it in your head.\\
    Now let's destroy the question:\\
    $\sum^{\infty}_{n=4}\frac{n}{n^3-6n2^+11n-6)}\\
    = \sum^{\infty}_{n=4} \frac{1}{2(n-1)}-\frac{2}{n-2}+\frac{3}{2(n-3)} \\
    = \sum^{\infty}_{n=4} \frac{1}{2(n-1)}-\frac{4}{2(n-2)}+\frac{3}{2(n-3)} \\
    = \frac{1}{2}\sum^{\infty}_{n=4} \frac{1}{(n-1)}-\frac{4}{(n-2)}+\frac{3}{(n-3)} \\
    = \frac{1}{2} \sum^{\infty}_{n=4} \frac{1}{(n-1)}-\frac{4}{(n-2)}+\frac{3}{(n-3)}$ \\
    To find the summation notice that the diagonals of the table below are summing to zero. I've crossed one out to make it more clear. The rest terms also get canceled other than the three values in the right corner, but they have not been crossed to make sure the pattern is evident.
    \begin{tabular}{|c|c|c|}
    \hline
    $\frac{1}{(n-1)}$ & $\frac{-4}{(n-2)}$ & $\frac{3}{(n-3)}$ \\
    \hline
    $\cancel{\frac{1}{3}}$ & $\frac{-4}{2}$ & $\frac{3}{1}$ \\
    $\frac{1}{4}$ & $\cancel{\frac{-4}{3}}$ & $\frac{3}{2}$ \\
    $\frac{1}{5}$ & $\frac{-4}{4}$ & $\cancel{\frac{3}{3}}$ \\
    $\frac{1}{6}$ & $\frac{-4}{5}$ & $\frac{3}{4}$ \\
    $\frac{1}{7}$ & $\frac{-4}{6}$ & $\frac{3}{5}$ \\
    \vdots & \vdots & \vdots \\
    \hline
\end{tabular}
Hence we can now say that:\\
$\frac{1}{2} \sum^{\infty}_{n=4} \frac{1}{(n-1)}-\frac{4}{(n-2)}+\frac{3}{(n-3)}\\
= \frac{1}{2} \cdot (\frac{-4}{2}+3+\frac{3}{2})\\
= \frac{1}{2} \cdot \frac{5}{2}\\
=\frac{5}{4}$
\end{proof}
That was quite beautiful! Let's do another question.
\begin{example}
(USAMTS 1999, edited)
    $\sqrt{1+\frac{1}{1^2}+\frac{1}{2^2}}+ \sqrt{1+\frac{1}{2^2}+\frac{1}{3^2}}+\sqrt{1+\frac{1}{3^2}+\frac{1}{4^2}}+\dots+ \sqrt{1+\frac{1}{2023^2}+\frac{1}{2024^2}}$
\end{example}
\begin{proof}
    [Solution]
    The question is basically:\\
    \[\sum^{2023}_{n=1}\sqrt{1+\frac{1}{n^2}+\frac{1}{(n+1)^2}}\]\\
    The square root seems to be the most irritatig part of the question. Let's see if we can somehow get rid. What will happen if we try to simplify the stuff inside the radical?\\
    $1+\frac{1}{n^2}+\frac{1}{(n+1)^2}\\
    =\frac{(n(n+1))^2+(n+1)^2+n^2}{(n(n+1))^2}\\
    = \frac{n^4+2n^3+3n^2+2n+1}{(n(n+1))^2}\\
    = \frac{n^4+n^3+n^2+n^3+n^2+n+n^2+n+1}{(n(n+1))^2}\\
    = \frac{n^2(n^2+n+1)+n(n^2+n+1)+1(n^2+n+1)}{(n(n+1))^2}\\
    =\frac{n^2+n+1}{n(n+1)}^2$\\
    Voila! The square root has been defeated. Now all is left is to solve whatever remains have been left.\\
    $\sum^{2023}_{n=1}\sqrt{1+\frac{1}{n^2}+\frac{1}{(n+1)^2}}\\
    =\sum^{2023}_{n=1}\sqrt{\frac{n^2+n+1}{n(n+1)}^2}\\
    =\sum^{2023}_{n=1}\frac{n^2+n+1}{n(n+1)}\\
    =\sum^{2023}_{n=1}\frac{n(n+1)+1}{n(n+1)}\\
    =\sum^{2023}_{n=1}1+\frac{1}{n(n+1)}\\
    =\sum^{2023}_{n=1}1+\sum^{2023}_{n=1}\frac{1}{n(n+1)}\\
    =2023+\sum^{2023}_{n=1}\frac{1}{n(n+1)}\\
    =2023+\sum^{2023}_{n=1}\frac{1}{n}-\frac{1}{n+1}\\
    =2023+\frac{1}{1}\cancel{-\frac{1}{2}+\frac{1}{2}}\cancel{-\frac{1}{3}+\frac{1}{3}} \dots \cancel{\frac{1}{2022}}-\cancel{\frac{1}{2023}+\frac{1}{2023}}-\frac{1}{2024}\\
    = 2023+1-\frac{1}{2024}\\
    =2023 +\frac{2023}{2024}$
\end{proof}
\section{Recurrence Series}
Let's for our final story, go to $12^{th}$ century. We are in Pisa. The construction for a white bell tower has just began. The church despite scientific advice, has decided to go for it in the land available. The subsoil seems a bit unsteady, but the architect is confident.\\
A mathematician has bought a pair of young male and female rabbits from the animal seller as his Christmas present to himself. He leaves them in his backyard. In January, they are all grown up. By February, they give birth to another pair which happens to be off opposite genders.\\
In March, the original pair gives birth to two more baby rabbits and the first children are of reproductive age. In April, we get two new pairs of rabbits and the second born are of reproductive age.
\begin{figure}
    \centering
    \includegraphics[width=0.75\linewidth]{Photos/Rabbits fibbonacci.png}
    \caption{The rabbit graph, source: Murderous Math: The Key to Universe}
    
\end{figure}
We can notice that the number of pairs of rabbits in a given month follow the sequence are:\\
$1, 1, 2,3, 5, 8, \dots$\\
The pattern is evident. The mathematician's name was Fibonacci and this sequence is called the Fibonacci sequence after him.\\
What happened of the rabbits? They were plentiful in Pisa(couldn't override the town as they neither live or breed for eternity)  and most either lived happily chewing on grass near the white tower. It is believed that their descendants are still there, chewing on the grass, unaware about their contribution.\\
But back to math, a series which depends on its previous terms for its further terms is called a recursive series. We studied about them from a combinatorial perspective in chapter 7. Most of it still applies, and I recommend that you go through it if you haven't already.\\
\begin{xcb}{Exercises}
\begin{enumerate}
\item (AIME) Find the eighth term of the sequence $1440, 1716, 1848,\dots$ whose terms are formed by multiplying the corresponding terms of two arithmetic sequences.
\item(AIME) Two geometric sequences $a_1, a_2, a_3, \ldots$ and $b_1, b_2, b_3, \ldots$ have the same common ratio, with $a_1 = 27$, $b_1=99$, and $a_{15}=b_{11}$. Find $a_9$.
\item (AIME) For $-1<r<1$, let $S(r)$ denote the sum of the geometric series\[12+12r+12r^2+12r^3+\cdots .\]Let $a$ between $-1$ and $1$ satisfy $S(a)S(-a)=2016$. Find $S(a)+S(-a)$
\item (AIME) An infinite geometric series has sum 2005. A new series, obtained by squaring each term of the original series, has 10 times the sum of the original series. The common ratio of the original series is $\frac mn$ where $m$ and $n$ are relatively prime integers. Find $m+n.$
\item (AIME) Call a $3$-digit number geometric if it has $3$ distinct digits which, when read from left to right, form a geometric sequence. Find the difference between the largest and smallest geometric numbers.
\item (AIME) The sum of an infinite geometric series is a positive number $S$, and the second term in the series is $1$. What is the smallest possible value of $S?$
\item (AIME) In an increasing sequence of four positive integers, the first three terms form an arithmetic progression, the last three terms form a geometric progression, and the first and fourth terms differ by $30$. Find the sum of the four terms.
\item (AIME) A sequence of positive integers with $a_1=1$ and $a_9+a_{10}=646$ is formed so that the first three terms are in geometric progression, the second, third, and fourth terms are in arithmetic progression, and, in general, for all $n\ge1,$ the terms $a_{2n-1}, a_{2n}, a_{2n+1}$ are in geometric progression, and the terms $a_{2n}, a_{2n+1},$ and $a_{2n+2}$ are in arithmetic progression. Let $a_n$ be the greatest term in this sequence that is less than $1000$. Find $n+a_n.$
\item (AIME) Consider the sequence defined by $a_k =\dfrac{1}{k^2+k}$ for $k\geq 1$. Given that $a_m+a_{m+1}+\cdots+a_{n-1}=\dfrac{1}{29}$, for positive integers $m$ and $n$ with $m<n$, find $m+n$.
\item (AIME) Given that
\begin{align*}x_{1}&=211,\\ x_{2}&=375,\\ x_{3}&=420,\\ x_{4}&=523,\ \text{and}\\ x_{n}&=x_{n-1}-x_{n-2}+x_{n-3}-x_{n-4}\ \text{when}\ n\geq5, \end{align*}
find the value of $x_{531}+x_{753}+x_{975}$.
\item (AIME) The sequence $\{a_n\}$ is defined by\[a_0 = 1,a_1 = 1, \text{ and } a_n = a_{n - 1} + \frac {a_{n - 1}^2}{a_{n - 2}}\text{ for }n\ge2.\]The sequence $\{b_n\}$ is defined by\[b_0 = 1,b_1 = 3, \text{ and } b_n = b_{n - 1} + \frac {b_{n - 1}^2}{b_{n - 2}}\text{ for }n\ge2.\]Find $\frac {b_{32}}{a_{32}}$.
\item (AIME 2004) Consider the sequence defined by $a_k =\dfrac{1}{k^2+k}$ for $k\geq 1$. Given that $a_m+a_{m+1}+\cdots+a_{n-1}=\dfrac{1}{29}$, for positive integers $m$ and $n$ with $m<n$, find $m+n$.
\item (AIME 1989) If the integer $k$ is added to each of the numbers $36, 300,$ and $596$ one obtains
the squares of three consecutive terms of an arithmetic series. Find $k$.
\item Compute this sum:\\
$(\frac{1}{2^2}+\frac{1}{3^2}+\dots)+(\frac{1}{2^3}+\frac{1}{3^3}+\dots)+\dots$
\item (Putnam 2013,edited) For positive integers $n$, let the numbers $c(n)$ be determined by the rules: $c(1) = 1, c(2n) = c(n),$ and $c(2n + 1) = (-1)^nc(n)$. Find the value of:\\
\[\sum_{n=1}^{2023}c(n)c(n+2)\]
\item If $a, b,c$ form an arithmetic progression, and
$a = x^2 + xy + y^2\\
b = x^2 + xz + z^2\\
c = y^2 + yz + z^2$
where $x + y + z = 0$, prove that $x$, $y$, and $z$ also form an arithmetic progression.
\item (IOQM 2023) The sequence $a_n$ with $n\geq 0$ is defined by $a_0 = 1, a_1 = -4$ and $a_{n+2} = -4a_{n+1}- 7a_n$ for $n \geq 0$ . Find $a_{50}^2 - a_{49}a_{51}$
\item \item (AMC 12) \points{5} Three balls are randomly and independently tossed into bins numbered with the positive integers so that for each ball, the probability that it is tossed into bin i is $2^{-i}$ for $i = 1, 2, 3, \dots$. More than one ball is allowed in each bin. The probability that the balls end up evenly spaced in distinct bins is $p/q$ , where p and q are relatively prime positive integers. (For example, the balls are evenly spaced if they are tossed into bins 3, 17, and
10.) What is $p + q$?

\end{enumerate}
\end{xcb}

\part{The Red Pill}
\chapter{Calculus I: Limits, Continuity and Derivative}
\begin{quote}
    Calculus, the mathematics of change, and change, mysterious. Some things grow imperceptibly... others zoom... hair grows slowly and suddenly cut... temperatures rise and fall... smoke curls through the air... planets wheel through space... and time, time never stops... - Cartoon Guide to Calculus
\end{quote}
Imagine you're sitting in a car with your family, driving along a winding mountain road. The scenery outside is breathtaking, with lush green forests, towering cliffs, and a deep blue sky. As you cruise along, you can't help but notice how the world around you is constantly transforming.\\
Now, let's say you're curious about how fast things are changing. You're particularly interested in how quickly your car is ascending the mountain. You pull out your trusty stopwatch and start measuring the time it takes to climb each mile of the road. You also jot down your car's speed at each mile marker.\\
As you collect this data, something remarkable happens. You realize that the speed of your car is not constant—it varies with each mile. Sometimes you're going uphill at a slow pace, and other times you're speeding down a hill, faster than you thought possible.\\
What's even more intriguing is that you notice a pattern: the steeper the incline, the slower your car seems to go. Conversely, when the road levels out or slopes downward, your car speeds up. This variation in speed as you move along the road is what we call...\\
\section{The Rate of Change}
In a still shot of a baseball game, the ball is clear, not moving at the instant of the photo. A good camera can take many such photos each with the ball clear and stopped.\\ \begin{figure} [h]
    \centering
    \includegraphics[width=0.5\linewidth]{Photos/Baseball calc.png}
    \caption{Is the ball moving? How do you know?\\
    courtesy, Champaign News-Gazette and photographer Robin Scholtz, 2003}    
\end{figure}
So if in any instant the ball is stopped, how is it truly moving? This is known as Zeno's paradox. The answer is rather simple. The ball, like every other moving object, has an unseen invisible quantity called velocity which is dictating its speed and direction.\\
While we can define velocity as rate of change of distance(displacement is more physically accurate as it also encompasses the direction), what about changing velocity?\\
As a baseball is thrown in the air, its velocity decreases as it slows down before it stops at its highest point. It then comes down, speeding up along the way.\\
However, here is a small issue. The speedometer of the car was showing you a speed all the time without ever actually having memory of what time it is and how much distance we had travelled.\\ 
How did that work? The speedometer was showing us the speed in the last 0.01 sec or even smaller. That was the cars instantaneous speed.\\
If this feels sort of strange, it should. This is all kind of paradoxical, as we are trying to look for the speed of something in a snapshot, which is not possible as we cannot divide by zero.\\
\begin{figure} [h]
    \centering
    \includegraphics[width=0.5\linewidth]{Photos/Calculus tangent.png}
    \caption{The graph at heart of calculus}
    
\end{figure}
If the $f(x)$ is out distance function and time is along the $x$ axis, we can represent velocity between time $x_0$ and $x_0+\Delta x$ as the slope of the line between $f(x_0)=P$ and $f(x_0+\Delta x)=Q$.\\
As $\Delta x \rightarrow 0$, $f(x_0+\Delta x)=Q \rightarrow P$, which makes the secant between $P$ and $Q$ a tangent at $P$ of which we can find the slope off.\\
This is essentially what your speedometer does, it measures the slope of the the tangent of your distance graph.\\
This is essentially what calculus is all about.\\
\section{Some functions}
We will end up using some functions again and again in calculus. These functions using in combination with each other make up most of the things we see in physics, which is the major use of calculus. I will take this moment to remind you that a function is a relation from a domain to co-domain.\\
\subsection{Constant Function}
A simple function which returns the same value even in plugging out different values. Domain is  $x \in \mathbb{R}$, while the co-domain is a single Real number. \\
\begin{figure} [h]
    \centering
    \includegraphics[width=0.5\linewidth]{Photos/Constant function.png}
    \caption{The constant function}
    
\end{figure}
\subsection{Power Function}
These are the functions with formulas $x, x^2, x^3, \dots x^n$, where $n$ is a positive integer. When n is even, these functions all have bowl-shaped graphs as $(-x)^{2n} = x^{2n}$.\\ If n is odd, then $(-x)^{2n-1} = -(x^{2n-1})$, causing the graphs bend downward on the left.
\begin{figure} [h]
    \centering
    \includegraphics[width=0.5\linewidth]{Photos/Power Function.png}
    \caption{Power Function}
    
\end{figure}
\subsection{Polynomials}
Polynomials are basically made by multiplying powers by constants and adding them. You already know from algebra that a polynomial with degree $n$ has at most $n$ roots or $n$ points where the graph cuts zero. We can also notice that every polynomial either goes to $\infty$ or $-\infty$, so every time it cuts at zero more than once, it will also have a turning. So basically,  it has at most $n-1$ turnings. The turnings will become use full in a minute.
\subsection{Negative powers}
We can also write negative powers using the fact $x^{-n}=\frac{1}{x^n}$. Their graph also varies according to the parity of $n$.\\
\begin{figure} [h]
    \centering
    \includegraphics[width=0.5\linewidth]{Photos/Negative powers.png}
    \caption{Negative Powers}
    
\end{figure}
\subsection{Fractional Powers}
We can also plot $x^{\frac{1}{n}}=\sqrt[n]{x}$. Also without any surprise, you might have already understood that it also depends on pairity.\\
\begin{figure} [h]
    \centering
    \includegraphics[width=0.5\linewidth]{Photos/Fractional Powers.png}
    \caption{Fractional Powers}
    
\end{figure}
\subsection{Exponential functions}
Power functions: We get large pretty quickly\\
Exponential functions: Hold my cup...\\
Exponential functions are defined as $f(x)=a^x$ where $a$ is constant.  Physicists like using $e$ as the base where $e$ is Euler's number and is equal to $2.718 \dots$\\
Like $\pi$ it is irrational, however unlike $\pi$, it's definition is not geometric.  It was arrived upon by Jacob Bernoulli while pondering the given question\\
\begin{example}
    An account starts with $1$ dollar and pays $100$ percent interest per year. If the interest is credited once, at the end of the year, the value of the account at year-end will be $2$ dollars. What happens if the interest is computed and credited more frequently during the year?
\end{example}
Using our knowledge of sequence and series, we could compute that if the interest was compounded $n$ times an year, we would get $(1+\frac{1}{n})^n$.  If we keep making $n$ larger and larger, till it approaches $\infty$, we'll be left with $e$. How exactly? We'll talk about that in just a minute \\
We can using algebra also say that $e^r=a$ will have an unique solution for $r$ as long as $a \neq 0$. This is used so often that we have a notation for it: $\ln(a)=r$ where $\ln$ refers to the natural logarithm. It's properties are discussed in greater detail in the logarithm's chapter. However, using only the fact that it exists we can say $f(x)=e^{rx}$. \\
This graph grows exponentially for $r>0$, is constent for $r=0$ and decays exponentially for $r<0$.\\
\subsection{Trigonometric Functions}
If you are unfamiliar with trigonometry, I recommend reading through that chapter before continuing this .\\
\begin{figure} [h]
    \centering
    \includegraphics[width=0.5\linewidth]{Photos/Trigonometric circle.png}
    \caption{Meet the unit circle}
    
\end{figure}
Here we have a circle of radius $1$ and make two perpendicular lines through the center. A radii is made which goes from the center to some point on the circumference $P$ making an angle $\theta$  with the horizontal, which is measured in radians. Radians are another way to measure angles which is the norm in physics. It basically is the arc length of the arc subtended by your angle divided by the radius.  Physicist's like it better as it doesn't use an abstract number 360 as the reference for the angles instead uses an physical quantity\\
But back to the diagram, The perpendicular distance of this from the $y$ axis is called $\cos\theta$ and the perpendicular distance from the $x$ axis is called $\sin(\theta)$ . We define $\tan(\theta) =\frac{\sin(\theta)}{\cos(\theta)}$\\
A trigonometric function refers to $f(x)=\sin(x);f(x)=\cos(x); f(x)=\tan(x)$ or  a combination of them with coefficients  where the domain is $x \in \mathbb{R}$ and the co-domain for $\sin(x)$ and $\cos(x)$ is $\{-1,1\}$. The co-domain is all real numbers for $\tan(x)$. \\
\begin{figure} [h]
    \centering
    \includegraphics[width=0.5\linewidth]{Photos/Trigonometric functions.png}
    \caption{Trigonometric Functions}
    
\end{figure}
\subsection{Composing Functions}
A composing function is defined as $f(x)=h(g(x))$ basically, a combination of two functions. NOTE: While sometimes, $h(g(x))=g(h(x))$ this is often untrue. We should not interchange the functions unless we are absolutely sure we can.\\
\subsection{Inverting Functions}
If composing function $f(x)=h(g(x))=C$ where C is a constant then $h(x)$ is the inverse of $g(x)$ and vice versa. The inverse of a function $k(x)$ is sometimes denoted as $k^{-1}(x)$.
You may feel that an inverse composition may be switched around but consider this: $g(x)=x^2, h(x)=\sqrt{x}$ then while $f(x)=g(h(x))$ has a domain of all positive real numbers, $f(x)=h(g(x))$ has a domain of all real numbers; making them fundamentally different.\\
\section{Inverse Trigonometric functions}
This difference is even more profound with trigonometric functions(whose inverse are written as $\sin^{-1},\cos^{-1},\tan^{-1}$ and have domains of $(-1,1); (-1,1); \theta \in \mathbb{R}$ respectively.)\\
We need to note that while $\sin(x)$ keeps on going up and down, it is strictly increasing in the range $\frac{-\pi}{2}\leq x \leq \frac{\pi}{2}$. Fun fact: A lot of people refer to $\sin^{-1}$ as $\arcsin$. This is because it basically corresponds to the length of the arc the radii is subtending given the perpendiculer distence from the vertical.\\
\begin{figure} [h]
    \centering
    \includegraphics[width=0.5\linewidth]{Photos/Arcsin unit circle.png}
    \caption{The return of the unit circle}
\end{figure}
I expect you to find the co-domain of $\arccos$ by yourself now.
\begin{example}
    Find the co-domain of $\arccos$
\end{example}
Let's talk about $\arctan$. The co-domain of this is pleasantly surprising as it has a massive domain.\\
\begin{figure} [h]
    \centering
    \includegraphics[width=0.5\linewidth]{Photos/Arc tan Unit Circle.png}
    \caption{The Unit Circle Awakens}
    
\end{figure}
\section{Derivative}
As derivative is the slope of a function $f(x)$ at a particular point, we can also plot the derivative as a function $f'(x)$ as we move that particular point.\\
\begin{theorem}
[Differentiation formula]
    \[f'(x)=\frac{f(x+h)-f(x)}{h}, h \rightarrow 0\] 
\end{theorem}
This is just formalization of the definition of derivative.\\
Using this definition, we can quite easily figure some more things.\\
\begin{theorem}
[Two truths of differentiation]
    $(f(x)+g(x))'=f'(x)+g'(x)\\
    \therefore (Cf(x))'=Cf'(x)$ where $C$ is a constant
\end{theorem}
While this seems both trivial and extraordinary, here is the proof
\begin{proof}
    $(f(x)+g(x))'\\
    =\frac{f(x+h)+g(x+h)-f(x)-g(x)}{h}\\
    = \frac{f(x+h)-f(x)}{h}+\frac{g(x+h)-g(x)}{h}\\
    = f'(x)+g'(x)$
\end{proof}
We will also prove some standard derivatives which you should remember(just the identity, the proof is obvious).
\begin{theorem}
[Power rule of differentiation]
    If $f(x)=x^n$,\\
    $f'(x)=nx^{n-1}$
\end{theorem}
\begin{proof}
    $f'(x)=\frac{{x+h}^n-x^n}{h}\\
    = \frac{x^n+nx^{n-1}h+\dots+nxh^{n-1}+h^n -x^n}{h}
    = nx^{n-1}+\binom{n}{2}x^{n-2}h+\dots +nxh^{n-2}+h^{n-1} \text{Using } h \rightarrow 0\\
    =nx^{n-1}$
\end{proof}
With this much, we are now qualified enough to find the derivative of all polynomials.
\begin{theorem}
    If $f(x)=\sin(x)$, then:\\
    $f'(x)=\cos(x)$\\
    If $f(x)=\cos(x)$, then:\\
    $f'(x)=-\sin(x)$
\end{theorem}
\begin{proof}
    $f'(x)=\frac{\sin(x+h)-\sin(x)}{h} \text{By the trigonometric property for } \sin(\alpha+\beta)\\
    = \frac{\sin(x)\cos(h)+\cos(x)\sin(h)-\sin(x)}{h}\\
    = \cos(x)\frac{\sin(h)}{h}+\sin(x)\frac{\cos(h)-1}{h}$
    We can notice that $\sin(h)=h$ for small values of $h$ as the perpendicular distance to horizontal and the arc length are almost equal.\\
    We also need to notice that $\cos(h)-1$ is almost equal to $0$ as the perpendicular distance to vertical is almost equal to the radius of the unit circle which is $1$.\\
    $\therefore \cos(x)\frac{\sin(h)}{h}+\sin(x)\frac{\cos(h)-1}{h}\\
    = \cos(x)\frac{h}{h}+\sin(x)\frac{0}{h}
    = \cos(x)$
\end{proof}
I'll leave the full proof for the derivative of $\cos(x)$ to you but the simple, non trigonometric, proof without words is in noticing that $\cos(x)$ is exactly like $\sin(x)$ shifted left by $\frac{\pi}{2}$.\\
\begin{figure} [h]
    \centering
    \includegraphics[width=0.5\linewidth]{Photos/Cos derivative proof without words.png}
    \caption{A proof without words}
    
\end{figure}
We'll come to $\tan(x)$ in a minute.
\begin{theorem}
    If $f(x)=e^x$:\\
    $f'(x)=e^x$\\
    $\therefore f(x)=a^x=f'(x) \iff a=e$.
\end{theorem}
This is what makes $e$ so special.
\begin{proof}
Let $f(x)=a^x=f'(x)$
    $f'(x)=\frac{a^{x+h}-a^x}{h}=a^x\\
    \iff a^x\frac{a^h-1}{h}=a^x\\
    \iff \frac{a^h-1}{h}=1\\
    \iff a^h=1+h$
    Remember the definition of $e$?, This is where that definition becomes significant(we'll prove it, wait for it). As $e=(1+\frac{1}{n})^n, n \rightarrow \infty$, we can set $h \rightarrow \frac{1}{n} \rightarrow 0$\\
    $\therefore e=(1+h)^{\frac{1}{h}}\\
    \therefore e^h=1+h$\\
    Which implies, $a=e$\\
    Hence proved.
\end{proof}
One of my most favorite proofs.
\section{Some more notation}
Calculus has had a dark history. Newton and Leibniz contested on who had invented it. The thing is that they both assumed that the other had copied it. Historical records show that the discovery was 1. independent, 2. already discovered approximately 2000 years ago by the Greek, Chinese, Arabic and Indian mathematicians 3. already in use in different forms by modern mathematicians and physicists like Galileo and Kepler\\
So in classic European fashion, they took something that was already there, gave it a name and then argued over who 'discovered' it.\\
What they did do was give us some sort of a better notation system. Till now we were using the Newton's notation where derivatives are shown as $f'(x)$ but we'll use Leibniz notation a bit more in the future as it simplifies a lot of things which Newton would complicate.\\
In Leibniz notation, the derivatve is taken of an equation rather than of a function. For example:\\
$y=x^2\\
\therefore \frac{dy}{dx}=2x$
Here, $\frac{dy}{dx}$ implies the derivative. The $d$ here is a version of $\Delta$, which is what we use to show big change, but $d$ shows microscopic changes. Basically, this is just the original meaning of differentiation.\\
We also need to notice that we just multiplied the entire equation by $\frac{d}{dx}$ and solved the derivatives and got the result.\\
This makes the two truths feel like an elementary fact.\\
\section{Product and Quotient rules}
\begin{theorem}
[Product rule]
    $(f(x)g(x))'=f'(x)g(x)+f(x)g'(x)$
\end{theorem}
\begin{proof}
We'll take $f(x+h)=f(x)+\Delta(f)$ where $h,\Delta(f) \rightarrow 0$\\
    $\therefore (f(x)g(x))'=\frac{f(x+h)*g(x+h)-f(x)g(x)}{h}\\
    = \frac{(f(x)+\Delta(f)*(g(x)+\Delta(g)-f(x)g(x)}{h}\\
    = \frac{f(x)g(x)+g(x)\Delta(f)+f(x)\Delta(g)+\Delta(f)\Delta(g)-f(x)g(x)}{h}\\
    = g(x)\frac{\Delta(f)}{h}+f(x)\frac{\Delta(g)}{h}+\frac{\Delta(f)\Delta(g)}{h}$\\
    Using the fact that $\Delta(f),\Delta(g) \rightarrow 0$ and $f(x+h)-f(x)=\Delta(f)$ and $g(x+h)-g(x)=\Delta(g)$\\
    $g(x)\frac{\Delta(f)}{h}+f(x)\frac{\Delta(g)}{h}+\frac{\Delta(f)\Delta(g)}{h}\\
    = g(x)\frac{f(x+h)-f(x)}{h}+f(x)\frac{g(x+h)-g(X)}{h}\\
    =g(x)f'(x)+g'(x)f(x)$
\end{proof}
Also I'd like you to note that, by the same proof, if we want to differentiate more than two functions we can go for: $(fgh\dots)'=f'gh\dots+fg'h\dots+fgh'\dots$\\
\begin{theorem}
[Reciprocal rule]
    $(\frac{1}{f(x)})'=\frac{-f'(x)}{(f(x))^2}$
\end{theorem}
\begin{proof}
    $(\frac{1}{f(x)})'=\frac{\frac{1}{f(x+h)}-\frac{1}{f(x)}}{h}\\
    =\frac{\frac{f(x)-f(x+h)}{f(x+h)f(x)}}{h}\\
    =\frac{-f'(x)}{f(x+h)(f(x)} \text{Using } h \rightarrow 0\\
    =\frac{-f'(x)}{(f(x))^2}$
\end{proof}
Combining the product and reciprocal rule gives us:\\
\begin{theorem}
    [Quotient Rule]
    $(\frac{f(x)}{g(x)})'=\frac{f'(x)g(x)-f(x)g'(x)}{(g(x))^2}$
\end{theorem}
We are now powerful enough to solve a few more differential questions:\\
\begin{example}
    For $f(x)=\tan(x)$, compute:\\
    $f'(x)$
\end{example}
\begin{proof}
    [Solution]
    We know that the formula for $\tan{\alpha+\beta}$ is quite messy. However, we are also know that $\tan(\theta) = \frac{\sin(\theta)}{\cos(\theta)}$ as well as the derivatives for $\sin(\theta)$ and $\cos(\theta)$, this seems like an use the quotient rule.\\
    $\therefore f'(x)=\frac{(\sin(x))'\cos(x)-(\cos(x))'\sin(x)}{\cos^2(x)}\\
    =\frac{\cos^2(x)+\sin^2(x)}{\cos^2(x)}$\\
    We know that $\sin^2(x)+\cos^2(x)=1$ using the unit circle,\\
    $\therefore \frac{\cos^2(x)+\sin^2(x)}{\cos^2(x)}\\
    = \frac{1}{\cos^2(x)}\\
    = \sec^2(x)$\\
    The final step is by the definition that $\frac{1}{\sin}=\csc; \frac{1}{\cos}=\sec; \frac{1}{\tan}=\cot$
\end{proof}
I will leave it to you as exercise to differentiate the rest of the trigonometric functions by yourself.\\
\begin{theorem}
    $x^{-n}=-nx^(-(n+1))$
\end{theorem}
While, this is exactly what the power rule states. It's proof is different from that of the power rule as we can't use the binomial expansion here. However, it's trivial as we are just using the reciprocal rule and the power rule together. I expect that you'll take it upon yourself to prove it once for practice.
\section{Chain rule}
While we can differentiate a lot of things, we still fail to differentiate things like $e^{2x}$ or $\sin(\cos(x))$.\\
Here is where the chain rule comes.\\
\begin{theorem}
    $f(g(x))'=g'(x)f'(g(x))$
\end{theorem}
This probably looks worse than it actually is. It's proof will appear in a minute. But till then let's do a question to understand this better.\\
\begin{example}
    $\frac{d}{dx} e^{2x}$
\end{example}
\begin{proof}
    [Solution]
    The chain rule basically states that for the derivative of $f(g(x))$ we will first take $f'(x)$ and plug $g(x)$ in place of $x$. We'll then multiply that by $g'(x)$\\
    In this case, $f(x)=e^x \implies f'(x)=e^x$ and $g(x)=2x \implies g'(x)=2$, therefore:\\
    $(e^{2x})'=2e^{2x}$
\end{proof}
\begin{example}
    $\frac{d}{d\theta}\sin(\cos(\theta))$
\end{example}
I believe this one will be a cake walk for you to do.\\
Now let's talk about inverses. The chain rule can solve for the inverse given we know the derivative of the original function.\\
\begin{theorem}
    [Inverse Rule]
    $(f^{-1}(x))'=\frac{1}{f'(f^{-1}(x)}$
\end{theorem}
\begin{proof}
    $x=f(f^{-1}(x))\\
    \iff \frac{dx}{dx}=\frac{d}{dx}f(f^{-1}(x))\\
    \iff 1=f{'}^{-1}(x) \cdot f'(f^{-1}(x))\\
    \iff (f^{-1})'(x)=\frac{1}{f'(f^{-1}(x)}$
\end{proof}
If you use this formula on $x^{\frac{1}{n}}$, you'll get what we get by the power rule. If you think that's cool, hold my cup:\\
\begin{theorem}
    $\frac{d}{dx} \ln(x)=\frac{1}{x}$
\end{theorem}
This looks wild. How did this even happen?\\
\begin{proof}
    As $f(x)=\ln(x)$ is the inverse of $g(x)=e^x \implies g'(x)=e^x$ , we can use the inverse rule.\\
    $\frac{d}{dx} \ln(x) = \frac{1}{e^{\ln(x)}}=\frac{1}{x}$
\end{proof}
This obviously doesn't happen with other functio...\\
\begin{theorem}
    $\frac{d}{dx} \arcsin(x)=\frac{1}{\sqrt{1-x^2}}$
\end{theorem}
\begin{proof}
    As $f(x)=\arcsin(x)$ is the inverse of $g(x)=\sin(x) \implies g'(x)=\cos(x)$ , we can use the inverse rule.\\
    $\frac{d}{dx} \arcsin(x) = \frac{1}{\cos(\arcsin(x)}$\\
    As $\sin^2(x)+\cos^2(x)=1$, therefore $\cos(x)=\sqrt{1-\sin^2(x)}=\sqrt{1-x^2}$\\
    Plugging in $x=\arcsin(x)$ we get: $\cos(x)=\sqrt{1-x^2}$, therefore:\\
    $\frac{1}{\cos(\arcsin(x)} = \frac{1}{\sqrt{1-x^2}}$
\end{proof}
So maybe this the only trig funct...\\
\begin{theorem}
    $\frac{d}{dx} \arctan(x)=\frac{1}{1+x^2}$
\end{theorem}
\begin{proof}
    As $f(x)=\arctan(x)$ is the inverse of $g(x)=\tan(x) \implies g'(x)=\sec^2(x)$ , we can use the inverse rule.\\
    $\frac{d}{dx} \arctan(x) = \frac{1}{\sec^2(\arctan(x)}$\\
    As $\sec^2(x)-\tan^2(x)=1$(Just write both in terms of $\sin$ and $\cos$ and it will become obvious), therefore $\sec^2(x)=\tan^2(x)+1$\\
    Plugging in $x=\arctan(x)$ we get: $\sec^2(\arctan(x))=\tan^2(\arctan(x))+1=x^2+1$, therefore:\\
    $\frac{1}{\sec^2(\arctan(x)} = \frac{1}{1+x^2}$
\end{proof}
Also, as you might have already guessed. The same happens to every trigonometric function. I leave the proof up to you.\\
The strange part is that nobody has an intuitive explanation why this happens. It just does through proven formulas and we don't question it.\\
But wait, we haven't proven the chain rule. So let's do it.\\
Let's go on a small tangent first. Remember that functions are just arrows from the domain to co-domain?\\
So we can write functions in a parallel view like this.
\begin{figure} [h]
    \centering
    \includegraphics[width=0.5\linewidth]{Photos/Parallel view of function_ Chain rule.png}
    \caption{Parallel view of function}
    
\end{figure}
Here I ask you a simple question, by what scale has $h$ been increased due to the function? $\frac{\Delta f}{h}$ is the obvious answer.\\
But what happens when $h \rightarrow 0$, things start to breakdown as they get smaller, don't they?\\
Let's talk about what we mean to be small. Smallness is relative, a mouse is small in comparison to an elephant. A flea is small in comparison to the mouse. The flea is beneath notice in comparison to the elephant.\\
In terms of math, elephant refers to macro numbers like $x, f(x)$ while they can be zero in some cases, they are mostly not.\\
The increment $h$ is the mouse. Which while very small, is not beyond notice. However, anything which when divided by $h$ is approaches zero can be considered a flea. This makes $h^2, h^3, h^{\frac{4}{3}}$ fleas as $h \rightarrow 0$.\\
We can say that $\frac{\text{flea}}{h}=\text{mouse}$ (as it is approaching 0, not reaching it)\\
We can also say that $h*\text{mouse}=\text{flea}$\\
This all may seem interesting but where are we going with this?\\
Remember the secant and tangent which we used to create the differentiation formula? We are finally going to return to it.\\
We know that $\frac{\Delta f}{h}=f'(x)$ as $h \rightarrow 0$\\
Then we can say $\frac{\Delta f}{h}-f'(x)=0$ as $h \rightarrow 0$\\
But as $h$ approaches $0$, not reaches it,  $\frac{\Delta f}{h}-f'(x)= \text{mouse}$\\
Multiplying by $h$ gives us: $\Delta f = hf'(x)+ \text{flea}$\\
Graphically this is nothing but the fact that as $P$ and $Q$ come closer, the secant and tangent have the difference of the most minuscule flea.\\
This in terms of our parallel view of functions we can now say that the scaling factor was $f'(x)h+\text{flea}$ where the flea can be ignored. This is monumental as we only need the value of $x$ to actually find by what the function becomes.\\
This proves the chain rule almost instantly.\\
\begin{proof}
    \begin{figure} [h]
    \centering
    \includegraphics[width=0.5\linewidth]{Photos/Chain rule proof.png}
    \caption{The chain rule, in parallel form}
\end{figure}
We can say that $\Delta u= u'(x)h$ and therefore:\\
$\Delta v= v'(u(x))\Delta u\\
=v'(u(x))u'(x)h\\
\therefore v(u(x))'=\frac{\Delta v}{h}=v'(u(x))u'(x)h$\\
And we are done.\\
\end{proof}
While all this is sure neat, but what is it actually used for? You'll see it in a minute.\\
\section{Limits}
All this time we have been taking derivatives of functions assuming that they exist. What about when they don't? What about times where a point is not in the domain of $f(x)$ but is in the domain of $f'(x)$? What about all that?\\
Imagine a blind pirate captain named Goldeyes who lands on an island with a mysterious heavy object hidden in a hole. He faces a dilemma: should he try to retrieve this object for his ship's treasure or leave it behind? To help him decide, Goldeyes turns to his two most trusted companions for their opinions. Here are the three possible outcomes:\\
Disagreement: Both companions provide conflicting opinions—one claims it's just a common rock, while the other insists it's the valuable treasure of John Timbers. In this scenario, Goldeyes is left uncertain about the true nature of the object. Consequently, he decides to leave it on the island, as he cannot confidently determine its identity.\\
Partial Information: One of his companions confidently identifies the object as John Timbers' treasure, but the other companion is unsure and cannot confirm. Again, Goldeyes is faced with uncertainty, and he opts to leave the object behind because he lacks a clear understanding of what it truly is.\\
Consensus: In the final case, both companions unanimously agree that the object is the same, whether they both claim it's a rock or John Timbers' gold. Goldeyes, despite being blind and unable to verify the object himself, trusts the consensus of his companions and takes action accordingly. If they both say it's gold, he will take it as treasure(even if it is rock); if they both say it's a rock, he will disregard it as a worthless item(even if it is tresure).\\
How does this have anything to do with calculus? In a minute it will become clear.\\
Till now we were dealing with continuous functions. A continuous graph refers to functions whose graph can be drawn without lifting the pen. However, we need to understand that some graphs are discontinuous at a number of points.\\
Let's consider the function $f(x)=\frac{x^2-9}{x-3}$ which has a domain of $\mathbb{R}-3$ as $\frac{0}{0}$ is undefined. This also makes it discontinuous as the graph has a hole at $x=3$\\
However, for rest of the real numbers, $f(x)=x+3$. So if we look slightly on the left of $x=3$ we can get a number close to $6$ and just on the right of $x=3$ we have a number close to $6$ as well. So we can say that: $\lim_{x \to 3} \frac{x^2-9}{x-3}=6$. We need to understand that like Captain Goldeye's we still have zero idea on what is in the hole, but based on what two people are telling us we are making out mind on what is in the hole. This is called limit of a function $f(x)$ for $x=a$.\\
The limit for $f(x)$ at $x=\alpha$ is said to exist if and only if $\lim_{x \to \alpha^-} f(x)=\lim_{x \to \alpha^+} f(x)$ and then $\lim_{x \to \alpha^-} f(x)=\lim_{x \to \alpha^+} f(x)=\lim_{x \to \alpha} f(x)$\\
However, this rule has an exception. If both the left hand limit(LHL) and the right hand limit(RHL) are going to $\infty$ or $-\infty$ the limit is known as infinite limit. This limit doesn't exist as infinite is very large. Let me explain.\\
For example, if a man from Paris and a woman from Bordeaux leave for Russia, do they meet? Unless they are characters in a rom-com, the event seem unlikely as Russia is quite large.\\
The same happens here. Infinity is quite large, how do we know that both the LHL and RHL meet?\\
\begin{theorem}
[Fundamental theorem of Limits]
    If $\lim_{x \to \alpha}f(x)=l$ and $\lim_{x \to \alpha}g(x)=k$ then:\\
    $\lim_{x \to \alpha}f(x) \pm g(x)=l \pm m\\
    \lim_{x \to \alpha}f(x) \cdot g(x)=l \cdot m\\
    \lim_{x \to \alpha}\frac{f(x)}{g(x)}=\frac{l}{m}\\
    \lim_{x \to \alpha}nf(x)=n\lim_{x \to \alpha}f(x)=nl\\
    \lim_{x \to \alpha}g(f(x))=g(\lim_{x \to \alpha}f(x))=g(l)$ provided $g$ is continuous at $g(x)=m$
\end{theorem}
We can now do some simple questions pertaining to limits.\\
\begin{example}
    Let $f(x) =
\begin{cases}
    \frac{1}{2} & \text{for } x > 1 \\
    0 & \text{for } x = 1 \\
    \frac{1}{2} & \text{for } x < 1
\end{cases}$\\
What is $\lim_{x \to 1} f(x)$?
\end{example}
I recommend you to think about this question and make a prediction about this answer. Do not look at the solution. Done.\\
\\
The answer is not $0$. Notice that the function is discontinous at $x=1$. We have a hole in the graph and what did Captain Goldeyes teach us? We will trust the LHL and RHL even if what they say is untrue. Here slightly less than $1$ would give us $\frac{1}{2}$ and so would slightly greater than $1$. Hence, the limit is $\frac{1}{2}$.\\
It is only in continuous functions(or continuous regions of discontinuous functions) where we can plug in whatever the number is approaching to and expect the limit to be the same. For a point of discontinuity, we will look at the LHL and RHL.\\
While the above examples was 150\% unnecessary and just me trying to be a little tricky and cheeky as this limit has no real significance, we should ask where limits are used?\\
In certain time functions which occur in physics, the discontinuous points lead to indeterminate forms. These are seven things which are indeterminable in math. However, time is continuous so we are certain that the function needs to have a value. This is what limits allows us to do. The indeterminate forms are $\frac{0}{0}, \frac{\infty}{\infty}, 0 \times \infty, \infty - \infty, \infty^0, 0^0, 1^{\infty}$. Through algebraic manipulations, all these forms may be inter-converted.\\
Here are also some forms which a lot of people mistake for indeterminate but are actually not: $\infty+\infty=\infty, \infty \times \infty = \infty, \frac{\alpha}{\infty}=0$ if $\alpha$ is finite.\\ 
Also before someone tries to argue $\infty+\infty=\infty \iff 2=1$ remember, $\frac{\infty}{\infty}$ is indeterminate.\\
Also note: $\frac{\infty}{a}, \frac{a}{0}$ are also indeterminate for $\alpha \in \mathbb{R}$. They are not part of the seven forms as they have a variable as part of their formulation.\\
\begin{example}
    We know that for $\lim_{x \to 0}\frac{x^2}{x}$ no limit exists, as LHL $\neq$ RHL. Does $\lim_{x \to 0}\frac{\lfloor x^2 \rfloor}{x}$ exist(here $\lfloor a \rfloor$ refers to the greatest integer less than or equal to $a$?
\end{example}
Again I encourage you to think about this. Before we see the solution. While this is also like the previous is 150\% unnecessary and just me trying to be a little tricky and cheeky, it touches on some critical concepts.\\
\\
If you answered that the limit doesn't exist, you are wrong. The greatest integer function(GIF) causes a subtle but monumental change. Let's consider the LHL for maybe $x=-0.001$, $\frac{\lfloor {-0.001}^2 \rfloor}{-0.001}=\frac{\lfloor 0.000001 \rfloor}{-0.001}=\frac{0}{-0.001}=0$. If we do the same for RHL we can notice that the GIF has caused LHL $=$ RHL $=0$ and hence the limit is equal to $0$.\\
We can resolve indeterminacy in a few ways. We can Simplify, factorize or rationalize the function like we did with $f(x)=\frac{x^2-9}{x-3}$. We can use certain standard limits which we will learn about in a moment. Or we can make some approximations and solve it, like we did with $f(x)=\frac{\sin(x)}{x}$. The most intuitive and simple one is using approximations. You'll see it destroy questions other methods took time around. The only thing is that you'll have to hold your curiosity as their proof is not gonna appear for quite some time, it is in another chapter. But for now, you'll have to trust me. If you do, then here are some common approximations:\\
\begin{theorem}
[Taylor-Maclaurin Series]
    $e^x=1+\frac{x}{1!}+\frac{x^2}{2!}+\frac{x^3}{3!}+ \dots$ \\
    $a^x=1+\frac{x\ln(a)}{1!}+\frac{x^2\ln^2(a)}{2!}+\frac{x^3\ln^3(a)}{3!}+\dots$
    $\ln(1+x)=x-\frac{x^2}{2}+\frac{x^3}{3}-\dots$ for $-1 < x \leq 1$\\
    $\sin(x)=x-\frac{x^3}{3!}+\frac{x^5}{5!}-\dots$\\
    $\cos(x)=1-\frac{x^2}{2!}+\frac{x^4}{4!}-\dots$\\
    $\tan(x)=x+\frac{x^3}{3}+\frac{2x^5}{15}+\dots$\\
    In general, $f(x)=f(0)+\frac{xf'(0)}{1!}+\frac{x^2f''(0)}{2!}+\frac{x^3f'''(0)}{3!}+\dots$
\end{theorem}
As a even further cheat, we normally use less than the first three terms in the approximations if $x \to 0$, mostly a single term does the trick. I call this the Brahmastra over the mythological weapon in Indian mythology which can destroy anything and everything the universe.\\
Let's explore the concepts more through some problems.\\
\begin{example}
    \[\lim_{x \to 3} \frac{x^2-2x-3}{x^2-4x+3}\]
\end{example}
\begin{proof}
    [Solution]
    $\frac{x^2-2x-3}{x^2-4x+3}\\
    = \frac{(x-3)(x+1)}{(x-3)(x-1)}\\
    =\frac{x+1}{x-1}\\
    \therefore lim_{x \to 3} \frac{x^2-2x-3}{x^2-4x+3}
    = lim_{x \to 3} \frac{3+1}{3-1}$ as the new function is continuous at $x=3$\\
    $= \frac{4}{2}=2$
\end{proof}
\begin{example}
    \[\lim_{x \to 2+\sqrt{3}}\frac{x^4-7x^3+14x^2-7x+1}{x^2-4x+1}\]
\end{example}
\begin{proof}
    [Solution]
    $\frac{x^4-7x^3+14x^2-7x+1}{x^2-4x+1}
    = \frac{(x^2-3x+1)(x^2-4x+1)}{x^2-4x+1}\\
    = x^2-3x+1\\
    \therefore \lim_{x \to 2+\sqrt{3}}\frac{x^4-7x^3+14x^2-7x+1}{x^2-4x+1}\\
    = \lim_{x \to 2+\sqrt{3}} x^2-3x+1$ as $x^2-3x+1$ is continuous at $x=2+\sqrt{3}$, we can plug it in to get the answer as $2+\sqrt{3}$ \\
\end{proof}
So those two were quite standard. Let's do something more fun!
\begin{example}
    \[\lim_{x \to 1} \frac{3-\sqrt{8x+1}}{5-\sqrt{24x+1}}\]
\end{example}
\begin{proof}
    [Solution]
    We can solve this in two ways. While one is more tedious other is just chad. We'll do the tedious one first.\\
    To prevent the need to type the limit again and again, assume it's presence throughout the solve.\\
    $\frac{3-\sqrt{8x+1}}{5-\sqrt{24x+1}}\\
    = \frac{3-\sqrt{8x+1}}{5-\sqrt{24x+1}}\cdot\frac{3+\sqrt{8x+1}}{5+\sqrt{24x+1}}\cdot\frac{5+\sqrt{24x+1}}{3+\sqrt{8x+1}}$\\
    Notice that the last fraction doesn't have any indeterminacy, and is continuous and can hence be separated from the limit using the properties of limits. Therefore we can write it as:\\
    $\frac{5}{3} \cdot \frac{9-8x-1}{25-24x-1}\\
    = \frac{5}{3} \cdot \frac{8\cancel{(x-1)}}{24\cancel{(x-1)}}\\
    = \frac{5}{3}*\frac{1}{3}=\frac{5}{9}$\\
    Now let's do it using the chad method.\\
    We want to use approximations, hence we need the limit to be tending to 0. We'll substitute $x=1+t$ and $t \to 0$,\\
    $\lim_{t \to 0} \frac{3-\sqrt{9+8t}}{5-\sqrt{25+24t}}$\\ While doesn't seem anything too special, remember the binomial approximation from chapter-9? Assume the limit to be present from here onward\\
    $=\frac{3-3(1+\frac{8}{9}t)^{\frac{1}{2}}}{5-5(1+\frac{24}{25}t)^{\frac{1}{2}}}\\
    = \frac{3-3(1+\frac{4t}{9}}{5-5(1+\frac{12t}{25}}\\
    =\frac{3(\frac{4t}{9})}{5(\frac{12t}{25})}\\
    =\frac{3}{5}\cdot \frac{4}{9} \cdot \frac{25}{12}\\
    =\frac{5}{9}$
\end{proof}
You may have felt that the chad method wasn't really that good...
\begin{example}
    \[\lim_{x \to 1} \frac{x}{\sqrt{1+x}-\sqrt{1-x}}\]
\end{example}
\begin{proof}
    [Solution]
    Using binomial approximation, \\
    $\frac{x}{\sqrt{1+x}-\sqrt{1-x}}
    = \frac{x}{\cancel{1}+\frac{x}{2}-\cancel{1}+\frac{1}{x}}\\
    =\frac{2\cancel{x}}{\cancel{x}}\\
    =2$
\end{proof}
I leave trying out the rigorous method to you. Let's now try a problem where the rigorous method will become so complex that I don't think it will be worth the time or effort.\\
\begin{example}
    \[\lim_{x \to 2} \frac{3-\sqrt[3]{x^2+5x+13}}{4-\sqrt{x^2+3x+6}}\]
\end{example}
\begin{proof}
    Let $x=2+t$ and $t \to 0$,\\
    $\frac{3-\sqrt[3]{x^2+5x+13}}{4-\sqrt{x^2+3x+6}}\\
    = \frac{3-\sqrt[3]{(t+2)^2+5(t+2)+13}}{4-\sqrt{(t+2)^2+3(t+2)+6}}\\
    = \frac{3-\sqrt[3]{(t^2+4t+4+5t+10+13}}{4-\sqrt{(t^2+4t+4+3t+6+6}}\\
    = \frac{3-\sqrt[3]{(t^2+9t+27}}{4-\sqrt{(t^2+7t+16}}$\\
    Remember the entire mice and flea discussion? Can we neglect the fleas($t^2$) as after all $t \to 0$?\\
    $=\frac{3-\sqrt[3]{(9t+27}}{4-\sqrt{(7t+16}}\\
    = \frac{3-3(1+\frac{t}{3})^{\frac{1}{3}}}{4-4(1+\frac{7t}{16})^{\frac{1}{2}}}\\
    = \frac{3(1-1-\frac{t}{9})}{4(1-1-\frac{7t}{32}}\\
    =\frac{3}{4} \cdot \frac{t}{9} \cdot \frac{32}{7t}\\
    =\frac{8}{21}$
\end{proof}
We need to make a small note here. We cannot just neglect powers which don't exist. For example in $\lim_{x \to 0}\frac{(x^2+x+1)-(x+1)}{x}$ we cannot neglect $x^2$ in comparison to $x$ as the $x$ itself is getting canceled out. Remember, size is relative. We need someone in comparison to actually make a neglection.\\
Now let's talk about some standard limits and why people who try to memorize them are ignorant fools.\\
\begin{theorem}
[Standard results]
    \begin{enumerate}
    \item $\lim_{x \to 0}\frac{\sin(x)}{x}=\lim_{x \to 0}\frac{\tan(x)}{x}=1$
    \item $\lim_{x \to 0}\frac{\arcsin(x)}{x}=\lim_{x \to 0}\frac{\arctan(x)}{x}=1$
    \item $\lim_{x \to \infty}(1+\frac{1}{x})^x=\lim_{x \to 0}(1+x)^{\frac{1}{x}}=e$
    \item $\lim_{x \to \infty}(1+\frac{a}{x})^x=\lim_{x \to 0}(1+ax)^{\frac{1}{x}}=e^a$
    \item $\lim_{x \to 0} \frac{e^x-1}{x}=1$
    \item $\lim_{x \to 0} \frac{a^x-1}{x}=\ln(a)$ for $a>0$
    \item $\lim_{x \to 0} \frac{\ln(1+x)}{x}=1$
    \item $\lim_{x \to \alpha}\frac{x^n-\alpha^n}{x-\alpha}=n\alpha^{n-1}$
\end{enumerate}
\end{theorem}
\begin{proof}
    The first and second were explained earlier using unit circle. If someone want's they can use the Taylor series but that's overkill.\\
    The third one is a specific case($a=1$) of the fourth which we'll prove. Assuming limit to be wherever necessary,\\
    Let $\lim_{x \to 0}(1+ax)^{\frac{1}{x}}=y\\
    \iff \frac{1}{x} \ln(1+ax)=\ln(y)\\
    \iff \frac{ax}{x}= \ln(y)$ This is using the Brahmastra.\\
    $\iff \ln(y)=a\\
    \iff y=e^a$\\
    The infinity form is just obtained by taking $x=\frac{1}{n}$.\\
    The fifth is a mild embarrassment using Brahmastra, $\frac{e^x-1}{x}=\frac{x+1-1}{x}=\frac{x}{x}=1$\\
    The sixth limit will use the same methodology. $\frac{a^x-1}{x}=\frac{e^{x\ln(a)}-1}{x}=\frac{x\ln{a}+1-1}{x}=\ln(a)$\\
    The seventh limit just shouts Brahmastra,\\
    $\frac{\ln(1+x)}{x}=\frac{x}{x}=1$\\
    The eight limit will however have to wait a minute.
\end{proof}
Taking the logarithm of the limit(as we did for the forth one) is quite a common technique to deal with limits with variable exponents and I refer to it as Vayuastra, as a reference to the Indian mythological weapon which gives the user complete control of the sky and air.
And third and final weapon for attacking limits is L'Hopital's rule, which I call Agniastra as a refrence to, you guessed it, Indian mythology. As Agniastra gives the user immense fire power, so does L'Hopital.\\
  \begin{theorem}
      [L'Hopital's Rule]
      \[\lim_{x \to x_0} \frac{f(x)}{g(x)}= \lim_{x \to x_0} \frac{f'(x)}{g'(x)}\]
      If and only if, $f(x)$ and $g(x)$ are differentiable on all points other than $x_0$ with $g'(x_0)\neq0$ and $\frac{f(x)}{g(x)}$ is indeterminate which means $f(x)$ or $g(x)$ are both either $0$ or $\infty$ 
  \end{theorem}
  \begin{proof}
      We know that $\Delta f=hf'(x) \iff f(x+h)-f(x)=hf'(x) \iff f(x+h)=f(x)+hf'(x)$\\
      Now we'll compute the limit for $x \to x_0$ where $f(x_0)=g(x_0)=0$\\
      $\lim_{x \to x_0} \frac{f(x)}{g(x)}\\
      = \lim_{x \to x_0, h \to 0} \frac{f(x+h)}{g(x+h)}\\
      = \lim_{x \to x_0, h \to 0} \frac{hf'(x)+f(x_0)}{hg'(x)+g(x_0)}\\
      = \lim_{x \to x_0, h \to 0} \frac{\cancel{h}f'(x)}{\cancel{h}g'(x)}\\
      =\lim_{x \to x_0} \frac{f'(x)}{g'(x)}$\\
      Hence, proved.
  \end{proof}
This makes proving the eighth result a piece of cake.\\
$\frac{x^n-\alpha^n}{x-\alpha}=\frac{nx^{n-1}}{1}=nx^{n-1}$\\
All these standard results are hece quite easy to get and keeping them in the brain wastes the space which may be used elsewhere. Let's do some examples.\\
\begin{example}
    \[\lim_{x \to 0}\frac{3^x-1}{2^x-1}\]
\end{example}
\begin{proof}
[Solution]
    While this question is destroyed in shreds by Brahmastra, I'll use the Agniastra for instructive purposes.\\
    $(a^x)'=(e^{x\ln(a)})'$ we can use the chain rule here, to get:\\
    $=\ln(a)e^{x\ln(a)}\\
    = \ln(a)a^x$\\
    Using this in our limit gives:\\
    $\lim_{x \to 0}\frac{3^x-1}{2^x-1}\\
    = \lim_{x \to 0}\frac{\ln(3)3^x}{\ln(2)2^x}$ as the function is now continuous, we can just plug in $x=0$\\
    $=\frac{\ln(3)}{\ln(2)}
    =\log_2(3)$
\end{proof}
\begin{example}
    \[\lim_{x \to 0}\frac{1-\cos{3x}}{x^2}\]
\end{example}
\begin{proof}
    [Solution]
    For the first time, we'll have to use the Brahmastra upto two places.\\
    $\frac{1-\cos{3x}}{x^2}\\
    =\frac{1-(1-\frac{(3x)^2}{2!}}{x^2}\\
    =\frac{9\cancel{x^2}}{2\cancel{x^2}}\\
    =\frac{9}{2}$
\end{proof}
\section{Continuity}
As we have already defined, continuity refers to the fact a graph(or a region of it) can be drawn without lifting the pencil. However, more formally:\\
\begin{definition}
    A function $f(x)$ is continuous at $x=a$ if and only if:\\
    $f(a^-)=f(a^+)=f(a)$
\end{definition}
Basically the right hand function must agree with the left hand function as well as the instantaneous value. Unlike with limits, here $f(a)$ actually matters.\\
\begin{example}
    Let $f(x)=\begin{cases}
        (1+x)^{\frac{1}{x}} & \text{for } x > 0 \\
    a & \text{for } x = 0 \\
    \frac{1-\cos(x)}{bx^2} & \text{for } x < 0
    \end{cases}$\\
    If $f(x)$ is continuous at $x=0$, find $\lfloor a+b \rfloor$
\end{example}
\begin{proof}
    [Solution]
    Using the defination of continuity, we can say:\\
    $(1+x)^{\frac{1}{x}}=a=\frac{1-\cos(x)}{bx^2}$ for $x \to 0$\\
    You might recall that the limit of $(1+x)^{\frac{1}{x}}=e$ for $x \to 0$\\
    That means $a=e$. Let's now find $b$.\\
    $\lim_{x \to 0}\frac{1-\cos(x)}{bx^2}=e\\
    \iff \frac{1}{b} \lim_{x \to 0}\frac{1-\cos(x)}{x^2}=e\\
    \iff b=\frac{1}{e} \lim_{x \to 0}\frac{1-\cos(x)}{x^2}\\
    \iff b=\frac{1}{e} \lim_{x \to 0}\frac{1-(1-\frac{x^2}{2})}{x^2}\\
    \iff b=\frac{1}{2e} \text{Using Brahmastra, obviously}$
    Therefore, $\lfloor a+b \rfloor=\lfloor e+\frac{1}{2e} \rfloor= \lfloor 2.90\dots \rfloor= 2$
\end{proof}
\section{Application of Differentiation}
We are finally nearing the end of our first foray into calculus. We are now going to return from where we had started, real life.\\
\begin{example}
    A 15m ladder is propped up against a window of 12m. It starts moving away from the wall at 1m/s. At what speed is it falling?
\end{example}
\begin{proof}
    [Solution]
    A classic physics problems which can be done using two more physics approachs. However, we'll study the mathematical approach here.\\
    Let the height be $x=12$ and the run(distence from the base of wall) be $y$, we can then say:\\
    $x^2+y^2=15^2 \iff y=9$
    We want to know the rate of change of $x$ in terms of time. So we can multiply by $\frac{d}{dt}$ to get:\\
    $\frac{d(x^2)}{dt}+\frac{d(y^2)}{dt}=\frac{d(15^2)}{dt}\\
    \iff 2x\frac{dx}{dt}+2y\frac{dy}{dt}=0\\
    \iff 2*12\frac{dx}{dt}+2*9*1=0\\
    \iff \frac{dx}{dt}=\frac{-3}{4}$\\
    Which reprasents that the ladder is moving downward at the speed of $\frac{3}{4}$m/s.
\end{proof}
Another use of derivatives is in optimization. Remember, I had promised that we'll talk about the turnings of the graph? The most optimized point for a function where it reaches it's maxima or minima will be the cusp of a turning.
\begin{figure} [h]
    \centering
    \includegraphics[width=0.75\linewidth]{Photos/Optimization zero slope.png}
    \caption{The turnings have become significant}
    
\end{figure}
We can notice that the slope of the tangent is $0$ at the points of maxima and minima. Hence, the extreme points are the roots of $f'(x)$.\\
But its not as simple, the converse is not always true, zero slope may just be a point of inflection(a point higher or lower than its neighbours but not the maxima or minima)\\
We need to be able to tell weather tell if a point is a maxima or a minima, so here is what we need to notice. The slope starts as positive for a maxima and then becomes negative. The opposite is true for minima. This means that we can take the rate of change of the slope and if it is less than $0$, the point is a maxima. If it is more than $0$, the point is a minima.\\
This is the second derivative test. But what if we don't wish to take two derivatives?\\
The we can notice that the sign of the slope changes at the points. So we can notice the value of $f'(x)$ between two roots to guess if its a minima or maxima.\\
We don't need to do this for every interval as the sign just alternates.\\
You'll understand this better through a problem:\\
\begin{example}
    An open topped box is to be constructed by removing equal squares from
each corner of a 3 metre by 8 metre rectangular sheet of aluminium and folding up the sides. Find the volume of the largest such box.
\end{example}
\begin{proof}
    [Solution]
    Let the side of the square removed be $x$. We can notice that the volume would be $f(x)=x(3-2x)(8-2x)=4x^3-22x^2+24x$  meter cubed.\\
    We can differentiate it to get the equation: $f'(x)=12x^2-44x+24$\\
    Setting this as zero has two solutions, $x=\frac{2}{3}, 3$\\
    While we know the answer is $\frac{2}{3}$ as one side of the shape is $3$. Let's confirm it is the maxima using both the methods.\\
    Single Derivative method: We can notice that $f'(0)=24>0$ which means that $f'(x)$ is positive till $x=\frac{2}{3}$, then negative till $x=2$, and then positive forever. \\
    We can use this to say $x=\frac{2}{3}$ is the point of maxima and $x=3$ the point of minima(realistically it is $x=\frac{3}{2}$)
Double derivative method: We can compute that $f''(x)=24x-44$ which is less than zero for $x=\frac{2}{3}$ and greater than zero for $x=3$ making them the maxima and minima repectivly.\\
With the tests out of the way, we can now find the largest possible volume by just plugging in $x=\frac{2}{3}$ to get that the maximum volume would be: $\frac{200}{3}$ meter cubed.
\end{proof}
And finally, we talk about video games. Most video games need to make trigonometric calculations in an instant especially online FPS and fighting games. While we could have it compute the actual values, that would be mind numbingly slow and make the gameplay worse. For the quick tactile feel we need, we use approximations. However, these approximations also need to be accurate or else the game will feel unrealistic.\\
How do we achieve this? Remember the mouse-flea discussion and its form we had created during the proof of L'-Hopital?\\
$f(x+h)=f(x)+hf'(x)$ may seem harmless enough but if we choose a value of $f(x)$ we already know and then move recursively ahead we can get increasingly accurate answers.\\
\begin{example}
    Calculate $\sqrt{73}$
\end{example}
\begin{proof}
    [Solution]
    Let's take $f(x)=\sqrt{x}$ and $x=8^2=64$,\\
    $\therefore h=73-64=9$ and $f'(x)=\frac{1}{2\sqrt{x}}$\\
Then we can say that, \\
$\sqrt{73}=f(73)=f(64+9)=f(64)+9f'(64)=8+\frac{9}{16}\\
= 8.5625$\\
The actual value of $\sqrt{73}=8.54400$ which is within $1\%$ of what we found using the formula only once.\\
A computer would now compute $8.5625^2$ and use that for an even more refined approximation. Most games do it 2-3 times, while even some calculators also use this but do it 5-10 times.\\
\end{proof}
And with that this chapter is concluded. 
\begin{xcb}{Exercises}
\begin{enumerate}
\item What is the co domain of $f(X)=\arctan(x)+\frac{1}{2}\arcsin(x)$?\\
\item Evaluate \[\lim_{x \to \frac{-1}{3}} \frac{1}{x} [\frac{-1}{x}]\] where $[x]$ represents the Greatest integer less than or equal to $x$
\item Evaluate: \[\lim_{n \to \infty} \frac{5^{n+1}+3^n-2^{2n}}{5^{n}+3^{2n+3}+2^{n}}\]
\item The value of the given limit is: \[\lim_{x \to 0} \frac{\cos(\sin(x))-\cos(x)}{x^4}\]
\item If \[\lim_{n \to \infty} \frac{1^2n+2^2(n-1)+3^2(n-2)+\dots+n^21}{1^3+2^3+^3+\dots +n^3} = \frac{a}{b}\] then find the value of $a^3+b^3$
\item \[\lim_{x \to 0} (\sum^n_{r=1} r^{\csc^2(x)})^{\sin^2(x)}\]
\item \[\lim_{n \to \infty} [(1+\frac{1}{n})^n-(1-\frac{1}{n})]^{-n}\]
\item  If $a_1$ is the greatest value of $f(x)$, where $f(x)=\frac{1}{2+[\sin(x)]}$ and $a_{n+1}=a_n+\frac{(-1)^{n+2}}{n+1}$, then $\lim_{n \to \infty} a_n =?$(here $[x]$ is the greatest integer function)\\
\item Calculate \[\lim_{x \to 0} \{[\frac{100x}{\sin(x)}] + [\frac{99 \sin(x)}{x}]\}\]
\item If $a_n$ and $b_n$ are positive integers and $a_n+\sqrt{2}b_n=(2+\sqrt{2})^n$ then calculate $\lim_{n \to \infty} (\frac{a_n}{b_n})=?$
\item If $f(x)=\begin{cases}
    k\sqrt{x+1}, \text{if} x\leq 3\\
    2+mx, \text{if} x> 3\\
\end{cases}$ is diffrentiatable at $x=3$, then find the vale if $m$ and $k$ is:
\item (JEE Mains 2018) Let $S = {t \in R:f(x)=|x -\pi|\cdot(e^{|x|} - 1)\sin(x) \text{is not differentiable at } t}$ then the set $S$ is equal to:
\item(JEE Mains 2021) The number of points at which the function $f(x)=|2x+1|-3|x+2|+|x^2+x-2|$ with $x \in \mathbb{R}$ is not differentiable, is: 
\item Let $f(x)=e^{x-1}-ax^2+b$ and $g(x)=\begin{cases}
    e^{x-1} \text{if} x \leq 1\\
    x^2+1 \text{if} x > 1\\
\end{cases}$ then find the value of $a$ and $b$ such that $f(x) \times g(x)$ is differentiable at $x=1$
\item (JEE Mains 2020) For a function $f$ defined on $(\frac{-1}{3},\frac{1}{3})$ by $f(x)= \begin{cases}
    \frac{1}{x} \ln{\frac{1+3x}{1-2x}} \text{ when } x \neq 0\\
    k, \text{ when } x=0
\end{cases}$ is continous, then $k$ is equal to\\
\item (AIEEE 2011) The value of $p$ and $q$ for which the function:\\
\[f(x)=\begin{cases}
    \frac{\sin(p+1)x+\sin(x)}{x}, x<0\\
    q, x=0\\
    \frac{\sqrt{x^2+x}-\sqrt{x}}{x^{\frac{3}{2}}}, x>0
\end{cases}\]
is continuous for all $x \in \mathbb{R}$ are
\item (JEE Mains 2020) If $y(a)=\sqrt{2(\frac{\tan(x)+\cot(x)}{1+\sin^2(a)}+\frac{1}{\sin^2(a)}}$ where $a \in (\frac{3\pi}{4}, \pi)$, then $\frac{dy}{dx}$ at $y=\frac{5\pi}{6}$ is:\\
\item(IIT 2006) If $f''(x)=-f(x)$ and $g(x)=f'(x)$ and $F(x)=(f(\frac{x}{2}))^2+(g(\frac{x}{2}))^2$ and given that $F(5)=5$, then $F(10)$ is
\item (IIT Adv 2014) Let $f : \mathbb{R} \to  \mathbb{R}$ and $g : \mathbb{R} \to  \mathbb{R}$ be respectively given by $f(x) = |x| + 1$ and $g(x) = x^2 + 1$. Defined $h:  \mathbb{R} \to  \mathbb{R}$ by $h(x)=\begin{cases}
    \max \{f(x),g(x)\}, x \leq 0\\
    \min \{f(x),g(x)\}, x > 0\\
\end{cases}$ The number of points at which $h(x)$ is not differentiable is:
\end{enumerate}
\end{xcb}
\input{4redpill/calc2.tex}
\chapter{Ch-14 Linear Algebra}
Let me address the fundamental question on the face of it: what is Linear Algebra, and why should we bother learning it? You see, mathematics is not just a set of abstract rules and theorems; it's a language that allows us to describe and understand the world around us. Linear Algebra is a crucial part of this mathematical language, and it serves as a bridge between various mathematical concepts and the real-world problems we aim to solve.\\
Now, linear algebra may seem like a bit of a departure from your previous mathematical endeavors, and I want to assure you that it's quite different, yet fundamentally tied to the mathematics you've encountered before. You see, on the face of it linear algebra serves as a bridge between algebra and geometry, and at a greater depth it achives a lot more.\\
At its core, Linear Algebra deals with matrices, which are essentially a different way of representing data than the vectors we'll explore later on. Think of a matrix as a structured arrangement of numbers, much like an Excel spreadsheet. You might remember incidence matrices from Power Overwhelming chapter. These numbers hold hidden patterns and relationships that we'll uncover throughout this chapter.\\
Now, I must offer a small apology in advance. This chapter, and indeed much of Linear Algebra, may seem abstract at first. We'll delve into concepts that may not seem immediately applicable to the real world. But rest assured, as we progress, we'll unlock the practical applications of these abstract notions. \\
Think of this abstract treatment as laying the groundwork, building a solid foundation for the magnificent mathematical structures we'll explore later. Much like constructing a sturdy building, we need a strong foundation to support the practical applications of Linear Algebra that we'll discuss in due time.\\
I want to also extend my apologies once again, before we embark on this journey through Linear Algebra. While our goal is to explore the beauty and utility of matrices, determinants, and their applications, I must acknowledge that I have not provided the rigorous proofs for every statement and concept we encounter.\\
You see, Linear Algebra can be quite intricate, and proving every assertion can often be a lengthy and intricate process, especially when we're dealing with general matrices. Consider, for instance, taking a random matrix, multiplying it by another, and then attempting to factorize the result. This task, though fundamental and basic algebra, can indeed become quite tedious. I however, encourage you to do so. You won't need any more knowledge than what we covered in basic algebric manipulations\\
With that out of the way, I must remind you that Linear algebra is not just about abstract problems and equations. Linear Algebra is the muscle behind modern technology. It powers your smartphone apps and games, helps Netflix recommend your next binge-worthy series, and even guides spacecraft to explore the cosmos. Heck, it's even there when you make a simple Google search. So, if you ever wondered how your world works, this is a fantastic place to start.\\
As for the how and why of these real-world applications, they'll come, I promise! You see, linear algebra is like a Swiss Army knife for scientists, engineers, and data scientists. We can use it to solve systems of equations, analyze networks, optimize processes, and even make sense of vast amounts of data in fields like machine learning and statistics. So, while we may start in the abstract, by the end of this journey, you'll understand why linear algebra is so essential in the modern world.\\
\section{Determinants}
Let's for now assume matrix to be a row and column of numbers. We'll talk more about them later.\\
The determinant is only defined for a matrix with equal columns and rows, or a square matrix. The number of rows or columns is called the order. We'll define it as follows:\\
\begin{definition}
    Determinant of matrix of order $1$:\\
    $|a| = a$
\end{definition}
\begin{definition}
    Determinant of matrix of order $2$:\\
    $\begin{vmatrix}
        a_1 & b_1\\
        a_2 & b_2
    \end{vmatrix} = a_1\cdot b_2 - b_1\cdot a_2$
\end{definition}
We can remember this definition by looking at it as cross multiplying on then diagonals and then  subtracting.\\
Here are a few examples of taking the determinant:\\
\begin{example}
[Motivating Example]
    $\begin{vmatrix}
        4 & 3\\
        1 & -3
    \end{vmatrix}$
\end{example}
\begin{proof}
    [Solution]
    This quite simple. \\
    $4*(-3)-3*1\\
    = -12 -3\\
    = -15$
\end{proof}
\begin{example}
    If $\begin{vmatrix}
        e^x & \sin(x)\\
        \cos(x) & \ln{1+x}
    \end{vmatrix} = A + Bx + Cx^2+\dots$  What is the value of  $A$?
\end{example}
\begin{proof}
    [Solution]
    Before we pull out the expansions from calculus, notice that taking $x=0$ will make the determinant $A$ and that is what is asked from us.\\
    $\therefore A= \begin{vmatrix}
        e^0 & \sin(0)\\
        \cos(0) & \ln{1+0}
    \end{vmatrix}\\
    = \begin{vmatrix}
        1 & 0\\
        1 & 0
    \end{vmatrix}\\
    = 0$
\end{proof}
Now we enter the actually confusing area.  For any general matrix we will here onward write the term at the intersection of the $i^{th}$ column and $j^{th}$ row as $a_{ij}$. Using this notation we define the determinant of order 3 as:\\
\begin{definition}
    Determinant of order 3
    $\begin{vmatrix}
    a_{11} & a_{12} & a_{13} \\
    a_{21} & a_{22} & a_{23} \\
    a_{31} & a_{32} & a_{33} \\
\end{vmatrix} = a_{11} \begin{vmatrix}
    a_{22} & a_{23}\\
    a_{32} & a_{33}\\
\end{vmatrix} - a_{12} \begin{vmatrix}
    a_{21} & a_{23}\\
    a_{31} & a_{33}\\
\end{vmatrix} + a_{13} \begin{vmatrix}
    a_{21} & a_{22}\\
    a_{31} & a_{32}\\
\end{vmatrix}\\
= a_{11}(a_{22}\cdot a_{33} - a_{23}\cdot a_{32})-a_{12}(a_{21}\cdot a_{33} - a_{23}\cdot a_{31})+ a_{13}(a_{21}\cdot a_{32} - a_{22}\cdot a_{31})$ 
\end{definition}
We can remember this by basically removing the row and column of a number, taking the determinant of the rest of the matrix(now of order 2) and then multiply them. Then we take the next term in the column or row, repeat the process but alternating the sign. We can do this for any row or column to get the determinant(the definition is on the first row but we can choose the one which leads to the shortest calculations).\\
Let's try an example:\\
\begin{example}
[Motivating example]
    $\begin{vmatrix}
    3 & 3 & 2 \\
    5 & 4 & 7 \\
    5 & 7 & 6 \\
\end{vmatrix}$
\end{example}
\begin{proof}
    [Solution]
    Using the first row seems promising as it has relatively small numbers.\\
    $\begin{vmatrix}
    3 & 3 & 2 \\
    5 & 4 & 7 \\
    5 & 7 & 6 \\
\end{vmatrix} = 3(4*6-7*7)-3(5*6-7*5)+2*(5*7-4*5)\\
= 3*(-25)-3(-5)+2(15)\\
= -75+15+30\\
= 30$
\end{proof}
Now we'll define two new terms called minor and cofactor.\\
\begin{definition}
    The minor $M_{ij}$ of an element $a_{ij}$ in a matrix $\begin{bmatrix}
    a_{11} & a_{12} & a_{13} \\
    a_{21} & a_{22} & a_{23} \\
    a_{31} & a_{32} & a_{33} \\
\end{bmatrix}$ is the determinant obtained by deleting the $i^{th}$ row and $j^{th}$ column.\\
\end{definition}
\begin{definition}
    The co-factor $C_{ij}$ of an element $a_{ij}$ is given by:\\
    $C_{ij}=(-1)^{i+j}M_{ij}$
\end{definition}
Let's do an example to get comfortable with the definition.\\
\begin{example}
    [Motivating example]
    Find the co-factors of all the elements of : $\begin{bmatrix}
        1 &-5 & -1\\
        5 &0 &3\\
        -3 &7 &9
    \end{bmatrix}$ 
\end{example}
\begin{proof}
    [Solution]
    For $a_{11}=1$, the $C_{11}=(-1)^{1+1}M_{12}\\
    = (-1)^{2}\begin{vmatrix}
        0 & 3\\
        7 & 9\\
    \end{vmatrix}\\
    = -21$\\
    For $a_{12}=1$, the $C_{12}=(-1)^{1+2}M_{12}\\
    = (-1)^{3}\begin{vmatrix}
        5 & 3\\
        -3 & 9\\
    \end{vmatrix}\\
    = -54$\\
    For $a_{13}=1$, the $C_{13}=(-1)^{1+3}M_{13}\\
    = (-1)^{4}\begin{vmatrix}
        5 & 0\\
        -3 & 7\\
    \end{vmatrix}\\
    = 38$\\
And we can do the same with the rest of the matrix. I think, you will be able to do that by yourself.\\
\end{proof}
What we need to notice is that we will always have some matrix 
\begin{definition}
The adjacent of some matrix $A$
$A= \begin{bmatrix}
    a_{11} & a_{12} & a_{13} \\
    a_{21} & a_{22} & a_{23} \\
    a_{31} & a_{32} & a_{33} \\
\end{bmatrix} $ Is defined as $adj(A) = \begin{bmatrix}
    C_{11} & C_{12} & C_{13} \\
    C_{21} & C_{22} & C_{23} \\
    C_{31} & C_{32} & C_{33} \\
\end{bmatrix}$\\
\end{definition}
This leads us to another definition of the determinant:\\
\begin{definition}
    $\begin{vmatrix}
        a_{11} & a_{12} & a_{13} \\
    a_{21} & a_{22} & a_{23} \\
    a_{31} & a_{32} & a_{33} \\
    \end{vmatrix} = a_{11} \cdot C_{11}+a_{12} \cdot C_{12}+a_{13} \cdot C_{13}$
    Or in words we can say: The sum of product of elements of any row or column of $A$ with the corresponding row or column of $adj(A)$ is equal to the determinant of A $det(A)$. 
\end{definition}
A classic matrix we should remember the determinant to is:\\
\begin{example}
    $\begin{vmatrix}
        1 & z & -y\\
        -z & 1 & x\\
        y & -x & 1
    \end{vmatrix}$
\end{example}
\begin{proof}
    [Solution]
    Expanding along the first row gives us:\\
    $(1+x^2)-z(-z-xy)+(-y)(zx-y)\\
    = 1+x^2+z^2+xyz-xyz+y^2\\
    =1+x^2+y^2+z^2$
\end{proof}
Another thing which is quite interesting, although useless, is:\\
\begin{theorem}
    The sum of product of any row or column of matrix $A$ with the corresponding any other row or column(other than the corresponding) of $adj(A)$ is zero.
\end{theorem}
\section{Properties of Determinants}
\begin{theorem}
Value of determinant remains unchanged on interchanging rows and columns.\\
    $\begin{vmatrix}
         a_{11} & a_{12} & a_{13} \\
    a_{21} & a_{22} & a_{23} \\
    a_{31} & a_{32} & a_{33} \\
    \end{vmatrix} = \begin{vmatrix}
         a_{11} & a_{21} & a_{31} \\
    a_{12} & a_{22} & a_{32} \\
    a_{13} & a_{23} & a_{33} \\
    \end{vmatrix}$
    The flipping of rows and columns in called transposing.\\
\end{theorem}
\begin{theorem}
If any two rows (columns) of a determinant are interchanged, the value of determinant changes sign.
$\begin{vmatrix}
    a_{11} & a_{12} & a_{13} \\
    a_{21} & a_{22} & a_{23} \\
    a_{31} & a_{32} & a_{33} \\
\end{vmatrix} = - \begin{vmatrix}
    a_{31} & a_{32} & a_{33} \\
    a_{21} & a_{22} & a_{23} \\
    a_{11} & a_{12} & a_{13} \\
\end{vmatrix}$
\end{theorem}
\begin{theorem}
If any two rows (columns) are identical then value
of determinant is zero 
\end{theorem}
We can use this properties to solve problems which are otherwise harder to solve:\\
\begin{example}
    Find $x$ if\\
    $\begin{vmatrix}
        1  & 1 & 0\\
        (x^2-5x+7) & 1 & 0\\
        2 & 1 & 1\\
    \end{vmatrix} = 0$
\end{example}
\begin{proof}
    [Solution]
    This is quite simple as we can notice that two terms of the first and second row are equal. Hence if the third are also equal, the determinant will become 0.\\
    $\therefore x^2-5x+7=1\\
    \therefore x^2-5x+6=0\\
    \therefore x=2,3$
\end{proof}
Here is another theorem, this one is actually quite use full, in both questions and in real world.\\
\begin{theorem}
     If all the elements of any row or column be multiplied by a number $K$ then value of determinant is multiplied by $K$\\
     $K \times \begin{vmatrix}
    a_{11} & a_{12} & a_{13} \\
    a_{21} & a_{22} & a_{23} \\
    a_{31} & a_{32} & a_{33} \\
\end{vmatrix} = \begin{vmatrix}
    Ka_{11} & Ka_{12} & Ka_{13} \\
    a_{21} & a_{22} & a_{23} \\
    a_{31} & a_{32} & a_{33} \\
\end{vmatrix}= \begin{vmatrix}
    Ka_{11} & a_{12} & a_{13} \\
    Ka_{21} & a_{22} & a_{23} \\
    Ka_{31} & a_{32} & a_{33} \\
\end{vmatrix}$
\end{theorem}
Let's annihilate a JEE advance question to prove the power of this:\\
\begin{example}
     If $b_{ij} = 2^{i+j}a_{ij}$ where $a_{ij}$ and $b_{ij}$ are elements of $3 \times 3$ determinants $\Delta_1$ and $\Delta_2$ respectively, the find $\Delta_2$ if $\Delta_1= 2$
\end{example}
\begin{proof}
    [Solution]
    Let the determinant $\Delta_1$ be of the matrix:\\
    $\begin{vmatrix}
    a_{11} & a_{12} & a_{13} \\
    a_{21} & a_{22} & a_{23} \\
    a_{31} & a_{32} & a_{33} \\
\end{vmatrix}$, therefore\\
$\Delta_2= \begin{vmatrix}
    2^2 a_{11} & 2^3 a_{12} & 2^4 a_{13} \\
    2^3 a_{21} & 2^4 a_{22} 2^5 & a_{23} \\
    2^4 a_{31} & 2^5 a_{32} & 2^6 a_{33} \\
\end{vmatrix} \\
= 2 \begin{vmatrix}
    2^2 a_{11} & 2^3 a_{12} & 2^4 a_{13} \\
    2^2 a_{21} & 2^3 a_{22} 2^4 & a_{23} \\
    2^4 a_{31} & 2^5 a_{32} & 2^6 a_{33} \\
\end{vmatrix}\\
= 2*2^2 \begin{vmatrix}
    2^2 a_{11} & 2^3 a_{12} & 2^4 a_{13} \\
    2^2 a_{21} & 2^3 a_{22} 2^4 & a_{23} \\
    2^2 a_{31} & 2^3 a_{32} & 2^4 a_{33} \\
\end{vmatrix} \\
= 2^3 * 2^2 *2^3 *2^4 \begin{vmatrix}
    a_{11} & a_{12} & a_{13} \\
    a_{21} & a_{22} & a_{23} \\
    a_{31} & a_{32} & a_{33} \\
\end{vmatrix}\\
= 2^12 \cdot \Delta_1\\
=2^13$
\end{proof}
\begin{theorem}
If each element of any row (or column) is expressed as sum of two (or more) terms then the determinant can be expressed as the sum of two (or more) determinants.\\
$\begin{vmatrix}
    a_{11}+x & a_{12}+y & a_{13}+z \\
    a_{21} & a_{22} & a_{23} \\
    a_{31} & a_{32} & a_{33} \\
\end{vmatrix} = \begin{vmatrix}
    a_{11} & a_{12} & a_{13} \\
    a_{21} & a_{22} & a_{23} \\
    a_{31} & a_{32} & a_{33} \\
\end{vmatrix} + \begin{vmatrix}
    x & y & z \\
    a_{21} & a_{22} & a_{23} \\
    a_{31} & a_{32} & a_{33} \\
\end{vmatrix}$
\end{theorem}
This theorem is more commonly used to combine matrices than to split them. Here is an example which will also brush up your sequence and series:\\
\begin{example}
    If $\Delta_r = \begin{vmatrix}
    r+x & n(n+1) & n^2+n^3 \\
    2^r & 4(2^n-1) & n^2+n+1 \\
    3^r & 3(3^n-1) & 2n+1 \\
\end{vmatrix}$ then $\sum^n_{r=1}\Delta_r=?$
\end{example}
\begin{proof}
    [Solution]
    We can notice that the second column and the third column are same in all the matrices. Therefore we can simply combine over the first column.\\
    As $\sum^n_{r=1}r=\frac{r(r+1)}{2}\\
    \sum^n_{r=1}2^r=\frac{2(2^n-1)}{2-1}\\
    \sum^n_{r=1}3^r=\frac{3(3^n+1)}{3-1}$\\
    Therefore, $\sum^n_{r=1}\Delta_r= \begin{vmatrix}
    \frac{r(r+1)}{2} & n(n+1) & n^2+n^3 \\
    \frac{2(2^n-1)}{2-1} & 4(2^n-1) & n^2+n+1 \\
    \frac{3(3^n+1)}{3-1} & 3(3^n-1) & 2n+1 \\
\end{vmatrix}\\
= \frac{1}{2}\begin{vmatrix}
    r(r+1) & n(n+1) & n^2+n^3 \\
    4(2^n-1) & 4(2^n-1) & n^2+n+1 \\
    3(3^n+1) & 3(3^n-1) & 2n+1 \\
\end{vmatrix}\\
= 0$\\
As column 1 is equal to column 2.\\
\end{proof}
\begin{theorem}
    The value of determinants is not altered by adding or subtracting the multiple of any row or column in other row or column. \\
    $\begin{vmatrix}
    a_{11} & a_{12} & a_{13} \\
    a_{21} & a_{22} & a_{23} \\
    a_{31} & a_{32} & a_{33} \\
\end{vmatrix} = \begin{vmatrix}
    a_{11} + Ka_{31} & a_{12} & a_{13} \\
    a_{21} + Ka_{32} & a_{22} & a_{23} \\
    a_{31}+Ka_{33} & a_{32} & a_{33} \\
\end{vmatrix}$
\end{theorem}
What makes this theorem even more powerful is the fact we can do 2  operations at at a time.\\
What I mean to say is that: $\begin{vmatrix}
    a_{11} & a_{12} & a_{13} \\
    a_{21} & a_{22} & a_{23} \\
    a_{31} & a_{32} & a_{33} \\
\end{vmatrix} = \begin{vmatrix}
    a_{11}-a_{12} & a_{12}-a_{13} & a_{13} \\
    a_{21}-a_{22} & a_{22}-a_{23} & a_{23} \\
    a_{31}-a_{32} & a_{32}-a_{33} & a_{33} \\
\end{vmatrix}$
This allows us to solve a lot of problems at extreme speeds as long as we can find the correct manipulation.\\
For example: $\begin{vmatrix}
    1 & a & b+c \\
    1 & b & c+a \\
    1 & c & a+b \\
\end{vmatrix}$ can be calculated in an instant by simply summing up the second and third column and then taking $a+b+c$ as common.\\
Let's do something a tad more difficult:\\
\begin{example}
    Evaluate: $\begin{vmatrix}
    1 & 1 & 1 \\
    a & b & c \\
    a^2 & b^2 & c^2 \\
\end{vmatrix}$
\end{example}
\begin{proof}
    [Solution]
    While expanding the determinant along the first row is quite trivial, let's use properties to make a joke of this.\\
    Subtracting the first column from the second and third from the second:\\
    $\begin{vmatrix}
    0 & 0 & 1 \\
    a-b & b-c & c \\
    a^2-b^2 & b^2-c^2 & c^2 \\
\end{vmatrix}$\\
We could now expand with even greater ease, but wait there's more:\\
$(a-b)(b-c)\begin{vmatrix}
    0 & 0 & 1 \\
    1 & 1 & c \\
    a+b & b+c & c^2 \\
\end{vmatrix}$\\
Which simplifies to $(a-b)(b-c)(c-a)$ and we are done.\\
\end{proof}
I recommend remembering this result as a standard form.\\
Here is an very similar example for you to try:\\
\begin{example}
    $\begin{vmatrix}
    1 & a & bc \\
    1 & b & ac \\
    1 & c & ab \\
\end{vmatrix}$
\end{example}
As simple way to hide the three ones looks like:\\
\begin{example}
    $\begin{vmatrix}
    a & b & c \\
    b & c & a \\
    c & a & b \\
\end{vmatrix}$
\end{example}
\begin{proof}
    [Solution]
    We notice that the sum of all three rows is equal.\\
    Therefore, we add along the rows to get:\\
    $\begin{vmatrix}
    a+b+c & b & c \\
    a+b+c & c & a \\
    a+b+c & a & b \\
\end{vmatrix}\\
= (a+b+c) \begin{vmatrix}
    1 & b & c \\
    1 & c & a \\
    1 & a & b \\
\end{vmatrix}$\\
And the three one's  present themselves.
\end{proof}
It's a common approach to try to get the three one's to simplify the determinant.\\
\begin{example}
    $\begin{vmatrix}
    b^2c^2 & bc & b+c \\
    c^2a^2 & ca & c+a \\
    a^2b^2 & ab & a+b \\
\end{vmatrix}$
\end{example}
\begin{proof}
    [Solution]
    We'll multiply the first row by $a$, second row by $b$ and the third row by $c$ to get:\\
    $\frac{1}{abc}\begin{vmatrix}
    ab^2c^2 & abc & a(b+c) \\
    bc^2a^2 & bca & b(c+a) \\
    ca^2b^2 & cab & c(a+b) \\
\end{vmatrix}\\
= \frac{(abc)^2}{abc}\begin{vmatrix}
    bc & 1 & ab+ac \\
    ca & 1 & bc+ab \\
    ab & 1 & ac+bc \\
\end{vmatrix}$\\
This may seem good enough as we already have the three one's but we are gonna utterly humiliate the question by adding column $3$ to column $1$. \\
$abc\begin{vmatrix}
    ab+bc+ca & 1 & ab+ac \\
    ab+bc+ca & 1 & bc+ab \\
    ab+bc+ca & 1 & ac+bc \\
\end{vmatrix}\\
= abc(ab+bc+ca)\begin{vmatrix}
    1 & 1 & ab+ac \\
    1 & 1 & bc+ab \\
    1 & 1 & ac+bc \\
\end{vmatrix}\\
= 0$\\
And we are done.
\end{proof}
\section{Application of Determinants}
We are now done with determinants. Before looking at Matrices, let's talk about some applications of them.\\
\begin{theorem}
    [Shoelace formula]
    Area of a triangle with the vertices at coordinates $(x_1,y_1),(x_2,y_2),(x_3,y_3) =$ magnitude of $\frac{1}{2} \begin{vmatrix}
        x_1 & y_1 & 1\\
        x_2 & y_2 & 1\\
        x_3 & y_3 & 1\\
    \end{vmatrix}$
\end{theorem}
We'll see a more simplified and generalized form of this in coordinate geometry, from which this is derived. But the determinant form has its own use, specifically in the form:\\
\begin{theorem}
    [Condition of Colinearity]
    If $\begin{vmatrix}
        x_1 & y_1 & 1\\
        x_2 & y_2 & 1\\
        x_3 & y_3 & 1\\
    \end{vmatrix}=0$ then the points $(x_1,y_1);(x_2,y_2);(x_3,y_3)$ are colinear.
\end{theorem}
This is true as the area of a line is zero.\\
This tends to occur in questions in a very typical pattern:\\
\begin{example}
    If the points $(at^2_1, 2at_1), (at^2_2, 2at_2), (a, 0)$ are colinear. If $(t_1 \neq t_2)$, then find the minimum value of $t^2_1+9t^2_2$.
\end{example}
\begin{proof}
[Solution]
    Using the condition of Colinearity, $\begin{vmatrix}
        at^2_1 & 2at_1 & 1\\
        at^2_2 & 2at_2 & 1\\
        a & 0 & 1\\
    \end{vmatrix} = 0\\
    \iff \begin{vmatrix}
        at^2_1-a & 2at_1 & 0\\
        at^2_2-a & 2at_2 & 0\\
        a & 0 & 1\\
    \end{vmatrix} = 0\\
    \iff (at^2_1-a)(2at_2)=(2t^2_2-a)(2at_1)\\
    \iff 2a^2t_2t^2_1-2a^2t_2=2a^2t_1t^2_2-2a^2t_1\\
    $$\iff t_2t^2_1-t_2=t_1t^2_2-t_1\\
    \iff t_2-t_1=t_2t_1(t_1-t_2)\\
    \iff t_1t_2=-1$\\
    At this point we can use AM-GM inequality to say that:\\
    $\frac{t_1^2+9t_2^2}{2} \geq \sqrt{9t_1^2t_2^2}\\
    \iff t_1^2+9t_2^2 \geq 2\sqrt{9}\\
    \iff t_1^2+9t_2^2 \geq 6$\\
    Hence, the minimum value of $t_1^2+9t_2^2$ is 6.
\end{proof}
Here is an example for you to try. This one uses the AM-HM inequality.\\
\begin{example}
If the points $(a, 0), (x, y), (0, b)$ are colinear where
$a, b, x, y > 0$ then find the minimum value of $\frac{a}{x}+\frac{b}{y}$?
\end{example}
\begin{theorem}
    [Concurrency of Lines]
    Three lines $a_1x+b_1y+c_1=0; a_2x+b_2y+c_2=0; a_3x+b_3y+c_3=0$ are concurrent if and only if they are non-parallel and $\begin{vmatrix}
        a_1 & b_1 & c_1\\
        a_2 & b_2 & c_2\\
        a_3 & b_3 & c_3\\
    \end{vmatrix}=0$
\end{theorem}
This can also be proven using coordinate geometry.\\
Let's use this to solve a pretty difficult question from JEE advance, which has appeared multiple times in mains as well:\\
\begin{example}
     If the lines $ax + y + 1 = 0$, $x + by + 1 = 0$ and $x + y + c = 0$
where $a, b, c$ being distinct and different from unity, are
concurrent, then the value of $\frac{1}{1-a}+\frac{1}{1-b}+\frac{1}{1-c}$ is
\end{example}
\begin{proof}
    [Solution]
    Using the concurrency of lines conditon:\\
    $\begin{vmatrix}
        a & 1 & 1\\
        1 & b & 1\\
        1 & 1 & c\\
    \end{vmatrix} = 0\\
    \iff \begin{vmatrix}
        a-1 & 1-b & 0\\
        0 & b-1 & 1-c\\
        1 & 1 & c\\
    \end{vmatrix}=0$
Expanding along the first column,\\
$(a-1)[(b-1)c-(1-c)]+(1-b)(1-c)=0\\
\iff (a-1)(b-1)(c)-(a-1)(1-c)+(1-b)(1-c)=0$\\
This seems like a good time to divide by $(a-1)(b-1)(c-1)$\\
$\frac{c}{c-1}+\frac{1}{b-1}+\frac{1}{a-1}=0\\
\iff \frac{c-1+1}{c-1}+\frac{1}{b-1}+\frac{1}{a-1}=0\\
\iff \frac{1}{c-1}++\frac{1}{b-1}+\frac{1}{a-1}-1=0\\
\iff \frac{1}{c-1}+\frac{1}{b-1}+\frac{1}{a-1}=1\\$
\end{proof}
\section{Crammer's Rule}
Have you ever wondered how computers can solve 7-8 simultaneous linear equations in seconds. The answer I am not looking for is 'They are computers, duh!"\\
The answer is using Matrices and Determinants. The matrix version of this is going to come up in a minute, but the determinant version is known as Crammer's Rule over Gabriel Crammer who generalized this for $n$ variables and $n$ equations. We will only get to use it for $2,3$ variables and $2,3$ equations as we don't want to compute determinents of order 4 by hand. Those versions were originally found by Colin Maclaurin  of the Taylor-Maclaurin Series. But he had already stuck his name elsewhere, so he let Crammer have it(also after all Crammer was the one who generalized it)\\
\begin{theorem}
    [Crammer's Rule]
    For the system of liner equations: $a_1x+b_1y+c_1z=d_1;a_2x+b_2y+c_2z=d_2;a_3x+b_3y+c_3z=d_3$, the solution is: $x=\frac{\Delta_x}{\Delta}, y=\frac{\Delta_y}{\Delta}, z=\frac{\Delta_z}{\Delta}$\\
    where $\Delta= \begin{vmatrix}
        a_1 & b_1 & c_1 \\
        a_2 & b_2 & c_2 \\
        a_3 & b_3 & c_3 \\
    \end{vmatrix}$, $\Delta_x = \begin{vmatrix}
        d_1 & b_1 & c_1 \\
        d_2 & b_2 & c_2 \\
        d_3 & b_3 & c_3 \\
    \end{vmatrix}$, $\Delta_y= \begin{vmatrix}
        a_1 & d_1 & c_1 \\
        a_2 & d_2 & c_2 \\
        a_3 & d_3 & c_3 \\
    \end{vmatrix}$, $\Delta_z= \begin{vmatrix}
        a_1 & b_1 & d_1 \\
        a_2 & b_2 & d_2 \\
        a_3 & b_3 & d_3 \\
    \end{vmatrix}$
\end{theorem}
The simplest proof follows from opening the determinants and comparing them to the equation's substitutions.\\
I recommend you trying this out on random systems of equations. But before yo do that here is small life saver:\\
\begin{definition}
    If system has solution (unique or many) than it is called consistent, otherwise it is called inconsistent.\\
    If while using the Crammer's rule $\Delta=\Delta_x=\Delta_y=\Delta_z=0$ then the system has infinite solutions, and if $\Delta=0$ but even one of the rest of the determinants is non-zero, then the system has no solutions.\\
    If $\Delta\neq0$ then it has one unique solution. We need to make sure that all coefficients are not zero in any of the $\Delta=0$ cases.
\end{definition}
This definition may seem like the normal math speak for be careful, but we actually use it quite a lot in questions as well. Here is an example for us to nuke from IIT 2004.\\
\begin{example}
    (IIT 2004) Find k such that , $2x – y + 2z = 2, x – 2y + z = –4, x + y + kz = 4$ has no solution
\end{example}
\begin{proof}
    [Solution]
    For the equation to have no solution, $\Delta=0$\\
    $\therefore \begin{vmatrix}
        2 & -1 & 2\\
        1 & -2 & 1\\
        1 & 1 & k\\
    \end{vmatrix}=0$\\
    Opening the determinant along the third row,\\
    $3-3k=0\\
    \iff k=1$\\
    Technically, we need to check if $\Delta_x, \Delta_y, \Delta_z$ are zero, but knowing that this is an exam with an answer, we can be assured that there is only one solution which is $k=1$
\end{proof}
However, we have yet to consider the case where the system is homogeneous.\\
\begin{definition}
    If the constant terms in the system of equations (i.e. $d_1,d_2,d_3$) are all zero, then system is called homogeneous system of equations
\end{definition}
Here are some more definitions pertaining to homogeneous systems:\\
\begin{theorem}
(1) Homogeneous system is always consistent (as $(0, 0, 0)$
always satisfies it ). 
(2) $(0, 0, 0)$ is also called trivial solution.
(3) Homogeneous system has infinite non-trivial ( i.e. non-zero) solutions if and only if $\Delta = 0$
\end{theorem}
We can use this questions such as:\\
\begin{example}
(IIT 2000)
    If the system of equations
$x – Ky – z = 0\\
Kx – y – z = 0 \\
x + y – z = 0$
Has a non zero solution then $K =$
\end{example}
\begin{proof}
    [Solution]
    We basically want $\begin{vmatrix}
        1 & -k & -1\\
        k & -1 & -1\\
        1 & 1 & -1\\
    \end{vmatrix} =0\\
    \iff -(k+1)-(-1)(1+k)+(-1)(-1+k^2)=0\\
    \iff -(k+1)+(1+k)+(1-k^2)=0\\
    \iff -k-1+1+k+1-k^2=0\\
    \iff k^2=1\\
    \iff k=\pm 1$
\end{proof}
\section{Types Matrices}
If you know determinants, you already know most of matrix.\\
This section is just me telling you all the names you need to know in order to solve ahead.\\
\begin{definition}
    Matix an arrangement of m x n elements in ‘m’ rows and ‘n’ columns.
    $\begin{bmatrix}
        a_{11} & a_{12} & \dots & a_{1n} \\
        a_{21} & a_{22} & \dots & a_{2n} \\
        \vdots & \vdots & \dots & \vdots \\
        a_{m1} & a_{m2} & \dots & a_{mn} \\
    \end{bmatrix}$
\end{definition}
Unlike determinants, it is possible that $m \neq n$. \\
Now let's discuss some special types of matrices.\\
\begin{definition}
    Row Matrix is a matrix having only one row.
    $\begin{bmatrix}
        a & b & c & \dots\\ 
    \end{bmatrix}$
\end{definition}
\begin{definition}
    Column Matrix is a matrix having only one column.\\
    $\begin{bmatrix}
        a\\
        b\\
        c\\
        \vdots\\
    \end{bmatrix}$
\end{definition}
\begin{definition}
    Null matrix or zero is a matrix with all elements 0.\\
    $\begin{bmatrix}
        0 & 0 & \dots & 0 \\
        0 & 0 & \dots & 0 \\
        \vdots & \vdots & \dots & \vdots \\
        0 & 0 & \dots & 0 \\
    \end{bmatrix}$
\end{definition}
\begin{definition}
    Square matrix is a matrix where $m=n$
\end{definition}
In a square matrix, we define the following as well:\\
\begin{definition}
    \begin{enumerate}
        \item $a_{ii}$ are called the diagonal elements\\
        \item $a_{ij}$ and $a_{ji}$ are the conjugate elements\\
        \item $\sum^n_{i=1} a_{ii}$ is called the trace of the matrix\\
    \end{enumerate}
\end{definition}
While the square matrix is in itself very spacial, we define some more special matrix within it as well.\\
\begin{definition}
    A Triangular Matrix is a square matrix with elements on only one side of the diagonal. For example:\\
    $\begin{bmatrix}
        a_{11} & 0 & 0\\
        a_{21} & a_{22} & 0\\
        a_{31} & a_{32} & a_{33}\\
    \end{bmatrix}$
    is an lower triangular matrix while:\\
    $\begin{bmatrix}
        a_{11} & a_{12} & a_{13}\\
        0 & a_{22} & a_{23}\\
        0 & 0 & 0\\
    \end{bmatrix}$
    is an upper triangular matrix.\\
    We need to note that the determinant of a triangular matrix is the product of its diagonal elements. That is $a_{11} \cdot a_{22} \cdot a_{33}$ in the above examples.
\end{definition}
\begin{definition}
    A diagonal matrix is a square matrix with all non-diagonal elements being 0. For example:\\
    $\begin{bmatrix}
        a_{11} & 0 & 0\\
        0 & a_{22} & 0\\
        0 & 0 & a_{33}\\
    \end{bmatrix}$
    is a diagonal matrix. Its determinant is also the product of the diagonal elements.
\end{definition}
\begin{definition}
    A scalar matrix is a diagonal matrix with all elements on the diagonal equal. For example:\\
    $\begin{bmatrix}
        a & 0 & 0 \\
        0 & a & 0\\
        0 & 0 & a\\
    \end{bmatrix}$
    Is a scalar matrix. It's determinant is $a^3$
\end{definition}
\begin{definition}
    The Identity matrix is a scalar matrix with $a=1$. The scaler matrix of order three is:\\
    $\begin{bmatrix}
        1 & 0 & 0 \\
        0 & 1 & 0\\
        0 & 0 & 1\\
    \end{bmatrix}$
\end{definition}
That concludes the first round of definitions.\\
\section{Arithmetic of Matrices}
Two matrices are said to be equal if they have the same number of rows and column(order of matrix) and the corresponding terms are equal in the rows and columns are equal.\\
We can only add matrices of same order. The addition of matrix $A$ and $B$ is just creating a new matrix $C$ where: $c_{ij}=a_{ij}+b_{ij}$\\
Let's solve an example to understand.
\begin{example}
[Motivating Example]
    $\begin{bmatrix}
        3 &9 &10 &3\\
        4 &4 &7 &5\\
    \end{bmatrix}+ \begin{bmatrix}
        10 &6 &8 &10\\
        3 &6 &2 &3\\
    \end{bmatrix}$
\end{example}
\begin{proof}
    [Solution]
    The sum is $\begin{bmatrix}
        3+10 & 9+6 & 10+8 & 3+10\\
        3+4 & 4+6 & 7+2 & 5+3\\
    \end{bmatrix}\\
    = \begin{bmatrix}
        13 & 15 & 18 & 13\\
        7 & 10 & 9 & 8\\
    \end{bmatrix}$
\end{proof}
The multiplication of a matrix by a constant follows as:\\
\begin{definition}
    $K \cdot \begin{bmatrix}
        a_{11} & a_{12} & a_{13}\\
        a_{21} & a_{22} & a_{23}\\
        a_{31} & a_{32} & a_{33}\\
    \end{bmatrix}\\
    = \begin{bmatrix}
        Ka_{11} & Ka_{12} & Ka_{13}\\
        Ka_{21} & Ka_{22} & Ka_{23}\\
        Ka_{31} & Ka_{32} & Ka_{33}\\
    \end{bmatrix}$
\end{definition}
We need to realize that this is different from the determinant multiplication by scalar as that only was multiplied to one row or column, while here $K$ is multiplied to every element. We will now discuss the most important part of matrices, matrix multiplication to matrix.\\
\begin{definition}
    We define matrix multiplication as $A_{m \times n} \times B_{n \times p}= C_{m \times p}$ where $A,B$ and $C$ are matrices.\\
    $A$ is called the pre-multiplier while $B$ is the post multiplier. two Matrices can only be multiplied  if number of columns of
pre-multiplier is equal to number of rows of post multiplier.
\end{definition}
How do we actually do the multiplication? We multiply the rows of the pre multiplier to the columns of the post multiplier.\\
For example $\begin{bmatrix}
    a_{11} & a_{12} \\
    a_{21} & a_{22} \\
    a_{31} & a_{32}\\
\end{bmatrix} \times \begin{bmatrix}
    b_{11} & b_{12} & b_{13}\\
    b_{21} & b_{22} & b_{23}\\
\end{bmatrix} =
    $
    $\begin{bmatrix}
    a_{11} \cdot b_{11} + a_{12} \cdot b_{21} & a_{11} \cdot b_{12} + a_{12} \cdot b_{22} & a_{11} \cdot b_{13} + a_{12} \cdot b_{23} \\
    a_{21} \cdot b_{11} + a_{22} \cdot b_{21} & a_{21} \cdot b_{12} + a_{22} \cdot b_{22} & a_{21} \cdot b_{13} + a_{22} \cdot b_{23} \\
    a_{31} \cdot b_{11} + a_{32} \cdot b_{21} & a_{31} \cdot b_{12} + a_{32} \cdot b_{22} & a_{31} \cdot b_{13} + a_{32} \cdot b_{23} \\
\end{bmatrix}$
Not the prettiest thing, but I hope you can somewhat understand what's going on.  I have given two very simple examples for you to solve to check if you have understood the concept.\\
\begin{example}
    [Motivating Example]
    $\begin{bmatrix}
        2 & 3 & 4\\
        1 & 2 & 3\\
    \end{bmatrix} \times \begin{bmatrix}
        1 & 0 & 1 & 2\\
        2 & 1 & 1 & 2\\
        0 & 1 & -1 & 3\\
    \end{bmatrix}$
\end{example}
\begin{example}
    [Motivating Example]
    $\begin{bmatrix}
         1 & 0 & 2 & 3\\
    \end{bmatrix} \times \begin{bmatrix}
        1 & 0& 1& 2\\
        2 & 1& 1& 2\\
        0 & 1& -1& 3\\
    \end{bmatrix}$
\end{example}
Here are some properties of matrix multiplication:\\
\begin{theorem}
Here $A,B,C$ all represent Matrices whose product is defined
\begin{enumerate}
    \item It is not commutative. In general $AB \neq BA$.\\
    This leads to a peculiar form of $(A+B)^2=A^2+AB+BA+B^2$
    \item It is associative. $(A \times B) \times C= A \times (B \times C)$
    \item It distributes over addition. $A \times (B + C) = A \times B + A \times C$ and $(B + C) \times A = B \times A + C \times A$
    \item The identity matrix $I$ which can multiply $A$ will show the following $I \times A = A \times I = A$\\
    \item Any matrix multiplied with null matrix gives a null matrix. However the the converse is not true. If $A \times B$ is null matrix then it is not necessary that either $A$ or $B$ will be null matrix.\\
    \item Laws of exponents hold as the are:\\ 
    $A^m \times A^n = A^{m+n}\\
    A^n=A^{n-1} \times A\\
    {A^m}^{n}=A^{mn}$
\end{enumerate}
\end{theorem}
A very common question which comes using these properties is:\\
\begin{example}
    If $A = \begin{bmatrix}
        1 & 0\\
        1 & 1\\
    \end{bmatrix}$ then $A^{2023}$ is equal to:
\end{example}
\begin{proof}
    [Solution]
    I'll first show you the solution which a lot of books have. Which while great for competitive speed papers, is not acceptable in written example.\\
    Notice that:\\
    $A= \begin{bmatrix}
        1 & 0\\
        1 & 1\\
    \end{bmatrix}$\\
    $A^2= \begin{bmatrix}
        1 & 0\\
        1 & 1\\
    \end{bmatrix} \times \begin{bmatrix}
        1 & 0\\
        1 & 1\\
    \end{bmatrix} = \begin{bmatrix}
        1 & 0\\
        2 & 1\\
    \end{bmatrix}$\\
    This establishes a pattern, which means $A^{2023}=\begin{bmatrix}
        1 & 0\\
        2023 & 1\\
    \end{bmatrix}$\\
    You can see that this is a very bad explanation. The actul way to do it is to prove that $A^n=\begin{bmatrix}
        1 & 0\\
        n & 1\\
    \end{bmatrix}$ using Induction. The first one was engineers induction(see Power Overwhelming), explains quite well why such questions occur in engineering entrance exams the most.\\
\end{proof}
\section{The bridge}
We will now start the process of connecting Matrices with determinants.\\
We already know that we can  find determinant of a square matrix. Using that we can claim the following:\\
\begin{theorem}
     If A and B are two square matrices of same order then $|A \times B| =|A| \times |B|$
\end{theorem}
Also if we remember the multiplication of matrices and determinants by a constant and more specifically their one difference, we can also say:\\
\begin{theorem}
If $A_n$ is a square matrix of order $n$ and $K$ is a constant then:
$|KA_n| = K^n|A_n|$
\end{theorem}
You may also remember the transposing of determinant. We'll define it formally here.
\begin{definition}
    Matrix obtained by interchanging rows and columns is called transpose of matrix, for some matrix $A$ it is denoted by $A^T$.
    $A = \begin{bmatrix}
        a_{11} &a_{12} &a_{13} \\
        a_{21} &a_{22} &a_{23} \\
        a_{31} &a_{32} &a_{33} \\
    \end{bmatrix}$\\
    $\therefore A^T= \begin{bmatrix}
        a_{11} &a_{21} &a_{31} \\
        a_{12} &a_{22} &a_{32} \\
        a_{13} &a_{23} &a_{33}
    \end{bmatrix}$
\end{definition}
Using the definition we can also notice the following facts:\\
\begin{theorem}
    \begin{enumerate}
        \item ${(A^T)}^T=A$\\
        \item $(A+B)=A^T + B^T\\$
        \item ${(KA)}^T=K(A^T)$ where $K$ is a constant\\
        \item ${(AB)}^T=B^TA^T$\\
        \item ${(ABC)^T}=C^TB^TA^T$\\
        \item ${(A^n)}^T={(A^T)}^n$
    \end{enumerate}
\end{theorem}
\section{Some more special matrices}
Now that we have defined transpose, we can define some more special matrices on the basis of it.\\
\begin{definition}
    Symmetric matrix: If ${(A_n)}^T=A_n$  for a square matrix $A_n$ then it is called a symmetric matrix. Basically, $a_{ij}=a_{ji}$
\end{definition}
\begin{definition}
    Skew Symmetric Matrix: If ${(A_n)}^T=-A_n$  for a square matrix $A_n$ then it is called a skew symmetric matrix. Basically, $a_{ij}=-a_{ji}$
\end{definition}
With this much, we can start proving almost surprising facts:\\
\begin{example}
    Prove that for any square matrix $A$, $\frac{1}{2}(A+A^T)$ is symmetric matrix and $\frac{1}{2}(A-A^T)$ is skew symmetric matrix.
\end{example}
\begin{proof}
    We just take the transpose.\\
    ${(\frac{1}{2}(A+A^T))}^T\\
    = \frac{1}{2}(A^T+A)$ which makes it symmetric.\\
    ${(\frac{1}{2}(A-A^T))}^T\\
    = \frac{1}{2}(A-A^T)\\
    = \frac{-1}{2}(A^T-A)$ which makes it skew-symmetric.\\
\end{proof}
This also leads to a surprising fact: Every square matrix A can be represented as a sum of symmetric and skew symmetric matrix. Here is an example for you to solve\\
\begin{example}
 If $A$ and $B$ are symmetric matrices of same order then prove that $AB- BA$ is skew symmetric matrix.
\end{example}
\begin{definition}
    A square matrix is orthogonal if $AA^T=I$\\
    We can note that the determinant of $A$ must be $\pm 1$\\
    Expanding the multiplication gives us:  the sum of squares of elements in any row or column is one and the pairwise product and sum of two rows or columns is zero.\\
\end{definition}
The last one allows us to detect Orthogonal matrices in the wild. For example in this Question from IIT.\\
\begin{example}
    (IIT 2005) Find $P^TQ^{2005}P$, where $P=\begin{bmatrix}
        \frac{\sqrt{3}}{2} & \frac{1}{2} \\
        \frac{-1}{2} & \frac{\sqrt{3}}{2}\\
    \end{bmatrix}, A= \begin{bmatrix}
        1 & 1\\
        0 & 1\\
    \end{bmatrix}$ and $Q=PAP^T$
\end{example}
We need to only notice that $P$ is an orthogonal matrix. After that, the question basically solves itself.\\
\begin{definition}
    A square matrix is called idempotent if $A^2= A$. Clearly, $A^n$ will also be equal to $A$ for all $n \geq 2$
\end{definition}
At this time I think it is also important for you to know that $(A+I)^n$ can be opened like a binomial expansion. This can only be done for $(A+I)$ not for $(A+B)$. This is because $A \times I = I \times A = A$\\
And here I ask you to ponder before reading the solution:\\
\begin{example}
If A is an idempotent matrix then $(I+A)^n=$
\end{example}
\begin{proof}
    [Solution]
    Expanding using binomial theorem, $I^n+\binom{n}{1}I^{n-1}A+\dots + A^n\\
    = I+(2^n-1)A$
    Using the fact that $I^n=I$ and $A \times I=A$ and $\sum^n_{n=1}\binom{n}{0}=2^n-1$. The last one as you may recall comes from combinitorics.\\
\end{proof}
And we end this section with a few miscellaneous definitions.
\begin{definition}
    A square matrix is called involutory if $A^2= I$
\end{definition}
\begin{definition}
    A square matrix is called nilpotent matrix of order
m if:\\
$A^m=0$ and $A^{m-1} \neq 0$\\
Currently, there is no way to check whether matrix is nilpotent or not, other than checking the powers manually.
\end{definition}
\begin{definition}
    A matrix is called singular if its determinant is zero, otherwise it is called non-singular.\\
\end{definition}
\section{Adjoint and Inverse of Matrices}
Remember the co-factor matrix we had studied earlier? We'll use it in a minute to find inverse of a matrix, the one thing for which we have literally studied matrix for.\\
But first let's talk about adjoint of a matrix.\\
\begin{definition}
    For any square matrix, its adjoint is defined as transpose of its cofactor matrix.
\end{definition}
Using whatever we know about the co-factor matrix and about transpositions, we'll get at the following properties of adjoint matrices:\\
\begin{theorem}
    For square matrices $A$ and $B$ of order $n$, we have:\\
    \begin{enumerate}
        \item $|adj A|= |A|^{n-1}$\\
        \item $adj(adj A)= |A|^{n-2}A$\\
        \item $adj(A^T)= (adj A)^T$\\
        \item $adj(KA)= K^{n-1}(adj A)$\\
        \item $adj(A^n)=(adj A)^n$\\
        \item $adj(AB)=(adj B)(adj A)$
    \end{enumerate}
\end{theorem}
Here is the reason why we learnt about the adjoint:\\
\begin{definition}
Square matrix $B_n$ is called inverse matrix of $A_n$ if:
$AB = BA = I$\\
Clearly, if $B$ is inverse of $A$ then $A$ is also inverse of $B$.
Formula for $A^{-1}= \frac{1}{|A|} \times adj(A)$\\
Clearly, this makes singuler matrices have no inverses.\\
\end{definition}
At this point I recommend you taking the inverse of a random $3 \times 3$ matrix. While, we have done all the operations and transformations previously, doing it only once will give you some amount of confidence in what to do.\\
Here are some properties of the inverse:\\
\begin{theorem}
    \begin{enumerate}
        \item $|A^{-1}|=\frac{1}{|A|}$\\
        \item ${(A^T)}^{-1}={(A^{-1})}^T$
        \item $adj(A^{-1})={(adj A)}^{-1}=\frac{A}{|A|}$\\
        \item $(AB)^{-1}=B^{-1}A^{-1}$
    \end{enumerate}
\end{theorem}
These properties wreck questions such as:\\
\begin{example}
    (JEE Mains 2014) If $A$ is a $3\times3$ non-singular matrix such that $AA^T = A^TA$ and $B = A^{–1} A^T$ then $BB^T=$
\end{example}
\begin{proof}
    [Solution]
    This question is perfect as it uses a good number of the properties we discussed.\\
    $BB^T\\
    = (A^{-1}A^T){(A^{-1}A^T)}^T\\
    = A^{-1}A^T{(A^T)}^T{(A^{-1})}^T\\
    = A^{-1}(A^T{(A^T)}^T){(A^{-1})}^T\\
    = A^{-1}(A^TA){(A^T)}^{-1}\\
    = A^{-1}I{(A^T)}^{-1}\\
    = IA^{-1}{(A^T)}^{-1}\\
    = I^2\\
    = I$
\end{proof}
\section{System of linear Equations using Matrices}
Now we finally see the back end of Crammer's Rule.\\
But before that let's go on a little tangent.\\
Suppose we have three matrices $A,X,B$ where $A$ and $X$ are multiply-able and $A$ is inveritible(has an inverse, or is non-singular)\\
If $AX=B$ then can we just divide both sides by $A$? Obviously not. We can't truly divide matrices. But we can instead premultiply both the sides by $A^{-1}$ to get $A^{-1}AX=A^{-1}B$ which simplifies to $IX=A^{-1}B \iff X=A^{-1}B$\\
We need to note this is not the same as dividing as the order of the multiplication matters. If we, in confusion, take $X=BA^{-1}$, we will get a wrong answer. But how is all this related?
We need to notice that for a system of equations:\\
$a_1x+b_1y+c_1z=d_1\\
a_2x+b_2y+c_2z=d_2\\
a_3x+b_3y+c_3z=d_3$\\
Is in all ways and forms equivalent to $\begin{bmatrix}
    a_1 & b_1 & c_1\\
    a_2 & b_2 & c_2\\
    a_3 & b_3 & c_3\\
\end{bmatrix} \times \begin{bmatrix}
    x\\
    y\\
    z\\
\end{bmatrix} = \begin{bmatrix}
    d_1\\
    d_2\\
    d_3\\
\end{bmatrix}$\\
This is exactly the case we just discussed above. While we can use inverse to find the system of solutions and in school exams you have to do that(Don't for the love of god use Crammer's rule here, you will lose marks). However, competitively, Crammer's rule is much quicker form of doing the same.\\
Also to be complete here are the solution conditions for the inverse matrix form.\\
\begin{theorem}
Let's denote $\begin{bmatrix}
    a_1 & b_1 & c_1\\
    a_2 & b_2 & c_2\\
    a_3 & b_3 & c_3\\
\end{bmatrix}$ as $A$ and $\begin{bmatrix}
    d_1\\
    d_2\\
    d_3\\
\end{bmatrix}$ as $B$
    If $|A|\neq 0$, we have one unique solution.\\
    If $|A|=0$ and $(adj A) \times B=0$ then infinitely many solutions provided that all coefficients are not $0$\\
    If $|A|=0$ and $(adj A) \times B \neq 0$ then no solutions exist.
\end{theorem}
\section{Characteristic Equation and Cayley Hamilton Theorem}
\begin{definition}
 If A is any square matrix, then $|A –xI| = 0$ is called its characteristic equation. Roots of Characteristic Equation are called as Eigen values or Characteristic roots
\end{definition}
For example the characteristic equation $A=\begin{bmatrix}
    8 &6\\
    5 &9
\end{bmatrix}$ is:\\
$|A-xI|=0\\
\iff \begin{vmatrix}
    8-x &6\\
    5 & 9-x
\end{vmatrix} =0\\
\iff (8-x)(9-x)-30=0\\
\iff (72-17x+x^2)-30=0\\
\iff x^2-17x+42=0$
is the characteristic equation of A. \\
Like recursion, this is called the characteristic equation as it is satisfied by only and only $A$. In this case it means,  $A^2-17A+42I=0$\\
The fact that every matrix satisfies it's characteristic equation is known as Cayley-Hamiltonian theorem. We can observe from our solving and using vieta, the sum of its eigen values is equal to the trace and the product of the eigen values is equal to the determinant.\\
Also using the properties of $A^{-1}$ we can say that if $\lambda$ is an eigen value of $A$ then $\frac{1}{\lambda}$ is an eigan value of $A^{-1}$\\
All this is interesting, but what is the use?\\
Glad you asked:\\
\begin{example}
    If $A= \begin{bmatrix}
        1 &0 &2\\
        1 &2 &1\\
        2 &0 &3\\
    \end{bmatrix}$ then find $k$ such that $A^3 - kA^2 + 7A +2I=0$\\
\end{example}
\begin{proof}
    [Solution]
    Let's find the characteristic equation. \\
    $\begin{bmatrix}
        1-x &0 &2\\
        0 &2-x &1\\
        2 &0 &3-x\\
        \end{bmatrix}=0\\
        \iff (2-x)(3-x)(1-x)-4(2-x)=0\\
        \iff -x^3+6x^2-7x-2=0\\
        \implies A^3-6A^2+7A+2I=0$\\
        Thus, $k=6$
\end{proof}
\section{Netflix and Spotify and Matrices...}
Okay, so you know when Netflix is like, 'Hey, watch this show!' or when Spotify suggests your next favorite jam? Well, behind the scenes, there's some linear algebra happening. Netflix has a huge matrix where rows are people and columns are movies or shows. Each cell is like a 'how much they like it' score. \\
Using inverses and multiplication, They break this table into two smaller tables tables – one for people's taste (call it 'U' for users. One is linked with every account) and one for awesomeness of a show (we'll call it 'V' for value, one is linked with every movie). \\
If Netflix wants to find the right show for the right user so they multiply your $U$ with the shows $V$ and takes the determinant. The higher the value, the better the fit is. This might seem simple in speaking but under the hood this is a very complicated algorithm which whose nitty and gritty are better suited for a computing book. Also a lot of it is proprietary, or top secret, so it's not possible for us to know everything but this is the gist of it.\\
This is exactly how websites are ranked by assigning a matrix to the search quarry and a matrix to the site and then multiplying and taking determinant. We also use it in geometry and physics as we'll see later\\
\begin{xcb}{Exercises}
\begin{enumerate}
    \item $\begin{vmatrix}
    \sin(2x) & 1-\cos(2x) & 2\sin(x) \\
    \cos(x) & \sin(x) & 1 \\
    \sin(x) & \cos(x) & 1 \\
\end{vmatrix}$\\
\item (JEE Mains 2020)  Let $A = [a_{ij}]$ and $B = [b_{ij}]$ be two $3 \times 3$ real matrices such that $b_{ij} = 3^{i + j -2}a_{ji}$, where $i, j = 1, 2, 3$. If the determinant of B is $81$, then the determinant of A is: \\
\item If $\Delta_r=\begin{vmatrix}
    4 & 612 & 915\\
    101r^2 & 2r & 3r\\
    r & \frac{1}{r}& \frac{1}{r^2}
\end{vmatrix}$ then the value of $\lim_{n \to \infty} \frac{1}{n^3} \sum^{n}_{r=1}\Delta_r$\\
\item If $s=(a+b+c)$, then the value of $\begin{vmatrix}
    s+c & a & b \\
    c & s+a & b \\
    c & a & s+b \\
\end{vmatrix}$ is(in terms of $s$):\\
\item (IIT 2011) The Real roots of $\begin{vmatrix}
    \sin(x) & \cos(x) & \cos(x) \\
    \cos(x) & \sin(x) & \cos(x) \\
    \cos(x) & \cos(x) & \sin(x) \\
\end{vmatrix}$ in the interval $\frac{-\pi}{4}\leq x \leq \frac{\pi}{4}$ is(are):
\item (JEE Mains 2020) Let $a-2b+c=1$, If $f(x)=\begin{vmatrix}
    x+a & x+2 & x+1 \\
    x+b & x+3 & x+2 \\
    x+c & x+4 & x+3 \\
\end{vmatrix},$ then find the algebraic form of $f(x)$ \\
\item Find values of $p,q$ such that $x + y + z = 6; 2x + 5y + pz = q; x + 2y + 3z = 14$\\
(a) has unique solution\\
(b) has infinitely many solutions
\item If ‘t’ is real and $\lambda = \frac{t^2-3t+4}{t^2+3t+4}$ then find number of solution of
$3x – y + 4z = 3\\
x + 2y – 3z = – 2\\
6x + 5y + \lambda z = – 3$ 
\item (JEE Mains 2020) The system of equation $3x + 4y + 5z = \mu, x + 2y + 3z = 1,
4x + 4y + 4z = \delta$ is inconsistent, then $(\delta, \mu)$ can be 
\item For matrices $A,B$, If $AB = A$ and $BA = B$ then $B^2=$\\
\item If $A$ and $B$ are square matrices of order $3$ such that $|A| = -1, |B| = 3$, then the determinant of $2A^3B^2$ is equal to:
\item If P is a $3\times3$ matrix such that $P^T = 2P + I$ then prove that $P + I = 0$
\item Let $A$ be the set of all $3 \times 3$ matrices which are symmetric with entries $0$ or $1$. If there are five $1$’s and
four $0$’s, then number of matrices in $A$ is:\\
\item If $A= \begin{bmatrix}
    1 & 2 & 3\\
    3 & -2 & 1\\
    4 & 2 & 1\\
\end{bmatrix}$, then find $K$ such that $A^3-kA-40I=0$
\item (Poh-Shen Loh) Calculate the determinant of $\begin{vmatrix}
    1 & 2& 3& 4& 5& 6& 7\\
    2 & 3& 4& 5& 6& 7& 8\\
    1 & 1& 1& 1& 1& 1& 1\\
    1 &5 &3 &8 &1 &9 &9\\6 &5 &1 &1 &6 &6 &4\\1 &1 &3 &3 &8 &5 &6\\3 &2 &7 &8 &9 &9 &8
\end{vmatrix}$
\end{enumerate}
\end{xcb}
\input{4redpill/ineqrev.tex}

\part{Number Theory}
\input{5nt/ntintro.tex}
\input{5nt/modarith.tex}
\input{5nt/funceq.tex}
\input{5nt/dioeq.tex}

\part{The Number's Awaken}
\input{6ntadv/bazooka.tex}
\input{6ntadv/const.tex}

\backmatter

\appendix
\part{Appendix}
\input{Appendix/hints.tex}
\chapter{Borrowed Brilliance}
\label{app: borrow}

As I have mentioned many times, this book is \cancel{steals} borrows brilliance from a lot of people who are frankly much more talented and accomplished than me. It is impotent that we pay tribute to them somewhere.
\section{Almost every part}
\begin{enumerate}
    \item \emph{Art of Problem-Solving} website and forums: A lot of the theory, questions and their solutions were sourced from AOPS. If I even just mention the people whose work I directly used, this manuscript would get doubled. I am indebted beyond measure to the entire community.
    \item \emph{Brilliant.org} website: The website hosts a lot of free pages for different theorems and their exposition. I have taken a lot of inspiration from there along with questions. They could have easily kept these for the paid subscribers, but we can clearly see that they care more about math than money.
    \item \emph{Math Stack Exchange} forums: It is the reason this book was written in this lifetime. I am a very confused person, if the community there hadn't clarified the dumbest of my doubts, we wouldn't have this book. Also a lot of question and answers were sourced from here.
    \item \emph{Latex Stack Exchange} forums: It is the reason why this looks like a book and not a jumbled mess of words. It pains me that I can't feasibly mention everyone who helped by their username.
    \item \emph{Overleaf} latex editor: I didn't install Linux and then a latex editor and compiler and pdf viewer and what not to my computer. I used Overleaf as it was more organized, less cumbersome and easier to share and edit. All the die-hard Linux lovers are gonna lose their marbles. Guys, I don't know know how to do the installation and I will not learn until I absolutely need to.
    \item \emph{Math problem book} template by MAA: Used to give the book the 'book feel'.
    \item \emph{Random Hints} code by Evan Chan: The reason why we have the elegant random hints. It was the easiest one to use within overleaf.
\end{enumerate}
\section{Introductory Problems}
\begin{enumerate}
    \item \emph{Mathematical Circles: The Russian Experience} book by Dmitrii Vladimirovich Fomin, Ilia Itenberg, and S. Genkin: A classic book with a number of good questions. A must read for all primary school teachers.
    \item \emph{The USSR Olympiad problem book} Book by D. O. Shkliarskii: This book has a lot of interesting questions. Which require nothing more than the mind to solve. A clear reason why the Russian Olympiads are so fabled
    \item \emph{Friendship over Tea} video and problem by Arvind Gupta: I had decided while writing this book that I'll include Arvind Gupta sir at least once. He is such a legend. The way he has brought science and design education into the poorest parts of the world using toys from trash is commendable. Check him out, you'll not regret it. His Ted Talk was ranked as 2nd in the 5 favorite education talks.
    \item \emph{The Green Eyed Dragon and Other Mathematical Monsters} by David Morrin: A very fun book. I have long said to people that once a child turns of listening age, read a single problem from this every night and they'll develop the most wonderful sense of math and logic.
    \item {Counterfeit Coin Riddle} video by Jennifer Lu(Ted-Ed): Ted-Ed creates these beautiful animation videos explaining science concepts. They have entire series on world mythology, problems and basics of coding. In hostel, sometimes late at night, a bunch of me and my friends used to sit and see Ted-ed all night. We are all quite successful in math Olympiads, and ones watching something else all night, not quite...
    \item {The OTIS Excerpts} book by Evan Chen: Although it is borderline promotional material for his paid course, the book is quite educational.
\end{enumerate}
\section{Permutations and Combinations}
\begin{enumerate}
    \item \emph{Mastering AMC 10/12} by Sohil Rathee: A lot of the AMC 10 and 12 questions were sourced from the book. This is especially true for the part 1 of combinitorics. Sohil Rathi is an angel for going through the long history of AMC and painstakingly choosing the questions, and then putting it out for free.
    \item  \emph{Murderous Math: The Perfect Sausage and other fundamental formulas} book by Kjartan Poskitt: The muderous math series was some of first the non-academic math I had studied. A major influence on how I present math. 
    \item \emph{Introduction to Counting and Probability} handout by David Altzio: This handout in 2013-14 Math League gave me the idea and quite a few of the problems for The Guessing Game.
    \item  \emph{Permutation and Combination for IOQM 2023} lecture series by Abhay Mahajan(Vedantu Olympiad School): Some of the best lectures on combinitorics I have watched. No boubt would be much more popular if they were in English, unfortunately, they are in Hindi, so only fraction of you can enjoy them.
    \item \emph{Yale Putnam Handouts} by Pat Devlin: The long list handouts made by Pat for the Yale Putnam students was of enoumous use in sourcing some of the more difficult problems.
    \item \emph{Stanford Putnam Handouts} by Ravi Vakil: A legend of the math comunity, Ravi Vakil's handouts have been used all through the book, including in these for the 9 star problems.
\end{enumerate}
\section{Down The Rabbit Hole}
\begin{enumerate}
    \item \emph{Recursion in AIME} handout by Dylan Yu et al.: This handout, which is part of the Euclid's Orchid handouts, was a source for lot of the recursion theory and problems.
    \item \emph{Recurrence Relations} lecture by Prashant Jain: A great introductory video with a lot of classic examples. Sadly, its in Hindi, making it less accessible.
    \item \emph{Recurrence for INMO basics} lecture by Abhay Mahaajan: Abhay sir's question picking genius shines here. Every question is slightly harder and before you realize we are from AMC to IMO.
    \item \emph{IOQM 2022 practice sessions PnC} lecture by Prashant Jain: Prashant sir had introduced me to Catlan numbers here. The example over there is from Prashant sir's class.
    \item \emph{Bijections} by Yufei Zhao: The Catlan numbers were expanded upon using Yufai's handout.
    \item \emph{Counting in Two Ways} handout by Yufei Zhao: The idea of incidence matrix was completely taken from here.
    \item \emph{Introduction to Graph Theory} handout and powerpoint by Irene Lo: This 2019 Berkeley Math Circle handout was the basis for majority of the graph theory chapter.
    \item \emph{Olympiad Graph Theory} handout by Adam Kelly: This handout was mainly used for the questions in the graph theory chapter.
\end{enumerate}
\section{Algebra}
\begin{enumerate}
    \item \emph{Polynomials in AIME} handout by Dylan Lu et al.: Another great handout in the Euclid's orchid collection. Worth its weight in gold.
    \item \emph{Sequences and Series in the AMC and AIME} handout by Dylan Lu et al.: Also from the Euclid's orchid collection. Dylan and his gang have really great material on algebra.
    \item \emph{Series and Sequences} by David Altizio: Used mainly as a question source.
    \item \emph{A Brief Introduction to Olympiad Inequalities} handout by Evan Chen: This was the main reference for both of the inequalities chapters
    \item \emph{Inequalities} handout by Ananth Shyamal, Divya Shyamal, Kevin Yang, and Reece Yang: This was a handout for the Iowa City Math Circle. If any of the readers is in Iowa, I recommend visiting these peeps.
    \item \emph{Inequalities} handout by Dimitar Grantcharov: The handout made for Berkeley Math Circle while lacking in theory has a bunch of great questions.
\end{enumerate}
\section{The Red Pill}
\begin{enumerate}
    \item \emph{This Is the Calculus They Won't Teach You} video by A Well Rested Dog: This YouTube video made during the first Summer of Math Exposition talks about the history of calculus and was referenced as source if historical context.
    \item \emph{The Cartoon Guide to Calculus} book by Larry Gonick: This book was what I had used to study calculus and is my first recommendation to those learning it. A lot of the graphs and analogies were taken from this book.\\
    \item \emph{Limits and Continuity of Function} Lecture by Prashant Jain: This lecture, part of Bounce back series on Unacademy Atoms YouTube channel, was binged by me on a bus trip home. I have been able to solve some the most difficult questions of limits since then.
    \item \emph{Limits} lecture by Abhay Mahajan: These lectures(on the Vedantu JEE made EJEE channel) were the source of many of the questions\\
    \item \emph{Diffrentiation and Continuity} lecture by Abhay Mahajan: Again, used mainly for questions.
    \item \emph{Indefinite Integration} lecture by Abhay Mahajan: These lectures were the basis of most of the integration by substitution in the integration chapter. 
    \item \emph{DI method} videos by Steve Chow(Blackpen Redpen): The DI Method was first introduced to me by Steve Chow. His other videos on competition math, calculus and math for fun are just a delight to watch.
    \item \emph{Differential Equations} lecture notes by Nikenasih Binatari of State University of Yogyakarta: While the notes were for the semester course in differential equations, the first few chapters were directly referenced for the section.
    \item  \emph{Matrices and Determinents} lecture by Abhay Mahajan: These lectures were the basis for most of the linear algebra chapter.
    \item \emph{Lagrange Multipliers} videos by Khan Academy: The basis for the Lagrange multipliers section. There explanation was the simplest to understand and most straightforward, which allowed me to easily integrate it into the book.
    \item \emph{Lagrange Murderpliers Done Correctly} handout by Evan Chen: Most of the Lagrange inequalities were sourced from this handout.
    \item \emph{Sum Uses of Calculus} by David Altizio: The basis of the summation section. Parts of it were removed as they were too complex, but its overall a great handout.
\end{enumerate}
\section{Number Theory}
\begin{enumerate}
    \item \emph{Olypiad Number Thoery Through Challenging Problems} book by Justin Stevens: One of the most used pieces of reference for the entire number theory part. Great explanation, even better questions. Stevens is doing all of a favour by making the text available for free.
    \item \emph{Mordern Olympiad Number Theory} book by Aditya Khurmi: Another major reference for the Number Theory part. Covers some of the most complex number theory concepts in a digestible manner. Khurmi has given a gift by making this book free on AOPS. The book is so detailed and so beautiful that I had to literally fight myself for every complex concept I wanted to add.
    \item \emph{A Decade of the Berkeley Math Circle Volume I} book by Zvezdelina Stankova and Tom Rike: The session $4$ of this book was the inspiration for the name and the introductory passage for the first chapter of number theory. The general format of the number theory part is also inspired from here.
    \item \emph{Prime Numbers, Factors, and Division Tricks} handout by Linda Green: While more of her handouts occur in The Numbers Awaken, all of them have very creative examples and increase the question level gradually. Probably the best BMC handouts on number theory.
    \item \emph{Introduction to Modular Arithmetic} by David Altzio: It is so strange that David either gives me very complex questions or introductory questions. This one was had introductory questions.
    \item \emph{Bases Part I-II} problem set by Pratima Karpe: The source of almost all of the base questions.
    \item \emph{Putnam Problem Solving Seminar: Number Theory} handouts by Ravi Vakil: These handouts by Ravi Vakil from the years $2002, 2003, 2004, 2006$ was the source of the Putnam or Putnam adjacent problems.
    \item \emph{Functional equations} handout by Maxim Li: The only functional equation handout which I was able to understand in the first read. If it were not for this handout, I would myself never have understood functional equations in the first place.
    \item \emph{Functional Equations $\&$ Recurrence Relations} handout by Ted Alper: While I didn't use the Reccurence relations, or the analogy between functional equation and recurrence relations, I did use it for the purpose of getting introductory questions.
    \item \emph{Functional Equations} handout by Igor Ganichev: This was literally a list of problems. I solved all of them and used the ones I particularly found instructive.
    \item \emph{Putnam Problem Solving Seminar: Functional Equations} handouts by Vin De Silva, Mark Lucianovic and Ravi Vakil: This was also from the same seminars in the years $2005, 2006$. Again, they are source of the Putnam or Putnam adjacent problems.
    \item \emph{Introduction to Functional Equations} handout by Evan Chen: I really want to know want to know what Evan means by "Introduction" because this was by far the most complex reference. However, it had a lot of good examples and problems.
    \item \emph{Diophantine Equations} by David Altzio: This one is again somewhat on the easier side. Great for the introductory problems. However, I think David doesn't like NT much.
\end{enumerate}
\section{The Numbers Awaken}
\begin{enumerate}
\item \emph{Sledgehammers in number theory} by CJ Quines: The main reason the Bazooka chapter exists. Lovely handout!
\item \emph{Order's Modulo a Prime} by Evan Chen: While I complained about David's number thoery handouts being easy, I'll admit that Evan's handouts required more focus than boiling water with your minds.\\
\end{enumerate}
\chapter{Appendix E: Problem Sources}
This chapter talks more about the various exams we have \cancel{copied and pasted} sourced from. Note, individual authors and books have been not mentioned here. They are in Appendix F.\\
\begin{itemize}
    \item AIME - American Invitatitional Math Contest
    \item ARML - American Regions Mathematics League
    \item AMC 8 - American Math Contest 8
    \item AMC 10 - American Math Contest 10
    \item AMC 12 - American Math Contest 12
    \item Cayley - Centre for Education in Mathematics and Computing Cayley Math Contest
    \item China - Chine Team Selection Test
    \item Canada - Canada Team Selection Test
    \item Fermat - Centre for Education in Mathematics and Computing Fermat Math Contest
    \item IOQM - Indian Olympiad Qualifiers for Mathematics
    \item IMO- International Math Olympiad
    \item IMOSL - International Math Olympiad Short list
    \item IrMO - Irish Math Olympiad
    \item ISI - Indian Statistical Institute Admission test
    \item Italy - Italy Team Selection Test
    \item Pascal - Centre for Education in Mathematics and Computing Pascal Math Contest
    \item PROMYS - Program in Mathematics for Young Scientists application form
    \item Purple Comet - Purple Comet! Math Meet
    \item Putnam - William Lowell Putnam Mathematical Competition
    \item USAMO - United States of America Mathematical Olympiad
\end{itemize}
\clearpage
\end{document}
